\documentclass{article}
\usepackage{pgf}
\usepackage{lmodern}
\usepackage{graphicx}
\usepackage{subcaption}  % For subfigures if needed

\title{Data-Driven Sampled-Data LQR Results}
\author{Your Name}
\date{\today}

\begin{document}

\maketitle

\section{Introduction}

This document demonstrates how to include the PGF/TikZ plots generated from the simulation results.

\section{Results}

\subsection{State Trajectories}

Figure~\ref{fig:states} shows the state trajectories of the cart-pole system under DD-SDLQR control.

\begin{figure}[htbp]
  \centering
  %% Creator: Matplotlib, PGF backend
%%
%% To include the figure in your LaTeX document, write
%%   \input{<filename>.pgf}
%%
%% Make sure the required packages are loaded in your preamble
%%   \usepackage{pgf}
%%
%% Also ensure that all the required font packages are loaded; for instance,
%% the lmodern package is sometimes necessary when using math font.
%%   \usepackage{lmodern}
%%
%% Figures using additional raster images can only be included by \input if
%% they are in the same directory as the main LaTeX file. For loading figures
%% from other directories you can use the `import` package
%%   \usepackage{import}
%%
%% and then include the figures with
%%   \import{<path to file>}{<filename>.pgf}
%%
%% Matplotlib used the following preamble
%%   
%%   \makeatletter\@ifpackageloaded{underscore}{}{\usepackage[strings]{underscore}}\makeatother
%%
\begingroup%
\makeatletter%
\begin{pgfpicture}%
\pgfpathrectangle{\pgfpointorigin}{\pgfqpoint{6.000000in}{4.000000in}}%
\pgfusepath{use as bounding box, clip}%
\begin{pgfscope}%
\pgfsetbuttcap%
\pgfsetmiterjoin%
\definecolor{currentfill}{rgb}{1.000000,1.000000,1.000000}%
\pgfsetfillcolor{currentfill}%
\pgfsetlinewidth{0.000000pt}%
\definecolor{currentstroke}{rgb}{1.000000,1.000000,1.000000}%
\pgfsetstrokecolor{currentstroke}%
\pgfsetdash{}{0pt}%
\pgfpathmoveto{\pgfqpoint{0.000000in}{0.000000in}}%
\pgfpathlineto{\pgfqpoint{6.000000in}{0.000000in}}%
\pgfpathlineto{\pgfqpoint{6.000000in}{4.000000in}}%
\pgfpathlineto{\pgfqpoint{0.000000in}{4.000000in}}%
\pgfpathlineto{\pgfqpoint{0.000000in}{0.000000in}}%
\pgfpathclose%
\pgfusepath{fill}%
\end{pgfscope}%
\begin{pgfscope}%
\pgfsetbuttcap%
\pgfsetmiterjoin%
\definecolor{currentfill}{rgb}{1.000000,1.000000,1.000000}%
\pgfsetfillcolor{currentfill}%
\pgfsetlinewidth{0.000000pt}%
\definecolor{currentstroke}{rgb}{0.000000,0.000000,0.000000}%
\pgfsetstrokecolor{currentstroke}%
\pgfsetstrokeopacity{0.000000}%
\pgfsetdash{}{0pt}%
\pgfpathmoveto{\pgfqpoint{0.591528in}{0.585278in}}%
\pgfpathlineto{\pgfqpoint{5.780625in}{0.585278in}}%
\pgfpathlineto{\pgfqpoint{5.780625in}{3.850000in}}%
\pgfpathlineto{\pgfqpoint{0.591528in}{3.850000in}}%
\pgfpathlineto{\pgfqpoint{0.591528in}{0.585278in}}%
\pgfpathclose%
\pgfusepath{fill}%
\end{pgfscope}%
\begin{pgfscope}%
\pgfpathrectangle{\pgfqpoint{0.591528in}{0.585278in}}{\pgfqpoint{5.189097in}{3.264722in}}%
\pgfusepath{clip}%
\pgfsetrectcap%
\pgfsetroundjoin%
\pgfsetlinewidth{0.803000pt}%
\definecolor{currentstroke}{rgb}{0.690196,0.690196,0.690196}%
\pgfsetstrokecolor{currentstroke}%
\pgfsetdash{}{0pt}%
\pgfpathmoveto{\pgfqpoint{0.591528in}{0.585278in}}%
\pgfpathlineto{\pgfqpoint{0.591528in}{3.850000in}}%
\pgfusepath{stroke}%
\end{pgfscope}%
\begin{pgfscope}%
\pgfsetbuttcap%
\pgfsetroundjoin%
\definecolor{currentfill}{rgb}{0.000000,0.000000,0.000000}%
\pgfsetfillcolor{currentfill}%
\pgfsetlinewidth{0.803000pt}%
\definecolor{currentstroke}{rgb}{0.000000,0.000000,0.000000}%
\pgfsetstrokecolor{currentstroke}%
\pgfsetdash{}{0pt}%
\pgfsys@defobject{currentmarker}{\pgfqpoint{0.000000in}{-0.048611in}}{\pgfqpoint{0.000000in}{0.000000in}}{%
\pgfpathmoveto{\pgfqpoint{0.000000in}{0.000000in}}%
\pgfpathlineto{\pgfqpoint{0.000000in}{-0.048611in}}%
\pgfusepath{stroke,fill}%
}%
\begin{pgfscope}%
\pgfsys@transformshift{0.591528in}{0.585278in}%
\pgfsys@useobject{currentmarker}{}%
\end{pgfscope}%
\end{pgfscope}%
\begin{pgfscope}%
\definecolor{textcolor}{rgb}{0.000000,0.000000,0.000000}%
\pgfsetstrokecolor{textcolor}%
\pgfsetfillcolor{textcolor}%
\pgftext[x=0.591528in,y=0.488056in,,top]{\color{textcolor}\rmfamily\fontsize{10.000000}{12.000000}\selectfont 0}%
\end{pgfscope}%
\begin{pgfscope}%
\pgfpathrectangle{\pgfqpoint{0.591528in}{0.585278in}}{\pgfqpoint{5.189097in}{3.264722in}}%
\pgfusepath{clip}%
\pgfsetrectcap%
\pgfsetroundjoin%
\pgfsetlinewidth{0.803000pt}%
\definecolor{currentstroke}{rgb}{0.690196,0.690196,0.690196}%
\pgfsetstrokecolor{currentstroke}%
\pgfsetdash{}{0pt}%
\pgfpathmoveto{\pgfqpoint{1.629347in}{0.585278in}}%
\pgfpathlineto{\pgfqpoint{1.629347in}{3.850000in}}%
\pgfusepath{stroke}%
\end{pgfscope}%
\begin{pgfscope}%
\pgfsetbuttcap%
\pgfsetroundjoin%
\definecolor{currentfill}{rgb}{0.000000,0.000000,0.000000}%
\pgfsetfillcolor{currentfill}%
\pgfsetlinewidth{0.803000pt}%
\definecolor{currentstroke}{rgb}{0.000000,0.000000,0.000000}%
\pgfsetstrokecolor{currentstroke}%
\pgfsetdash{}{0pt}%
\pgfsys@defobject{currentmarker}{\pgfqpoint{0.000000in}{-0.048611in}}{\pgfqpoint{0.000000in}{0.000000in}}{%
\pgfpathmoveto{\pgfqpoint{0.000000in}{0.000000in}}%
\pgfpathlineto{\pgfqpoint{0.000000in}{-0.048611in}}%
\pgfusepath{stroke,fill}%
}%
\begin{pgfscope}%
\pgfsys@transformshift{1.629347in}{0.585278in}%
\pgfsys@useobject{currentmarker}{}%
\end{pgfscope}%
\end{pgfscope}%
\begin{pgfscope}%
\definecolor{textcolor}{rgb}{0.000000,0.000000,0.000000}%
\pgfsetstrokecolor{textcolor}%
\pgfsetfillcolor{textcolor}%
\pgftext[x=1.629347in,y=0.488056in,,top]{\color{textcolor}\rmfamily\fontsize{10.000000}{12.000000}\selectfont 2}%
\end{pgfscope}%
\begin{pgfscope}%
\pgfpathrectangle{\pgfqpoint{0.591528in}{0.585278in}}{\pgfqpoint{5.189097in}{3.264722in}}%
\pgfusepath{clip}%
\pgfsetrectcap%
\pgfsetroundjoin%
\pgfsetlinewidth{0.803000pt}%
\definecolor{currentstroke}{rgb}{0.690196,0.690196,0.690196}%
\pgfsetstrokecolor{currentstroke}%
\pgfsetdash{}{0pt}%
\pgfpathmoveto{\pgfqpoint{2.667167in}{0.585278in}}%
\pgfpathlineto{\pgfqpoint{2.667167in}{3.850000in}}%
\pgfusepath{stroke}%
\end{pgfscope}%
\begin{pgfscope}%
\pgfsetbuttcap%
\pgfsetroundjoin%
\definecolor{currentfill}{rgb}{0.000000,0.000000,0.000000}%
\pgfsetfillcolor{currentfill}%
\pgfsetlinewidth{0.803000pt}%
\definecolor{currentstroke}{rgb}{0.000000,0.000000,0.000000}%
\pgfsetstrokecolor{currentstroke}%
\pgfsetdash{}{0pt}%
\pgfsys@defobject{currentmarker}{\pgfqpoint{0.000000in}{-0.048611in}}{\pgfqpoint{0.000000in}{0.000000in}}{%
\pgfpathmoveto{\pgfqpoint{0.000000in}{0.000000in}}%
\pgfpathlineto{\pgfqpoint{0.000000in}{-0.048611in}}%
\pgfusepath{stroke,fill}%
}%
\begin{pgfscope}%
\pgfsys@transformshift{2.667167in}{0.585278in}%
\pgfsys@useobject{currentmarker}{}%
\end{pgfscope}%
\end{pgfscope}%
\begin{pgfscope}%
\definecolor{textcolor}{rgb}{0.000000,0.000000,0.000000}%
\pgfsetstrokecolor{textcolor}%
\pgfsetfillcolor{textcolor}%
\pgftext[x=2.667167in,y=0.488056in,,top]{\color{textcolor}\rmfamily\fontsize{10.000000}{12.000000}\selectfont 4}%
\end{pgfscope}%
\begin{pgfscope}%
\pgfpathrectangle{\pgfqpoint{0.591528in}{0.585278in}}{\pgfqpoint{5.189097in}{3.264722in}}%
\pgfusepath{clip}%
\pgfsetrectcap%
\pgfsetroundjoin%
\pgfsetlinewidth{0.803000pt}%
\definecolor{currentstroke}{rgb}{0.690196,0.690196,0.690196}%
\pgfsetstrokecolor{currentstroke}%
\pgfsetdash{}{0pt}%
\pgfpathmoveto{\pgfqpoint{3.704986in}{0.585278in}}%
\pgfpathlineto{\pgfqpoint{3.704986in}{3.850000in}}%
\pgfusepath{stroke}%
\end{pgfscope}%
\begin{pgfscope}%
\pgfsetbuttcap%
\pgfsetroundjoin%
\definecolor{currentfill}{rgb}{0.000000,0.000000,0.000000}%
\pgfsetfillcolor{currentfill}%
\pgfsetlinewidth{0.803000pt}%
\definecolor{currentstroke}{rgb}{0.000000,0.000000,0.000000}%
\pgfsetstrokecolor{currentstroke}%
\pgfsetdash{}{0pt}%
\pgfsys@defobject{currentmarker}{\pgfqpoint{0.000000in}{-0.048611in}}{\pgfqpoint{0.000000in}{0.000000in}}{%
\pgfpathmoveto{\pgfqpoint{0.000000in}{0.000000in}}%
\pgfpathlineto{\pgfqpoint{0.000000in}{-0.048611in}}%
\pgfusepath{stroke,fill}%
}%
\begin{pgfscope}%
\pgfsys@transformshift{3.704986in}{0.585278in}%
\pgfsys@useobject{currentmarker}{}%
\end{pgfscope}%
\end{pgfscope}%
\begin{pgfscope}%
\definecolor{textcolor}{rgb}{0.000000,0.000000,0.000000}%
\pgfsetstrokecolor{textcolor}%
\pgfsetfillcolor{textcolor}%
\pgftext[x=3.704986in,y=0.488056in,,top]{\color{textcolor}\rmfamily\fontsize{10.000000}{12.000000}\selectfont 6}%
\end{pgfscope}%
\begin{pgfscope}%
\pgfpathrectangle{\pgfqpoint{0.591528in}{0.585278in}}{\pgfqpoint{5.189097in}{3.264722in}}%
\pgfusepath{clip}%
\pgfsetrectcap%
\pgfsetroundjoin%
\pgfsetlinewidth{0.803000pt}%
\definecolor{currentstroke}{rgb}{0.690196,0.690196,0.690196}%
\pgfsetstrokecolor{currentstroke}%
\pgfsetdash{}{0pt}%
\pgfpathmoveto{\pgfqpoint{4.742806in}{0.585278in}}%
\pgfpathlineto{\pgfqpoint{4.742806in}{3.850000in}}%
\pgfusepath{stroke}%
\end{pgfscope}%
\begin{pgfscope}%
\pgfsetbuttcap%
\pgfsetroundjoin%
\definecolor{currentfill}{rgb}{0.000000,0.000000,0.000000}%
\pgfsetfillcolor{currentfill}%
\pgfsetlinewidth{0.803000pt}%
\definecolor{currentstroke}{rgb}{0.000000,0.000000,0.000000}%
\pgfsetstrokecolor{currentstroke}%
\pgfsetdash{}{0pt}%
\pgfsys@defobject{currentmarker}{\pgfqpoint{0.000000in}{-0.048611in}}{\pgfqpoint{0.000000in}{0.000000in}}{%
\pgfpathmoveto{\pgfqpoint{0.000000in}{0.000000in}}%
\pgfpathlineto{\pgfqpoint{0.000000in}{-0.048611in}}%
\pgfusepath{stroke,fill}%
}%
\begin{pgfscope}%
\pgfsys@transformshift{4.742806in}{0.585278in}%
\pgfsys@useobject{currentmarker}{}%
\end{pgfscope}%
\end{pgfscope}%
\begin{pgfscope}%
\definecolor{textcolor}{rgb}{0.000000,0.000000,0.000000}%
\pgfsetstrokecolor{textcolor}%
\pgfsetfillcolor{textcolor}%
\pgftext[x=4.742806in,y=0.488056in,,top]{\color{textcolor}\rmfamily\fontsize{10.000000}{12.000000}\selectfont 8}%
\end{pgfscope}%
\begin{pgfscope}%
\pgfpathrectangle{\pgfqpoint{0.591528in}{0.585278in}}{\pgfqpoint{5.189097in}{3.264722in}}%
\pgfusepath{clip}%
\pgfsetrectcap%
\pgfsetroundjoin%
\pgfsetlinewidth{0.803000pt}%
\definecolor{currentstroke}{rgb}{0.690196,0.690196,0.690196}%
\pgfsetstrokecolor{currentstroke}%
\pgfsetdash{}{0pt}%
\pgfpathmoveto{\pgfqpoint{5.780625in}{0.585278in}}%
\pgfpathlineto{\pgfqpoint{5.780625in}{3.850000in}}%
\pgfusepath{stroke}%
\end{pgfscope}%
\begin{pgfscope}%
\pgfsetbuttcap%
\pgfsetroundjoin%
\definecolor{currentfill}{rgb}{0.000000,0.000000,0.000000}%
\pgfsetfillcolor{currentfill}%
\pgfsetlinewidth{0.803000pt}%
\definecolor{currentstroke}{rgb}{0.000000,0.000000,0.000000}%
\pgfsetstrokecolor{currentstroke}%
\pgfsetdash{}{0pt}%
\pgfsys@defobject{currentmarker}{\pgfqpoint{0.000000in}{-0.048611in}}{\pgfqpoint{0.000000in}{0.000000in}}{%
\pgfpathmoveto{\pgfqpoint{0.000000in}{0.000000in}}%
\pgfpathlineto{\pgfqpoint{0.000000in}{-0.048611in}}%
\pgfusepath{stroke,fill}%
}%
\begin{pgfscope}%
\pgfsys@transformshift{5.780625in}{0.585278in}%
\pgfsys@useobject{currentmarker}{}%
\end{pgfscope}%
\end{pgfscope}%
\begin{pgfscope}%
\definecolor{textcolor}{rgb}{0.000000,0.000000,0.000000}%
\pgfsetstrokecolor{textcolor}%
\pgfsetfillcolor{textcolor}%
\pgftext[x=5.780625in,y=0.488056in,,top]{\color{textcolor}\rmfamily\fontsize{10.000000}{12.000000}\selectfont 10}%
\end{pgfscope}%
\begin{pgfscope}%
\definecolor{textcolor}{rgb}{0.000000,0.000000,0.000000}%
\pgfsetstrokecolor{textcolor}%
\pgfsetfillcolor{textcolor}%
\pgftext[x=3.186076in,y=0.309043in,,top]{\color{textcolor}\rmfamily\fontsize{10.000000}{12.000000}\selectfont Time (s)}%
\end{pgfscope}%
\begin{pgfscope}%
\pgfpathrectangle{\pgfqpoint{0.591528in}{0.585278in}}{\pgfqpoint{5.189097in}{3.264722in}}%
\pgfusepath{clip}%
\pgfsetrectcap%
\pgfsetroundjoin%
\pgfsetlinewidth{0.803000pt}%
\definecolor{currentstroke}{rgb}{0.690196,0.690196,0.690196}%
\pgfsetstrokecolor{currentstroke}%
\pgfsetdash{}{0pt}%
\pgfpathmoveto{\pgfqpoint{0.591528in}{0.720931in}}%
\pgfpathlineto{\pgfqpoint{5.780625in}{0.720931in}}%
\pgfusepath{stroke}%
\end{pgfscope}%
\begin{pgfscope}%
\pgfsetbuttcap%
\pgfsetroundjoin%
\definecolor{currentfill}{rgb}{0.000000,0.000000,0.000000}%
\pgfsetfillcolor{currentfill}%
\pgfsetlinewidth{0.803000pt}%
\definecolor{currentstroke}{rgb}{0.000000,0.000000,0.000000}%
\pgfsetstrokecolor{currentstroke}%
\pgfsetdash{}{0pt}%
\pgfsys@defobject{currentmarker}{\pgfqpoint{-0.048611in}{0.000000in}}{\pgfqpoint{-0.000000in}{0.000000in}}{%
\pgfpathmoveto{\pgfqpoint{-0.000000in}{0.000000in}}%
\pgfpathlineto{\pgfqpoint{-0.048611in}{0.000000in}}%
\pgfusepath{stroke,fill}%
}%
\begin{pgfscope}%
\pgfsys@transformshift{0.591528in}{0.720931in}%
\pgfsys@useobject{currentmarker}{}%
\end{pgfscope}%
\end{pgfscope}%
\begin{pgfscope}%
\definecolor{textcolor}{rgb}{0.000000,0.000000,0.000000}%
\pgfsetstrokecolor{textcolor}%
\pgfsetfillcolor{textcolor}%
\pgftext[x=0.316836in, y=0.672706in, left, base]{\color{textcolor}\rmfamily\fontsize{10.000000}{12.000000}\selectfont \ensuremath{-}5}%
\end{pgfscope}%
\begin{pgfscope}%
\pgfpathrectangle{\pgfqpoint{0.591528in}{0.585278in}}{\pgfqpoint{5.189097in}{3.264722in}}%
\pgfusepath{clip}%
\pgfsetrectcap%
\pgfsetroundjoin%
\pgfsetlinewidth{0.803000pt}%
\definecolor{currentstroke}{rgb}{0.690196,0.690196,0.690196}%
\pgfsetstrokecolor{currentstroke}%
\pgfsetdash{}{0pt}%
\pgfpathmoveto{\pgfqpoint{0.591528in}{1.174003in}}%
\pgfpathlineto{\pgfqpoint{5.780625in}{1.174003in}}%
\pgfusepath{stroke}%
\end{pgfscope}%
\begin{pgfscope}%
\pgfsetbuttcap%
\pgfsetroundjoin%
\definecolor{currentfill}{rgb}{0.000000,0.000000,0.000000}%
\pgfsetfillcolor{currentfill}%
\pgfsetlinewidth{0.803000pt}%
\definecolor{currentstroke}{rgb}{0.000000,0.000000,0.000000}%
\pgfsetstrokecolor{currentstroke}%
\pgfsetdash{}{0pt}%
\pgfsys@defobject{currentmarker}{\pgfqpoint{-0.048611in}{0.000000in}}{\pgfqpoint{-0.000000in}{0.000000in}}{%
\pgfpathmoveto{\pgfqpoint{-0.000000in}{0.000000in}}%
\pgfpathlineto{\pgfqpoint{-0.048611in}{0.000000in}}%
\pgfusepath{stroke,fill}%
}%
\begin{pgfscope}%
\pgfsys@transformshift{0.591528in}{1.174003in}%
\pgfsys@useobject{currentmarker}{}%
\end{pgfscope}%
\end{pgfscope}%
\begin{pgfscope}%
\definecolor{textcolor}{rgb}{0.000000,0.000000,0.000000}%
\pgfsetstrokecolor{textcolor}%
\pgfsetfillcolor{textcolor}%
\pgftext[x=0.316836in, y=1.125778in, left, base]{\color{textcolor}\rmfamily\fontsize{10.000000}{12.000000}\selectfont \ensuremath{-}4}%
\end{pgfscope}%
\begin{pgfscope}%
\pgfpathrectangle{\pgfqpoint{0.591528in}{0.585278in}}{\pgfqpoint{5.189097in}{3.264722in}}%
\pgfusepath{clip}%
\pgfsetrectcap%
\pgfsetroundjoin%
\pgfsetlinewidth{0.803000pt}%
\definecolor{currentstroke}{rgb}{0.690196,0.690196,0.690196}%
\pgfsetstrokecolor{currentstroke}%
\pgfsetdash{}{0pt}%
\pgfpathmoveto{\pgfqpoint{0.591528in}{1.627075in}}%
\pgfpathlineto{\pgfqpoint{5.780625in}{1.627075in}}%
\pgfusepath{stroke}%
\end{pgfscope}%
\begin{pgfscope}%
\pgfsetbuttcap%
\pgfsetroundjoin%
\definecolor{currentfill}{rgb}{0.000000,0.000000,0.000000}%
\pgfsetfillcolor{currentfill}%
\pgfsetlinewidth{0.803000pt}%
\definecolor{currentstroke}{rgb}{0.000000,0.000000,0.000000}%
\pgfsetstrokecolor{currentstroke}%
\pgfsetdash{}{0pt}%
\pgfsys@defobject{currentmarker}{\pgfqpoint{-0.048611in}{0.000000in}}{\pgfqpoint{-0.000000in}{0.000000in}}{%
\pgfpathmoveto{\pgfqpoint{-0.000000in}{0.000000in}}%
\pgfpathlineto{\pgfqpoint{-0.048611in}{0.000000in}}%
\pgfusepath{stroke,fill}%
}%
\begin{pgfscope}%
\pgfsys@transformshift{0.591528in}{1.627075in}%
\pgfsys@useobject{currentmarker}{}%
\end{pgfscope}%
\end{pgfscope}%
\begin{pgfscope}%
\definecolor{textcolor}{rgb}{0.000000,0.000000,0.000000}%
\pgfsetstrokecolor{textcolor}%
\pgfsetfillcolor{textcolor}%
\pgftext[x=0.316836in, y=1.578849in, left, base]{\color{textcolor}\rmfamily\fontsize{10.000000}{12.000000}\selectfont \ensuremath{-}3}%
\end{pgfscope}%
\begin{pgfscope}%
\pgfpathrectangle{\pgfqpoint{0.591528in}{0.585278in}}{\pgfqpoint{5.189097in}{3.264722in}}%
\pgfusepath{clip}%
\pgfsetrectcap%
\pgfsetroundjoin%
\pgfsetlinewidth{0.803000pt}%
\definecolor{currentstroke}{rgb}{0.690196,0.690196,0.690196}%
\pgfsetstrokecolor{currentstroke}%
\pgfsetdash{}{0pt}%
\pgfpathmoveto{\pgfqpoint{0.591528in}{2.080146in}}%
\pgfpathlineto{\pgfqpoint{5.780625in}{2.080146in}}%
\pgfusepath{stroke}%
\end{pgfscope}%
\begin{pgfscope}%
\pgfsetbuttcap%
\pgfsetroundjoin%
\definecolor{currentfill}{rgb}{0.000000,0.000000,0.000000}%
\pgfsetfillcolor{currentfill}%
\pgfsetlinewidth{0.803000pt}%
\definecolor{currentstroke}{rgb}{0.000000,0.000000,0.000000}%
\pgfsetstrokecolor{currentstroke}%
\pgfsetdash{}{0pt}%
\pgfsys@defobject{currentmarker}{\pgfqpoint{-0.048611in}{0.000000in}}{\pgfqpoint{-0.000000in}{0.000000in}}{%
\pgfpathmoveto{\pgfqpoint{-0.000000in}{0.000000in}}%
\pgfpathlineto{\pgfqpoint{-0.048611in}{0.000000in}}%
\pgfusepath{stroke,fill}%
}%
\begin{pgfscope}%
\pgfsys@transformshift{0.591528in}{2.080146in}%
\pgfsys@useobject{currentmarker}{}%
\end{pgfscope}%
\end{pgfscope}%
\begin{pgfscope}%
\definecolor{textcolor}{rgb}{0.000000,0.000000,0.000000}%
\pgfsetstrokecolor{textcolor}%
\pgfsetfillcolor{textcolor}%
\pgftext[x=0.316836in, y=2.031921in, left, base]{\color{textcolor}\rmfamily\fontsize{10.000000}{12.000000}\selectfont \ensuremath{-}2}%
\end{pgfscope}%
\begin{pgfscope}%
\pgfpathrectangle{\pgfqpoint{0.591528in}{0.585278in}}{\pgfqpoint{5.189097in}{3.264722in}}%
\pgfusepath{clip}%
\pgfsetrectcap%
\pgfsetroundjoin%
\pgfsetlinewidth{0.803000pt}%
\definecolor{currentstroke}{rgb}{0.690196,0.690196,0.690196}%
\pgfsetstrokecolor{currentstroke}%
\pgfsetdash{}{0pt}%
\pgfpathmoveto{\pgfqpoint{0.591528in}{2.533218in}}%
\pgfpathlineto{\pgfqpoint{5.780625in}{2.533218in}}%
\pgfusepath{stroke}%
\end{pgfscope}%
\begin{pgfscope}%
\pgfsetbuttcap%
\pgfsetroundjoin%
\definecolor{currentfill}{rgb}{0.000000,0.000000,0.000000}%
\pgfsetfillcolor{currentfill}%
\pgfsetlinewidth{0.803000pt}%
\definecolor{currentstroke}{rgb}{0.000000,0.000000,0.000000}%
\pgfsetstrokecolor{currentstroke}%
\pgfsetdash{}{0pt}%
\pgfsys@defobject{currentmarker}{\pgfqpoint{-0.048611in}{0.000000in}}{\pgfqpoint{-0.000000in}{0.000000in}}{%
\pgfpathmoveto{\pgfqpoint{-0.000000in}{0.000000in}}%
\pgfpathlineto{\pgfqpoint{-0.048611in}{0.000000in}}%
\pgfusepath{stroke,fill}%
}%
\begin{pgfscope}%
\pgfsys@transformshift{0.591528in}{2.533218in}%
\pgfsys@useobject{currentmarker}{}%
\end{pgfscope}%
\end{pgfscope}%
\begin{pgfscope}%
\definecolor{textcolor}{rgb}{0.000000,0.000000,0.000000}%
\pgfsetstrokecolor{textcolor}%
\pgfsetfillcolor{textcolor}%
\pgftext[x=0.316836in, y=2.484993in, left, base]{\color{textcolor}\rmfamily\fontsize{10.000000}{12.000000}\selectfont \ensuremath{-}1}%
\end{pgfscope}%
\begin{pgfscope}%
\pgfpathrectangle{\pgfqpoint{0.591528in}{0.585278in}}{\pgfqpoint{5.189097in}{3.264722in}}%
\pgfusepath{clip}%
\pgfsetrectcap%
\pgfsetroundjoin%
\pgfsetlinewidth{0.803000pt}%
\definecolor{currentstroke}{rgb}{0.690196,0.690196,0.690196}%
\pgfsetstrokecolor{currentstroke}%
\pgfsetdash{}{0pt}%
\pgfpathmoveto{\pgfqpoint{0.591528in}{2.986290in}}%
\pgfpathlineto{\pgfqpoint{5.780625in}{2.986290in}}%
\pgfusepath{stroke}%
\end{pgfscope}%
\begin{pgfscope}%
\pgfsetbuttcap%
\pgfsetroundjoin%
\definecolor{currentfill}{rgb}{0.000000,0.000000,0.000000}%
\pgfsetfillcolor{currentfill}%
\pgfsetlinewidth{0.803000pt}%
\definecolor{currentstroke}{rgb}{0.000000,0.000000,0.000000}%
\pgfsetstrokecolor{currentstroke}%
\pgfsetdash{}{0pt}%
\pgfsys@defobject{currentmarker}{\pgfqpoint{-0.048611in}{0.000000in}}{\pgfqpoint{-0.000000in}{0.000000in}}{%
\pgfpathmoveto{\pgfqpoint{-0.000000in}{0.000000in}}%
\pgfpathlineto{\pgfqpoint{-0.048611in}{0.000000in}}%
\pgfusepath{stroke,fill}%
}%
\begin{pgfscope}%
\pgfsys@transformshift{0.591528in}{2.986290in}%
\pgfsys@useobject{currentmarker}{}%
\end{pgfscope}%
\end{pgfscope}%
\begin{pgfscope}%
\definecolor{textcolor}{rgb}{0.000000,0.000000,0.000000}%
\pgfsetstrokecolor{textcolor}%
\pgfsetfillcolor{textcolor}%
\pgftext[x=0.424861in, y=2.938064in, left, base]{\color{textcolor}\rmfamily\fontsize{10.000000}{12.000000}\selectfont 0}%
\end{pgfscope}%
\begin{pgfscope}%
\pgfpathrectangle{\pgfqpoint{0.591528in}{0.585278in}}{\pgfqpoint{5.189097in}{3.264722in}}%
\pgfusepath{clip}%
\pgfsetrectcap%
\pgfsetroundjoin%
\pgfsetlinewidth{0.803000pt}%
\definecolor{currentstroke}{rgb}{0.690196,0.690196,0.690196}%
\pgfsetstrokecolor{currentstroke}%
\pgfsetdash{}{0pt}%
\pgfpathmoveto{\pgfqpoint{0.591528in}{3.439361in}}%
\pgfpathlineto{\pgfqpoint{5.780625in}{3.439361in}}%
\pgfusepath{stroke}%
\end{pgfscope}%
\begin{pgfscope}%
\pgfsetbuttcap%
\pgfsetroundjoin%
\definecolor{currentfill}{rgb}{0.000000,0.000000,0.000000}%
\pgfsetfillcolor{currentfill}%
\pgfsetlinewidth{0.803000pt}%
\definecolor{currentstroke}{rgb}{0.000000,0.000000,0.000000}%
\pgfsetstrokecolor{currentstroke}%
\pgfsetdash{}{0pt}%
\pgfsys@defobject{currentmarker}{\pgfqpoint{-0.048611in}{0.000000in}}{\pgfqpoint{-0.000000in}{0.000000in}}{%
\pgfpathmoveto{\pgfqpoint{-0.000000in}{0.000000in}}%
\pgfpathlineto{\pgfqpoint{-0.048611in}{0.000000in}}%
\pgfusepath{stroke,fill}%
}%
\begin{pgfscope}%
\pgfsys@transformshift{0.591528in}{3.439361in}%
\pgfsys@useobject{currentmarker}{}%
\end{pgfscope}%
\end{pgfscope}%
\begin{pgfscope}%
\definecolor{textcolor}{rgb}{0.000000,0.000000,0.000000}%
\pgfsetstrokecolor{textcolor}%
\pgfsetfillcolor{textcolor}%
\pgftext[x=0.424861in, y=3.391136in, left, base]{\color{textcolor}\rmfamily\fontsize{10.000000}{12.000000}\selectfont 1}%
\end{pgfscope}%
\begin{pgfscope}%
\definecolor{textcolor}{rgb}{0.000000,0.000000,0.000000}%
\pgfsetstrokecolor{textcolor}%
\pgfsetfillcolor{textcolor}%
\pgftext[x=0.261280in,y=2.217639in,,bottom,rotate=90.000000]{\color{textcolor}\rmfamily\fontsize{10.000000}{12.000000}\selectfont State Values}%
\end{pgfscope}%
\begin{pgfscope}%
\pgfpathrectangle{\pgfqpoint{0.591528in}{0.585278in}}{\pgfqpoint{5.189097in}{3.264722in}}%
\pgfusepath{clip}%
\pgfsetrectcap%
\pgfsetroundjoin%
\pgfsetlinewidth{0.803000pt}%
\definecolor{currentstroke}{rgb}{0.121569,0.466667,0.705882}%
\pgfsetstrokecolor{currentstroke}%
\pgfsetdash{}{0pt}%
\pgfpathmoveto{\pgfqpoint{0.591528in}{3.439361in}}%
\pgfpathlineto{\pgfqpoint{0.627333in}{3.440473in}}%
\pgfpathlineto{\pgfqpoint{0.651202in}{3.443050in}}%
\pgfpathlineto{\pgfqpoint{0.661581in}{3.446979in}}%
\pgfpathlineto{\pgfqpoint{0.690640in}{3.460408in}}%
\pgfpathlineto{\pgfqpoint{0.701018in}{3.461049in}}%
\pgfpathlineto{\pgfqpoint{0.713990in}{3.459692in}}%
\pgfpathlineto{\pgfqpoint{0.724888in}{3.456525in}}%
\pgfpathlineto{\pgfqpoint{0.731633in}{3.451811in}}%
\pgfpathlineto{\pgfqpoint{0.738898in}{3.443744in}}%
\pgfpathlineto{\pgfqpoint{0.746682in}{3.431662in}}%
\pgfpathlineto{\pgfqpoint{0.757579in}{3.409824in}}%
\pgfpathlineto{\pgfqpoint{0.769514in}{3.380569in}}%
\pgfpathlineto{\pgfqpoint{0.784043in}{3.338219in}}%
\pgfpathlineto{\pgfqpoint{0.800648in}{3.283065in}}%
\pgfpathlineto{\pgfqpoint{0.825037in}{3.193112in}}%
\pgfpathlineto{\pgfqpoint{0.866031in}{3.030447in}}%
\pgfpathlineto{\pgfqpoint{0.976040in}{2.580725in}}%
\pgfpathlineto{\pgfqpoint{1.044017in}{2.287796in}}%
\pgfpathlineto{\pgfqpoint{1.101097in}{2.045960in}}%
\pgfpathlineto{\pgfqpoint{1.138459in}{1.897334in}}%
\pgfpathlineto{\pgfqpoint{1.171150in}{1.776183in}}%
\pgfpathlineto{\pgfqpoint{1.201766in}{1.671063in}}%
\pgfpathlineto{\pgfqpoint{1.230825in}{1.579121in}}%
\pgfpathlineto{\pgfqpoint{1.257808in}{1.500927in}}%
\pgfpathlineto{\pgfqpoint{1.283753in}{1.432390in}}%
\pgfpathlineto{\pgfqpoint{1.308661in}{1.372651in}}%
\pgfpathlineto{\pgfqpoint{1.332531in}{1.321002in}}%
\pgfpathlineto{\pgfqpoint{1.355882in}{1.275607in}}%
\pgfpathlineto{\pgfqpoint{1.378714in}{1.236025in}}%
\pgfpathlineto{\pgfqpoint{1.400508in}{1.202539in}}%
\pgfpathlineto{\pgfqpoint{1.421264in}{1.174548in}}%
\pgfpathlineto{\pgfqpoint{1.442540in}{1.149640in}}%
\pgfpathlineto{\pgfqpoint{1.463296in}{1.128779in}}%
\pgfpathlineto{\pgfqpoint{1.483534in}{1.111553in}}%
\pgfpathlineto{\pgfqpoint{1.502733in}{1.098046in}}%
\pgfpathlineto{\pgfqpoint{1.522452in}{1.086848in}}%
\pgfpathlineto{\pgfqpoint{1.541651in}{1.078440in}}%
\pgfpathlineto{\pgfqpoint{1.560851in}{1.072443in}}%
\pgfpathlineto{\pgfqpoint{1.580570in}{1.068566in}}%
\pgfpathlineto{\pgfqpoint{1.600288in}{1.066926in}}%
\pgfpathlineto{\pgfqpoint{1.620007in}{1.067494in}}%
\pgfpathlineto{\pgfqpoint{1.640244in}{1.070338in}}%
\pgfpathlineto{\pgfqpoint{1.661001in}{1.075515in}}%
\pgfpathlineto{\pgfqpoint{1.683314in}{1.083431in}}%
\pgfpathlineto{\pgfqpoint{1.706665in}{1.094109in}}%
\pgfpathlineto{\pgfqpoint{1.731572in}{1.107985in}}%
\pgfpathlineto{\pgfqpoint{1.759075in}{1.125979in}}%
\pgfpathlineto{\pgfqpoint{1.787615in}{1.147370in}}%
\pgfpathlineto{\pgfqpoint{1.817711in}{1.172793in}}%
\pgfpathlineto{\pgfqpoint{1.848846in}{1.202053in}}%
\pgfpathlineto{\pgfqpoint{1.885170in}{1.239355in}}%
\pgfpathlineto{\pgfqpoint{1.924088in}{1.282406in}}%
\pgfpathlineto{\pgfqpoint{1.968714in}{1.335219in}}%
\pgfpathlineto{\pgfqpoint{2.016973in}{1.395796in}}%
\pgfpathlineto{\pgfqpoint{2.079242in}{1.477917in}}%
\pgfpathlineto{\pgfqpoint{2.163305in}{1.592387in}}%
\pgfpathlineto{\pgfqpoint{2.358934in}{1.859887in}}%
\pgfpathlineto{\pgfqpoint{2.433657in}{1.958518in}}%
\pgfpathlineto{\pgfqpoint{2.508899in}{2.054090in}}%
\pgfpathlineto{\pgfqpoint{2.562866in}{2.119427in}}%
\pgfpathlineto{\pgfqpoint{2.624097in}{2.190372in}}%
\pgfpathlineto{\pgfqpoint{2.680139in}{2.252030in}}%
\pgfpathlineto{\pgfqpoint{2.745003in}{2.319822in}}%
\pgfpathlineto{\pgfqpoint{2.798451in}{2.372894in}}%
\pgfpathlineto{\pgfqpoint{2.856569in}{2.427332in}}%
\pgfpathlineto{\pgfqpoint{2.911573in}{2.476034in}}%
\pgfpathlineto{\pgfqpoint{2.968134in}{2.523123in}}%
\pgfpathlineto{\pgfqpoint{3.021063in}{2.564428in}}%
\pgfpathlineto{\pgfqpoint{3.073992in}{2.602853in}}%
\pgfpathlineto{\pgfqpoint{3.131591in}{2.641713in}}%
\pgfpathlineto{\pgfqpoint{3.188152in}{2.677328in}}%
\pgfpathlineto{\pgfqpoint{3.240043in}{2.707415in}}%
\pgfpathlineto{\pgfqpoint{3.298161in}{2.738596in}}%
\pgfpathlineto{\pgfqpoint{3.364581in}{2.771538in}}%
\pgfpathlineto{\pgfqpoint{3.421661in}{2.797789in}}%
\pgfpathlineto{\pgfqpoint{3.474590in}{2.819765in}}%
\pgfpathlineto{\pgfqpoint{3.542048in}{2.844893in}}%
\pgfpathlineto{\pgfqpoint{3.603799in}{2.865960in}}%
\pgfpathlineto{\pgfqpoint{3.661917in}{2.883421in}}%
\pgfpathlineto{\pgfqpoint{3.741310in}{2.904682in}}%
\pgfpathlineto{\pgfqpoint{3.807211in}{2.919882in}}%
\pgfpathlineto{\pgfqpoint{3.893350in}{2.936869in}}%
\pgfpathlineto{\pgfqpoint{3.965479in}{2.948474in}}%
\pgfpathlineto{\pgfqpoint{4.043834in}{2.958827in}}%
\pgfpathlineto{\pgfqpoint{4.165778in}{2.972463in}}%
\pgfpathlineto{\pgfqpoint{4.269560in}{2.981466in}}%
\pgfpathlineto{\pgfqpoint{4.356737in}{2.987289in}}%
\pgfpathlineto{\pgfqpoint{4.454292in}{2.992053in}}%
\pgfpathlineto{\pgfqpoint{4.615673in}{2.997539in}}%
\pgfpathlineto{\pgfqpoint{4.768751in}{3.000253in}}%
\pgfpathlineto{\pgfqpoint{5.038584in}{3.000511in}}%
\pgfpathlineto{\pgfqpoint{5.267423in}{2.997386in}}%
\pgfpathlineto{\pgfqpoint{5.360308in}{2.995927in}}%
\pgfpathlineto{\pgfqpoint{5.780625in}{2.987255in}}%
\pgfpathlineto{\pgfqpoint{5.780625in}{2.987255in}}%
\pgfusepath{stroke}%
\end{pgfscope}%
\begin{pgfscope}%
\pgfpathrectangle{\pgfqpoint{0.591528in}{0.585278in}}{\pgfqpoint{5.189097in}{3.264722in}}%
\pgfusepath{clip}%
\pgfsetrectcap%
\pgfsetroundjoin%
\pgfsetlinewidth{0.803000pt}%
\definecolor{currentstroke}{rgb}{1.000000,0.498039,0.054902}%
\pgfsetstrokecolor{currentstroke}%
\pgfsetdash{}{0pt}%
\pgfpathmoveto{\pgfqpoint{0.591528in}{3.212825in}}%
\pgfpathlineto{\pgfqpoint{0.606576in}{3.213957in}}%
\pgfpathlineto{\pgfqpoint{0.621625in}{3.217363in}}%
\pgfpathlineto{\pgfqpoint{0.636673in}{3.223059in}}%
\pgfpathlineto{\pgfqpoint{0.649127in}{3.229875in}}%
\pgfpathlineto{\pgfqpoint{0.658467in}{3.237810in}}%
\pgfpathlineto{\pgfqpoint{0.668326in}{3.249173in}}%
\pgfpathlineto{\pgfqpoint{0.687526in}{3.272677in}}%
\pgfpathlineto{\pgfqpoint{0.706726in}{3.292034in}}%
\pgfpathlineto{\pgfqpoint{0.728001in}{3.311183in}}%
\pgfpathlineto{\pgfqpoint{0.735266in}{3.314482in}}%
\pgfpathlineto{\pgfqpoint{0.742531in}{3.315311in}}%
\pgfpathlineto{\pgfqpoint{0.749795in}{3.313738in}}%
\pgfpathlineto{\pgfqpoint{0.759136in}{3.309332in}}%
\pgfpathlineto{\pgfqpoint{0.768995in}{3.302087in}}%
\pgfpathlineto{\pgfqpoint{0.780411in}{3.290688in}}%
\pgfpathlineto{\pgfqpoint{0.795459in}{3.272576in}}%
\pgfpathlineto{\pgfqpoint{0.818291in}{3.240569in}}%
\pgfpathlineto{\pgfqpoint{0.874334in}{3.160958in}}%
\pgfpathlineto{\pgfqpoint{0.898722in}{3.130569in}}%
\pgfpathlineto{\pgfqpoint{0.923630in}{3.103153in}}%
\pgfpathlineto{\pgfqpoint{1.031044in}{2.990136in}}%
\pgfpathlineto{\pgfqpoint{1.079822in}{2.938423in}}%
\pgfpathlineto{\pgfqpoint{1.111475in}{2.907963in}}%
\pgfpathlineto{\pgfqpoint{1.137421in}{2.885845in}}%
\pgfpathlineto{\pgfqpoint{1.163366in}{2.866528in}}%
\pgfpathlineto{\pgfqpoint{1.188793in}{2.850272in}}%
\pgfpathlineto{\pgfqpoint{1.214738in}{2.836310in}}%
\pgfpathlineto{\pgfqpoint{1.240165in}{2.825104in}}%
\pgfpathlineto{\pgfqpoint{1.265073in}{2.816485in}}%
\pgfpathlineto{\pgfqpoint{1.292056in}{2.809562in}}%
\pgfpathlineto{\pgfqpoint{1.318520in}{2.805017in}}%
\pgfpathlineto{\pgfqpoint{1.347579in}{2.802353in}}%
\pgfpathlineto{\pgfqpoint{1.378195in}{2.801807in}}%
\pgfpathlineto{\pgfqpoint{1.410886in}{2.803451in}}%
\pgfpathlineto{\pgfqpoint{1.445653in}{2.807540in}}%
\pgfpathlineto{\pgfqpoint{1.490798in}{2.815197in}}%
\pgfpathlineto{\pgfqpoint{1.543208in}{2.826610in}}%
\pgfpathlineto{\pgfqpoint{1.621564in}{2.846297in}}%
\pgfpathlineto{\pgfqpoint{1.709778in}{2.870885in}}%
\pgfpathlineto{\pgfqpoint{1.805258in}{2.896641in}}%
\pgfpathlineto{\pgfqpoint{1.917861in}{2.924623in}}%
\pgfpathlineto{\pgfqpoint{1.997254in}{2.941829in}}%
\pgfpathlineto{\pgfqpoint{2.083393in}{2.958456in}}%
\pgfpathlineto{\pgfqpoint{2.166419in}{2.971672in}}%
\pgfpathlineto{\pgfqpoint{2.237509in}{2.980815in}}%
\pgfpathlineto{\pgfqpoint{2.351151in}{2.992393in}}%
\pgfpathlineto{\pgfqpoint{2.435214in}{2.999014in}}%
\pgfpathlineto{\pgfqpoint{2.544185in}{3.005193in}}%
\pgfpathlineto{\pgfqpoint{2.672875in}{3.008768in}}%
\pgfpathlineto{\pgfqpoint{2.933886in}{3.010771in}}%
\pgfpathlineto{\pgfqpoint{3.090597in}{3.009685in}}%
\pgfpathlineto{\pgfqpoint{3.866367in}{2.997865in}}%
\pgfpathlineto{\pgfqpoint{3.992981in}{2.995507in}}%
\pgfpathlineto{\pgfqpoint{4.130492in}{2.993356in}}%
\pgfpathlineto{\pgfqpoint{4.943624in}{2.988307in}}%
\pgfpathlineto{\pgfqpoint{5.057265in}{2.987708in}}%
\pgfpathlineto{\pgfqpoint{5.199965in}{2.986892in}}%
\pgfpathlineto{\pgfqpoint{5.506122in}{2.986313in}}%
\pgfpathlineto{\pgfqpoint{5.721988in}{2.985231in}}%
\pgfpathlineto{\pgfqpoint{5.780625in}{2.985206in}}%
\pgfpathlineto{\pgfqpoint{5.780625in}{2.985206in}}%
\pgfusepath{stroke}%
\end{pgfscope}%
\begin{pgfscope}%
\pgfpathrectangle{\pgfqpoint{0.591528in}{0.585278in}}{\pgfqpoint{5.189097in}{3.264722in}}%
\pgfusepath{clip}%
\pgfsetrectcap%
\pgfsetroundjoin%
\pgfsetlinewidth{0.803000pt}%
\definecolor{currentstroke}{rgb}{0.172549,0.627451,0.172549}%
\pgfsetstrokecolor{currentstroke}%
\pgfsetdash{}{0pt}%
\pgfpathmoveto{\pgfqpoint{0.591528in}{2.986290in}}%
\pgfpathlineto{\pgfqpoint{0.643419in}{3.032431in}}%
\pgfpathlineto{\pgfqpoint{0.669364in}{3.333639in}}%
\pgfpathlineto{\pgfqpoint{0.669883in}{3.327266in}}%
\pgfpathlineto{\pgfqpoint{0.697385in}{3.002047in}}%
\pgfpathlineto{\pgfqpoint{0.721255in}{2.837334in}}%
\pgfpathlineto{\pgfqpoint{0.750314in}{1.983142in}}%
\pgfpathlineto{\pgfqpoint{0.776260in}{1.477550in}}%
\pgfpathlineto{\pgfqpoint{0.801167in}{1.156808in}}%
\pgfpathlineto{\pgfqpoint{0.827113in}{0.977626in}}%
\pgfpathlineto{\pgfqpoint{0.853577in}{0.902671in}}%
\pgfpathlineto{\pgfqpoint{0.878485in}{0.872991in}}%
\pgfpathlineto{\pgfqpoint{0.901317in}{0.882444in}}%
\pgfpathlineto{\pgfqpoint{0.912214in}{0.885248in}}%
\pgfpathlineto{\pgfqpoint{0.928819in}{0.888359in}}%
\pgfpathlineto{\pgfqpoint{0.955802in}{0.839998in}}%
\pgfpathlineto{\pgfqpoint{0.983305in}{0.772307in}}%
\pgfpathlineto{\pgfqpoint{1.010288in}{0.743566in}}%
\pgfpathlineto{\pgfqpoint{1.034158in}{0.735031in}}%
\pgfpathlineto{\pgfqpoint{1.060622in}{0.759235in}}%
\pgfpathlineto{\pgfqpoint{1.088124in}{0.817841in}}%
\pgfpathlineto{\pgfqpoint{1.115108in}{0.902673in}}%
\pgfpathlineto{\pgfqpoint{1.157139in}{1.072547in}}%
\pgfpathlineto{\pgfqpoint{1.173226in}{1.143993in}}%
\pgfpathlineto{\pgfqpoint{1.222003in}{1.371891in}}%
\pgfpathlineto{\pgfqpoint{1.314888in}{1.835683in}}%
\pgfpathlineto{\pgfqpoint{1.344466in}{1.978972in}}%
\pgfpathlineto{\pgfqpoint{1.399470in}{2.234002in}}%
\pgfpathlineto{\pgfqpoint{1.433718in}{2.386359in}}%
\pgfpathlineto{\pgfqpoint{1.464334in}{2.509286in}}%
\pgfpathlineto{\pgfqpoint{1.509998in}{2.682600in}}%
\pgfpathlineto{\pgfqpoint{1.558775in}{2.848057in}}%
\pgfpathlineto{\pgfqpoint{1.620007in}{3.030082in}}%
\pgfpathlineto{\pgfqpoint{1.648028in}{3.108793in}}%
\pgfpathlineto{\pgfqpoint{1.666190in}{3.155648in}}%
\pgfpathlineto{\pgfqpoint{1.694211in}{3.221814in}}%
\pgfpathlineto{\pgfqpoint{1.721713in}{3.280664in}}%
\pgfpathlineto{\pgfqpoint{1.741951in}{3.319798in}}%
\pgfpathlineto{\pgfqpoint{1.796955in}{3.415220in}}%
\pgfpathlineto{\pgfqpoint{1.842619in}{3.488241in}}%
\pgfpathlineto{\pgfqpoint{1.868565in}{3.521690in}}%
\pgfpathlineto{\pgfqpoint{1.900218in}{3.555452in}}%
\pgfpathlineto{\pgfqpoint{1.922531in}{3.579320in}}%
\pgfpathlineto{\pgfqpoint{1.952628in}{3.605750in}}%
\pgfpathlineto{\pgfqpoint{2.016454in}{3.655124in}}%
\pgfpathlineto{\pgfqpoint{2.026832in}{3.661261in}}%
\pgfpathlineto{\pgfqpoint{2.049145in}{3.672605in}}%
\pgfpathlineto{\pgfqpoint{2.151370in}{3.700547in}}%
\pgfpathlineto{\pgfqpoint{2.207932in}{3.700700in}}%
\pgfpathlineto{\pgfqpoint{2.262417in}{3.695580in}}%
\pgfpathlineto{\pgfqpoint{2.303930in}{3.695750in}}%
\pgfpathlineto{\pgfqpoint{2.344924in}{3.686972in}}%
\pgfpathlineto{\pgfqpoint{2.364642in}{3.682452in}}%
\pgfpathlineto{\pgfqpoint{2.391107in}{3.673405in}}%
\pgfpathlineto{\pgfqpoint{2.419647in}{3.662519in}}%
\pgfpathlineto{\pgfqpoint{2.450262in}{3.653718in}}%
\pgfpathlineto{\pgfqpoint{2.464792in}{3.648898in}}%
\pgfpathlineto{\pgfqpoint{2.493851in}{3.636420in}}%
\pgfpathlineto{\pgfqpoint{2.514088in}{3.628636in}}%
\pgfpathlineto{\pgfqpoint{2.540553in}{3.609864in}}%
\pgfpathlineto{\pgfqpoint{2.574282in}{3.596798in}}%
\pgfpathlineto{\pgfqpoint{2.631362in}{3.569179in}}%
\pgfpathlineto{\pgfqpoint{2.647448in}{3.560458in}}%
\pgfpathlineto{\pgfqpoint{2.672875in}{3.544917in}}%
\pgfpathlineto{\pgfqpoint{2.720614in}{3.525113in}}%
\pgfpathlineto{\pgfqpoint{2.749673in}{3.514266in}}%
\pgfpathlineto{\pgfqpoint{2.820245in}{3.474837in}}%
\pgfpathlineto{\pgfqpoint{2.831661in}{3.470094in}}%
\pgfpathlineto{\pgfqpoint{2.861239in}{3.457575in}}%
\pgfpathlineto{\pgfqpoint{2.888741in}{3.443249in}}%
\pgfpathlineto{\pgfqpoint{2.911573in}{3.431642in}}%
\pgfpathlineto{\pgfqpoint{2.948416in}{3.413926in}}%
\pgfpathlineto{\pgfqpoint{2.967615in}{3.405481in}}%
\pgfpathlineto{\pgfqpoint{2.985258in}{3.396826in}}%
\pgfpathlineto{\pgfqpoint{3.020025in}{3.377193in}}%
\pgfpathlineto{\pgfqpoint{3.096305in}{3.338745in}}%
\pgfpathlineto{\pgfqpoint{3.137299in}{3.322905in}}%
\pgfpathlineto{\pgfqpoint{3.191265in}{3.298503in}}%
\pgfpathlineto{\pgfqpoint{3.220843in}{3.283663in}}%
\pgfpathlineto{\pgfqpoint{3.304907in}{3.251217in}}%
\pgfpathlineto{\pgfqpoint{3.342268in}{3.240041in}}%
\pgfpathlineto{\pgfqpoint{3.371846in}{3.233691in}}%
\pgfpathlineto{\pgfqpoint{3.463693in}{3.193963in}}%
\pgfpathlineto{\pgfqpoint{3.476147in}{3.188342in}}%
\pgfpathlineto{\pgfqpoint{3.528038in}{3.174356in}}%
\pgfpathlineto{\pgfqpoint{3.552427in}{3.171331in}}%
\pgfpathlineto{\pgfqpoint{3.589788in}{3.157202in}}%
\pgfpathlineto{\pgfqpoint{3.638566in}{3.140012in}}%
\pgfpathlineto{\pgfqpoint{3.657246in}{3.133484in}}%
\pgfpathlineto{\pgfqpoint{3.694608in}{3.127073in}}%
\pgfpathlineto{\pgfqpoint{3.735602in}{3.117265in}}%
\pgfpathlineto{\pgfqpoint{3.761028in}{3.108504in}}%
\pgfpathlineto{\pgfqpoint{3.786455in}{3.104190in}}%
\pgfpathlineto{\pgfqpoint{3.812919in}{3.094212in}}%
\pgfpathlineto{\pgfqpoint{3.843535in}{3.090490in}}%
\pgfpathlineto{\pgfqpoint{3.863254in}{3.088891in}}%
\pgfpathlineto{\pgfqpoint{3.896983in}{3.076895in}}%
\pgfpathlineto{\pgfqpoint{3.922928in}{3.070841in}}%
\pgfpathlineto{\pgfqpoint{3.962365in}{3.063054in}}%
\pgfpathlineto{\pgfqpoint{3.983641in}{3.057317in}}%
\pgfpathlineto{\pgfqpoint{3.995576in}{3.055146in}}%
\pgfpathlineto{\pgfqpoint{4.024634in}{3.052360in}}%
\pgfpathlineto{\pgfqpoint{4.048504in}{3.048968in}}%
\pgfpathlineto{\pgfqpoint{4.074450in}{3.047505in}}%
\pgfpathlineto{\pgfqpoint{4.098839in}{3.044167in}}%
\pgfpathlineto{\pgfqpoint{4.123746in}{3.044554in}}%
\pgfpathlineto{\pgfqpoint{4.157994in}{3.037921in}}%
\pgfpathlineto{\pgfqpoint{4.182383in}{3.035777in}}%
\pgfpathlineto{\pgfqpoint{4.202621in}{3.034165in}}%
\pgfpathlineto{\pgfqpoint{4.229604in}{3.028401in}}%
\pgfpathlineto{\pgfqpoint{4.262814in}{3.027395in}}%
\pgfpathlineto{\pgfqpoint{4.278900in}{3.026761in}}%
\pgfpathlineto{\pgfqpoint{4.305884in}{3.019858in}}%
\pgfpathlineto{\pgfqpoint{4.385277in}{3.015587in}}%
\pgfpathlineto{\pgfqpoint{4.422638in}{3.007890in}}%
\pgfpathlineto{\pgfqpoint{4.436130in}{3.006604in}}%
\pgfpathlineto{\pgfqpoint{4.474529in}{3.007359in}}%
\pgfpathlineto{\pgfqpoint{4.521750in}{3.005535in}}%
\pgfpathlineto{\pgfqpoint{4.546658in}{3.003308in}}%
\pgfpathlineto{\pgfqpoint{4.571046in}{3.000954in}}%
\pgfpathlineto{\pgfqpoint{4.622937in}{2.998354in}}%
\pgfpathlineto{\pgfqpoint{4.643175in}{2.996817in}}%
\pgfpathlineto{\pgfqpoint{4.670677in}{2.999257in}}%
\pgfpathlineto{\pgfqpoint{4.698179in}{2.996391in}}%
\pgfpathlineto{\pgfqpoint{4.721530in}{2.992510in}}%
\pgfpathlineto{\pgfqpoint{4.749551in}{2.991787in}}%
\pgfpathlineto{\pgfqpoint{4.773940in}{2.989163in}}%
\pgfpathlineto{\pgfqpoint{4.825831in}{2.990720in}}%
\pgfpathlineto{\pgfqpoint{4.851777in}{2.987632in}}%
\pgfpathlineto{\pgfqpoint{4.886025in}{2.988071in}}%
\pgfpathlineto{\pgfqpoint{4.926500in}{2.987737in}}%
\pgfpathlineto{\pgfqpoint{4.954521in}{2.988848in}}%
\pgfpathlineto{\pgfqpoint{5.030800in}{2.977305in}}%
\pgfpathlineto{\pgfqpoint{5.065567in}{2.975798in}}%
\pgfpathlineto{\pgfqpoint{5.124723in}{2.978241in}}%
\pgfpathlineto{\pgfqpoint{5.209305in}{2.982189in}}%
\pgfpathlineto{\pgfqpoint{5.215013in}{2.981750in}}%
\pgfpathlineto{\pgfqpoint{5.239402in}{2.980481in}}%
\pgfpathlineto{\pgfqpoint{5.265867in}{2.984443in}}%
\pgfpathlineto{\pgfqpoint{5.297001in}{2.980809in}}%
\pgfpathlineto{\pgfqpoint{5.337995in}{2.973681in}}%
\pgfpathlineto{\pgfqpoint{5.345260in}{2.973664in}}%
\pgfpathlineto{\pgfqpoint{5.380027in}{2.973942in}}%
\pgfpathlineto{\pgfqpoint{5.402340in}{2.974229in}}%
\pgfpathlineto{\pgfqpoint{5.443334in}{2.975737in}}%
\pgfpathlineto{\pgfqpoint{5.489517in}{2.973064in}}%
\pgfpathlineto{\pgfqpoint{5.533105in}{2.970331in}}%
\pgfpathlineto{\pgfqpoint{5.556456in}{2.972453in}}%
\pgfpathlineto{\pgfqpoint{5.583958in}{2.975769in}}%
\pgfpathlineto{\pgfqpoint{5.606790in}{2.976011in}}%
\pgfpathlineto{\pgfqpoint{5.631179in}{2.975056in}}%
\pgfpathlineto{\pgfqpoint{5.684627in}{2.978611in}}%
\pgfpathlineto{\pgfqpoint{5.711610in}{2.978373in}}%
\pgfpathlineto{\pgfqpoint{5.757793in}{2.980608in}}%
\pgfpathlineto{\pgfqpoint{5.780625in}{2.979067in}}%
\pgfpathlineto{\pgfqpoint{5.780625in}{2.979067in}}%
\pgfusepath{stroke}%
\end{pgfscope}%
\begin{pgfscope}%
\pgfpathrectangle{\pgfqpoint{0.591528in}{0.585278in}}{\pgfqpoint{5.189097in}{3.264722in}}%
\pgfusepath{clip}%
\pgfsetrectcap%
\pgfsetroundjoin%
\pgfsetlinewidth{0.803000pt}%
\definecolor{currentstroke}{rgb}{0.839216,0.152941,0.156863}%
\pgfsetstrokecolor{currentstroke}%
\pgfsetdash{}{0pt}%
\pgfpathmoveto{\pgfqpoint{0.591528in}{2.986290in}}%
\pgfpathlineto{\pgfqpoint{0.643419in}{3.259080in}}%
\pgfpathlineto{\pgfqpoint{0.669364in}{3.683111in}}%
\pgfpathlineto{\pgfqpoint{0.669883in}{3.679333in}}%
\pgfpathlineto{\pgfqpoint{0.694272in}{3.510216in}}%
\pgfpathlineto{\pgfqpoint{0.696348in}{3.501928in}}%
\pgfpathlineto{\pgfqpoint{0.709320in}{3.486153in}}%
\pgfpathlineto{\pgfqpoint{0.721255in}{3.474894in}}%
\pgfpathlineto{\pgfqpoint{0.749276in}{2.814815in}}%
\pgfpathlineto{\pgfqpoint{0.775741in}{2.455211in}}%
\pgfpathlineto{\pgfqpoint{0.801686in}{2.268220in}}%
\pgfpathlineto{\pgfqpoint{0.814140in}{2.241619in}}%
\pgfpathlineto{\pgfqpoint{0.825037in}{2.214205in}}%
\pgfpathlineto{\pgfqpoint{0.826075in}{2.215942in}}%
\pgfpathlineto{\pgfqpoint{0.837491in}{2.232671in}}%
\pgfpathlineto{\pgfqpoint{0.847869in}{2.244150in}}%
\pgfpathlineto{\pgfqpoint{0.851502in}{2.248296in}}%
\pgfpathlineto{\pgfqpoint{0.865512in}{2.282481in}}%
\pgfpathlineto{\pgfqpoint{0.880042in}{2.317273in}}%
\pgfpathlineto{\pgfqpoint{0.897166in}{2.375124in}}%
\pgfpathlineto{\pgfqpoint{0.911695in}{2.417193in}}%
\pgfpathlineto{\pgfqpoint{0.927781in}{2.457051in}}%
\pgfpathlineto{\pgfqpoint{0.929857in}{2.459811in}}%
\pgfpathlineto{\pgfqpoint{0.940235in}{2.462439in}}%
\pgfpathlineto{\pgfqpoint{0.950094in}{2.462649in}}%
\pgfpathlineto{\pgfqpoint{0.955802in}{2.460989in}}%
\pgfpathlineto{\pgfqpoint{0.967218in}{2.448526in}}%
\pgfpathlineto{\pgfqpoint{0.979153in}{2.432217in}}%
\pgfpathlineto{\pgfqpoint{0.981748in}{2.429816in}}%
\pgfpathlineto{\pgfqpoint{0.991088in}{2.428436in}}%
\pgfpathlineto{\pgfqpoint{1.000948in}{2.424671in}}%
\pgfpathlineto{\pgfqpoint{1.008731in}{2.421747in}}%
\pgfpathlineto{\pgfqpoint{1.018072in}{2.421988in}}%
\pgfpathlineto{\pgfqpoint{1.027412in}{2.420076in}}%
\pgfpathlineto{\pgfqpoint{1.033639in}{2.419018in}}%
\pgfpathlineto{\pgfqpoint{1.044536in}{2.427244in}}%
\pgfpathlineto{\pgfqpoint{1.054914in}{2.432378in}}%
\pgfpathlineto{\pgfqpoint{1.059584in}{2.435166in}}%
\pgfpathlineto{\pgfqpoint{1.072557in}{2.453105in}}%
\pgfpathlineto{\pgfqpoint{1.088643in}{2.474399in}}%
\pgfpathlineto{\pgfqpoint{1.103692in}{2.501151in}}%
\pgfpathlineto{\pgfqpoint{1.113032in}{2.517960in}}%
\pgfpathlineto{\pgfqpoint{1.130156in}{2.556766in}}%
\pgfpathlineto{\pgfqpoint{1.225117in}{2.753221in}}%
\pgfpathlineto{\pgfqpoint{1.263516in}{2.827517in}}%
\pgfpathlineto{\pgfqpoint{1.280121in}{2.855892in}}%
\pgfpathlineto{\pgfqpoint{1.326304in}{2.929537in}}%
\pgfpathlineto{\pgfqpoint{1.355882in}{2.968536in}}%
\pgfpathlineto{\pgfqpoint{1.378714in}{2.995311in}}%
\pgfpathlineto{\pgfqpoint{1.416075in}{3.036545in}}%
\pgfpathlineto{\pgfqpoint{1.425416in}{3.045825in}}%
\pgfpathlineto{\pgfqpoint{1.457069in}{3.067806in}}%
\pgfpathlineto{\pgfqpoint{1.484052in}{3.083562in}}%
\pgfpathlineto{\pgfqpoint{1.504290in}{3.095197in}}%
\pgfpathlineto{\pgfqpoint{1.526603in}{3.102712in}}%
\pgfpathlineto{\pgfqpoint{1.547878in}{3.111146in}}%
\pgfpathlineto{\pgfqpoint{1.554105in}{3.112957in}}%
\pgfpathlineto{\pgfqpoint{1.574343in}{3.115244in}}%
\pgfpathlineto{\pgfqpoint{1.624158in}{3.124883in}}%
\pgfpathlineto{\pgfqpoint{1.637650in}{3.127811in}}%
\pgfpathlineto{\pgfqpoint{1.668265in}{3.132268in}}%
\pgfpathlineto{\pgfqpoint{1.692654in}{3.132802in}}%
\pgfpathlineto{\pgfqpoint{1.735724in}{3.130271in}}%
\pgfpathlineto{\pgfqpoint{1.754923in}{3.126119in}}%
\pgfpathlineto{\pgfqpoint{1.784501in}{3.122631in}}%
\pgfpathlineto{\pgfqpoint{1.796955in}{3.121095in}}%
\pgfpathlineto{\pgfqpoint{1.819787in}{3.120134in}}%
\pgfpathlineto{\pgfqpoint{1.849365in}{3.119046in}}%
\pgfpathlineto{\pgfqpoint{1.868565in}{3.115802in}}%
\pgfpathlineto{\pgfqpoint{1.888802in}{3.109467in}}%
\pgfpathlineto{\pgfqpoint{1.909558in}{3.107220in}}%
\pgfpathlineto{\pgfqpoint{1.920456in}{3.105628in}}%
\pgfpathlineto{\pgfqpoint{1.964563in}{3.096737in}}%
\pgfpathlineto{\pgfqpoint{1.986876in}{3.094199in}}%
\pgfpathlineto{\pgfqpoint{2.020086in}{3.091164in}}%
\pgfpathlineto{\pgfqpoint{2.042399in}{3.087918in}}%
\pgfpathlineto{\pgfqpoint{2.048107in}{3.086378in}}%
\pgfpathlineto{\pgfqpoint{2.071977in}{3.078876in}}%
\pgfpathlineto{\pgfqpoint{2.096366in}{3.073288in}}%
\pgfpathlineto{\pgfqpoint{2.120755in}{3.069348in}}%
\pgfpathlineto{\pgfqpoint{2.155003in}{3.064314in}}%
\pgfpathlineto{\pgfqpoint{2.184062in}{3.057191in}}%
\pgfpathlineto{\pgfqpoint{2.227131in}{3.046859in}}%
\pgfpathlineto{\pgfqpoint{2.253596in}{3.041046in}}%
\pgfpathlineto{\pgfqpoint{2.295108in}{3.040329in}}%
\pgfpathlineto{\pgfqpoint{2.309638in}{3.039248in}}%
\pgfpathlineto{\pgfqpoint{2.336102in}{3.035099in}}%
\pgfpathlineto{\pgfqpoint{2.377096in}{3.030441in}}%
\pgfpathlineto{\pgfqpoint{2.391626in}{3.027354in}}%
\pgfpathlineto{\pgfqpoint{2.414977in}{3.022837in}}%
\pgfpathlineto{\pgfqpoint{2.441960in}{3.020903in}}%
\pgfpathlineto{\pgfqpoint{2.466349in}{3.019671in}}%
\pgfpathlineto{\pgfqpoint{2.490737in}{3.016125in}}%
\pgfpathlineto{\pgfqpoint{2.514607in}{3.014561in}}%
\pgfpathlineto{\pgfqpoint{2.539515in}{3.005129in}}%
\pgfpathlineto{\pgfqpoint{2.578433in}{3.004083in}}%
\pgfpathlineto{\pgfqpoint{2.646929in}{2.997422in}}%
\pgfpathlineto{\pgfqpoint{2.671318in}{2.992498in}}%
\pgfpathlineto{\pgfqpoint{2.715944in}{2.992947in}}%
\pgfpathlineto{\pgfqpoint{2.735144in}{2.994710in}}%
\pgfpathlineto{\pgfqpoint{2.749673in}{2.995151in}}%
\pgfpathlineto{\pgfqpoint{2.809867in}{2.988315in}}%
\pgfpathlineto{\pgfqpoint{2.830623in}{2.988032in}}%
\pgfpathlineto{\pgfqpoint{2.861239in}{2.989148in}}%
\pgfpathlineto{\pgfqpoint{2.887703in}{2.987613in}}%
\pgfpathlineto{\pgfqpoint{2.910016in}{2.986439in}}%
\pgfpathlineto{\pgfqpoint{2.946859in}{2.985858in}}%
\pgfpathlineto{\pgfqpoint{2.966059in}{2.986167in}}%
\pgfpathlineto{\pgfqpoint{2.984739in}{2.985805in}}%
\pgfpathlineto{\pgfqpoint{3.018469in}{2.982229in}}%
\pgfpathlineto{\pgfqpoint{3.077624in}{2.977534in}}%
\pgfpathlineto{\pgfqpoint{3.086965in}{2.977311in}}%
\pgfpathlineto{\pgfqpoint{3.131072in}{2.979974in}}%
\pgfpathlineto{\pgfqpoint{3.161169in}{2.979998in}}%
\pgfpathlineto{\pgfqpoint{3.195936in}{2.977968in}}%
\pgfpathlineto{\pgfqpoint{3.220324in}{2.975922in}}%
\pgfpathlineto{\pgfqpoint{3.326701in}{2.977767in}}%
\pgfpathlineto{\pgfqpoint{3.352646in}{2.980909in}}%
\pgfpathlineto{\pgfqpoint{3.371846in}{2.983350in}}%
\pgfpathlineto{\pgfqpoint{3.458504in}{2.976953in}}%
\pgfpathlineto{\pgfqpoint{3.476147in}{2.974470in}}%
\pgfpathlineto{\pgfqpoint{3.530632in}{2.977928in}}%
\pgfpathlineto{\pgfqpoint{3.552427in}{2.981898in}}%
\pgfpathlineto{\pgfqpoint{3.590826in}{2.979149in}}%
\pgfpathlineto{\pgfqpoint{3.634414in}{2.976831in}}%
\pgfpathlineto{\pgfqpoint{3.657246in}{2.975084in}}%
\pgfpathlineto{\pgfqpoint{3.691494in}{2.978680in}}%
\pgfpathlineto{\pgfqpoint{3.734045in}{2.979970in}}%
\pgfpathlineto{\pgfqpoint{3.761547in}{2.977515in}}%
\pgfpathlineto{\pgfqpoint{3.785936in}{2.979477in}}%
\pgfpathlineto{\pgfqpoint{3.813438in}{2.975760in}}%
\pgfpathlineto{\pgfqpoint{3.845092in}{2.979281in}}%
\pgfpathlineto{\pgfqpoint{3.863254in}{2.981749in}}%
\pgfpathlineto{\pgfqpoint{3.897502in}{2.976985in}}%
\pgfpathlineto{\pgfqpoint{3.923966in}{2.976337in}}%
\pgfpathlineto{\pgfqpoint{3.957695in}{2.976301in}}%
\pgfpathlineto{\pgfqpoint{3.981046in}{2.974573in}}%
\pgfpathlineto{\pgfqpoint{3.995576in}{2.974248in}}%
\pgfpathlineto{\pgfqpoint{4.023597in}{2.976348in}}%
\pgfpathlineto{\pgfqpoint{4.049542in}{2.976832in}}%
\pgfpathlineto{\pgfqpoint{4.073931in}{2.979112in}}%
\pgfpathlineto{\pgfqpoint{4.098839in}{2.979222in}}%
\pgfpathlineto{\pgfqpoint{4.123746in}{2.983005in}}%
\pgfpathlineto{\pgfqpoint{4.159032in}{2.980880in}}%
\pgfpathlineto{\pgfqpoint{4.182902in}{2.981869in}}%
\pgfpathlineto{\pgfqpoint{4.202621in}{2.982688in}}%
\pgfpathlineto{\pgfqpoint{4.230123in}{2.980160in}}%
\pgfpathlineto{\pgfqpoint{4.263852in}{2.983007in}}%
\pgfpathlineto{\pgfqpoint{4.278900in}{2.983992in}}%
\pgfpathlineto{\pgfqpoint{4.306402in}{2.979995in}}%
\pgfpathlineto{\pgfqpoint{4.382682in}{2.983750in}}%
\pgfpathlineto{\pgfqpoint{4.424714in}{2.978810in}}%
\pgfpathlineto{\pgfqpoint{4.436130in}{2.978820in}}%
\pgfpathlineto{\pgfqpoint{4.472454in}{2.982393in}}%
\pgfpathlineto{\pgfqpoint{4.517599in}{2.983923in}}%
\pgfpathlineto{\pgfqpoint{4.546139in}{2.983512in}}%
\pgfpathlineto{\pgfqpoint{4.571046in}{2.982697in}}%
\pgfpathlineto{\pgfqpoint{4.621381in}{2.983332in}}%
\pgfpathlineto{\pgfqpoint{4.643175in}{2.982825in}}%
\pgfpathlineto{\pgfqpoint{4.670158in}{2.986756in}}%
\pgfpathlineto{\pgfqpoint{4.698179in}{2.985370in}}%
\pgfpathlineto{\pgfqpoint{4.721530in}{2.982729in}}%
\pgfpathlineto{\pgfqpoint{4.749032in}{2.983466in}}%
\pgfpathlineto{\pgfqpoint{4.774459in}{2.981991in}}%
\pgfpathlineto{\pgfqpoint{4.825312in}{2.985779in}}%
\pgfpathlineto{\pgfqpoint{4.851777in}{2.983690in}}%
\pgfpathlineto{\pgfqpoint{4.886025in}{2.985453in}}%
\pgfpathlineto{\pgfqpoint{4.926500in}{2.986638in}}%
\pgfpathlineto{\pgfqpoint{4.955040in}{2.988748in}}%
\pgfpathlineto{\pgfqpoint{5.031319in}{2.980170in}}%
\pgfpathlineto{\pgfqpoint{5.067124in}{2.979769in}}%
\pgfpathlineto{\pgfqpoint{5.139771in}{2.984261in}}%
\pgfpathlineto{\pgfqpoint{5.168312in}{2.985593in}}%
\pgfpathlineto{\pgfqpoint{5.192181in}{2.986264in}}%
\pgfpathlineto{\pgfqpoint{5.215013in}{2.987863in}}%
\pgfpathlineto{\pgfqpoint{5.239402in}{2.986902in}}%
\pgfpathlineto{\pgfqpoint{5.265867in}{2.991235in}}%
\pgfpathlineto{\pgfqpoint{5.297520in}{2.988079in}}%
\pgfpathlineto{\pgfqpoint{5.338514in}{2.981791in}}%
\pgfpathlineto{\pgfqpoint{5.345260in}{2.981972in}}%
\pgfpathlineto{\pgfqpoint{5.378989in}{2.982777in}}%
\pgfpathlineto{\pgfqpoint{5.402859in}{2.983319in}}%
\pgfpathlineto{\pgfqpoint{5.442296in}{2.985080in}}%
\pgfpathlineto{\pgfqpoint{5.463052in}{2.984551in}}%
\pgfpathlineto{\pgfqpoint{5.489517in}{2.982676in}}%
\pgfpathlineto{\pgfqpoint{5.534143in}{2.979944in}}%
\pgfpathlineto{\pgfqpoint{5.556975in}{2.981911in}}%
\pgfpathlineto{\pgfqpoint{5.583439in}{2.984789in}}%
\pgfpathlineto{\pgfqpoint{5.606790in}{2.984741in}}%
\pgfpathlineto{\pgfqpoint{5.631698in}{2.983451in}}%
\pgfpathlineto{\pgfqpoint{5.684108in}{2.986035in}}%
\pgfpathlineto{\pgfqpoint{5.712129in}{2.985272in}}%
\pgfpathlineto{\pgfqpoint{5.757793in}{2.986557in}}%
\pgfpathlineto{\pgfqpoint{5.780625in}{2.984558in}}%
\pgfpathlineto{\pgfqpoint{5.780625in}{2.984558in}}%
\pgfusepath{stroke}%
\end{pgfscope}%
\begin{pgfscope}%
\pgfsetrectcap%
\pgfsetmiterjoin%
\pgfsetlinewidth{0.803000pt}%
\definecolor{currentstroke}{rgb}{0.000000,0.000000,0.000000}%
\pgfsetstrokecolor{currentstroke}%
\pgfsetdash{}{0pt}%
\pgfpathmoveto{\pgfqpoint{0.591528in}{0.585278in}}%
\pgfpathlineto{\pgfqpoint{0.591528in}{3.850000in}}%
\pgfusepath{stroke}%
\end{pgfscope}%
\begin{pgfscope}%
\pgfsetrectcap%
\pgfsetmiterjoin%
\pgfsetlinewidth{0.803000pt}%
\definecolor{currentstroke}{rgb}{0.000000,0.000000,0.000000}%
\pgfsetstrokecolor{currentstroke}%
\pgfsetdash{}{0pt}%
\pgfpathmoveto{\pgfqpoint{5.780625in}{0.585278in}}%
\pgfpathlineto{\pgfqpoint{5.780625in}{3.850000in}}%
\pgfusepath{stroke}%
\end{pgfscope}%
\begin{pgfscope}%
\pgfsetrectcap%
\pgfsetmiterjoin%
\pgfsetlinewidth{0.803000pt}%
\definecolor{currentstroke}{rgb}{0.000000,0.000000,0.000000}%
\pgfsetstrokecolor{currentstroke}%
\pgfsetdash{}{0pt}%
\pgfpathmoveto{\pgfqpoint{0.591528in}{0.585278in}}%
\pgfpathlineto{\pgfqpoint{5.780625in}{0.585278in}}%
\pgfusepath{stroke}%
\end{pgfscope}%
\begin{pgfscope}%
\pgfsetrectcap%
\pgfsetmiterjoin%
\pgfsetlinewidth{0.803000pt}%
\definecolor{currentstroke}{rgb}{0.000000,0.000000,0.000000}%
\pgfsetstrokecolor{currentstroke}%
\pgfsetdash{}{0pt}%
\pgfpathmoveto{\pgfqpoint{0.591528in}{3.850000in}}%
\pgfpathlineto{\pgfqpoint{5.780625in}{3.850000in}}%
\pgfusepath{stroke}%
\end{pgfscope}%
\begin{pgfscope}%
\pgfsetbuttcap%
\pgfsetmiterjoin%
\definecolor{currentfill}{rgb}{1.000000,1.000000,1.000000}%
\pgfsetfillcolor{currentfill}%
\pgfsetfillopacity{0.800000}%
\pgfsetlinewidth{1.003750pt}%
\definecolor{currentstroke}{rgb}{0.800000,0.800000,0.800000}%
\pgfsetstrokecolor{currentstroke}%
\pgfsetstrokeopacity{0.800000}%
\pgfsetdash{}{0pt}%
\pgfpathmoveto{\pgfqpoint{3.390575in}{0.654722in}}%
\pgfpathlineto{\pgfqpoint{5.683403in}{0.654722in}}%
\pgfpathquadraticcurveto{\pgfqpoint{5.711181in}{0.654722in}}{\pgfqpoint{5.711181in}{0.682500in}}%
\pgfpathlineto{\pgfqpoint{5.711181in}{1.501944in}}%
\pgfpathquadraticcurveto{\pgfqpoint{5.711181in}{1.529722in}}{\pgfqpoint{5.683403in}{1.529722in}}%
\pgfpathlineto{\pgfqpoint{3.390575in}{1.529722in}}%
\pgfpathquadraticcurveto{\pgfqpoint{3.362797in}{1.529722in}}{\pgfqpoint{3.362797in}{1.501944in}}%
\pgfpathlineto{\pgfqpoint{3.362797in}{0.682500in}}%
\pgfpathquadraticcurveto{\pgfqpoint{3.362797in}{0.654722in}}{\pgfqpoint{3.390575in}{0.654722in}}%
\pgfpathlineto{\pgfqpoint{3.390575in}{0.654722in}}%
\pgfpathclose%
\pgfusepath{stroke,fill}%
\end{pgfscope}%
\begin{pgfscope}%
\pgfsetrectcap%
\pgfsetroundjoin%
\pgfsetlinewidth{0.803000pt}%
\definecolor{currentstroke}{rgb}{0.121569,0.466667,0.705882}%
\pgfsetstrokecolor{currentstroke}%
\pgfsetdash{}{0pt}%
\pgfpathmoveto{\pgfqpoint{3.418353in}{1.418611in}}%
\pgfpathlineto{\pgfqpoint{3.557242in}{1.418611in}}%
\pgfpathlineto{\pgfqpoint{3.696131in}{1.418611in}}%
\pgfusepath{stroke}%
\end{pgfscope}%
\begin{pgfscope}%
\definecolor{textcolor}{rgb}{0.000000,0.000000,0.000000}%
\pgfsetstrokecolor{textcolor}%
\pgfsetfillcolor{textcolor}%
\pgftext[x=3.807242in,y=1.370000in,left,base]{\color{textcolor}\rmfamily\fontsize{10.000000}{12.000000}\selectfont Cart's Position (m)}%
\end{pgfscope}%
\begin{pgfscope}%
\pgfsetrectcap%
\pgfsetroundjoin%
\pgfsetlinewidth{0.803000pt}%
\definecolor{currentstroke}{rgb}{1.000000,0.498039,0.054902}%
\pgfsetstrokecolor{currentstroke}%
\pgfsetdash{}{0pt}%
\pgfpathmoveto{\pgfqpoint{3.418353in}{1.210278in}}%
\pgfpathlineto{\pgfqpoint{3.557242in}{1.210278in}}%
\pgfpathlineto{\pgfqpoint{3.696131in}{1.210278in}}%
\pgfusepath{stroke}%
\end{pgfscope}%
\begin{pgfscope}%
\definecolor{textcolor}{rgb}{0.000000,0.000000,0.000000}%
\pgfsetstrokecolor{textcolor}%
\pgfsetfillcolor{textcolor}%
\pgftext[x=3.807242in,y=1.161667in,left,base]{\color{textcolor}\rmfamily\fontsize{10.000000}{12.000000}\selectfont Pole's angle (rad)}%
\end{pgfscope}%
\begin{pgfscope}%
\pgfsetrectcap%
\pgfsetroundjoin%
\pgfsetlinewidth{0.803000pt}%
\definecolor{currentstroke}{rgb}{0.172549,0.627451,0.172549}%
\pgfsetstrokecolor{currentstroke}%
\pgfsetdash{}{0pt}%
\pgfpathmoveto{\pgfqpoint{3.418353in}{1.001944in}}%
\pgfpathlineto{\pgfqpoint{3.557242in}{1.001944in}}%
\pgfpathlineto{\pgfqpoint{3.696131in}{1.001944in}}%
\pgfusepath{stroke}%
\end{pgfscope}%
\begin{pgfscope}%
\definecolor{textcolor}{rgb}{0.000000,0.000000,0.000000}%
\pgfsetstrokecolor{textcolor}%
\pgfsetfillcolor{textcolor}%
\pgftext[x=3.807242in,y=0.953333in,left,base]{\color{textcolor}\rmfamily\fontsize{10.000000}{12.000000}\selectfont Cart's Velocity (m/s)}%
\end{pgfscope}%
\begin{pgfscope}%
\pgfsetrectcap%
\pgfsetroundjoin%
\pgfsetlinewidth{0.803000pt}%
\definecolor{currentstroke}{rgb}{0.839216,0.152941,0.156863}%
\pgfsetstrokecolor{currentstroke}%
\pgfsetdash{}{0pt}%
\pgfpathmoveto{\pgfqpoint{3.418353in}{0.793611in}}%
\pgfpathlineto{\pgfqpoint{3.557242in}{0.793611in}}%
\pgfpathlineto{\pgfqpoint{3.696131in}{0.793611in}}%
\pgfusepath{stroke}%
\end{pgfscope}%
\begin{pgfscope}%
\definecolor{textcolor}{rgb}{0.000000,0.000000,0.000000}%
\pgfsetstrokecolor{textcolor}%
\pgfsetfillcolor{textcolor}%
\pgftext[x=3.807242in,y=0.745000in,left,base]{\color{textcolor}\rmfamily\fontsize{10.000000}{12.000000}\selectfont Pole's angular velocity (rad/s)}%
\end{pgfscope}%
\end{pgfpicture}%
\makeatother%
\endgroup%

  \caption{State trajectories for DD-SDLQR controller. The plot shows cart position, pole angle, cart velocity, and pole angular velocity over time.}
  \label{fig:states}
\end{figure}

\subsection{Control Input}

The control input force applied to the system is shown in Figure~\ref{fig:input}.

\begin{figure}[htbp]
  \centering
  %% Creator: Matplotlib, PGF backend
%%
%% To include the figure in your LaTeX document, write
%%   \input{<filename>.pgf}
%%
%% Make sure the required packages are loaded in your preamble
%%   \usepackage{pgf}
%%
%% Also ensure that all the required font packages are loaded; for instance,
%% the lmodern package is sometimes necessary when using math font.
%%   \usepackage{lmodern}
%%
%% Figures using additional raster images can only be included by \input if
%% they are in the same directory as the main LaTeX file. For loading figures
%% from other directories you can use the `import` package
%%   \usepackage{import}
%%
%% and then include the figures with
%%   \import{<path to file>}{<filename>.pgf}
%%
%% Matplotlib used the following preamble
%%   
%%   \makeatletter\@ifpackageloaded{underscore}{}{\usepackage[strings]{underscore}}\makeatother
%%
\begingroup%
\makeatletter%
\begin{pgfpicture}%
\pgfpathrectangle{\pgfpointorigin}{\pgfqpoint{6.000000in}{4.000000in}}%
\pgfusepath{use as bounding box, clip}%
\begin{pgfscope}%
\pgfsetbuttcap%
\pgfsetmiterjoin%
\definecolor{currentfill}{rgb}{1.000000,1.000000,1.000000}%
\pgfsetfillcolor{currentfill}%
\pgfsetlinewidth{0.000000pt}%
\definecolor{currentstroke}{rgb}{1.000000,1.000000,1.000000}%
\pgfsetstrokecolor{currentstroke}%
\pgfsetdash{}{0pt}%
\pgfpathmoveto{\pgfqpoint{0.000000in}{0.000000in}}%
\pgfpathlineto{\pgfqpoint{6.000000in}{0.000000in}}%
\pgfpathlineto{\pgfqpoint{6.000000in}{4.000000in}}%
\pgfpathlineto{\pgfqpoint{0.000000in}{4.000000in}}%
\pgfpathlineto{\pgfqpoint{0.000000in}{0.000000in}}%
\pgfpathclose%
\pgfusepath{fill}%
\end{pgfscope}%
\begin{pgfscope}%
\pgfsetbuttcap%
\pgfsetmiterjoin%
\definecolor{currentfill}{rgb}{1.000000,1.000000,1.000000}%
\pgfsetfillcolor{currentfill}%
\pgfsetlinewidth{0.000000pt}%
\definecolor{currentstroke}{rgb}{0.000000,0.000000,0.000000}%
\pgfsetstrokecolor{currentstroke}%
\pgfsetstrokeopacity{0.000000}%
\pgfsetdash{}{0pt}%
\pgfpathmoveto{\pgfqpoint{0.660278in}{0.585278in}}%
\pgfpathlineto{\pgfqpoint{5.780625in}{0.585278in}}%
\pgfpathlineto{\pgfqpoint{5.780625in}{3.850000in}}%
\pgfpathlineto{\pgfqpoint{0.660278in}{3.850000in}}%
\pgfpathlineto{\pgfqpoint{0.660278in}{0.585278in}}%
\pgfpathclose%
\pgfusepath{fill}%
\end{pgfscope}%
\begin{pgfscope}%
\pgfpathrectangle{\pgfqpoint{0.660278in}{0.585278in}}{\pgfqpoint{5.120347in}{3.264722in}}%
\pgfusepath{clip}%
\pgfsetrectcap%
\pgfsetroundjoin%
\pgfsetlinewidth{0.803000pt}%
\definecolor{currentstroke}{rgb}{0.690196,0.690196,0.690196}%
\pgfsetstrokecolor{currentstroke}%
\pgfsetdash{}{0pt}%
\pgfpathmoveto{\pgfqpoint{0.660278in}{0.585278in}}%
\pgfpathlineto{\pgfqpoint{0.660278in}{3.850000in}}%
\pgfusepath{stroke}%
\end{pgfscope}%
\begin{pgfscope}%
\pgfsetbuttcap%
\pgfsetroundjoin%
\definecolor{currentfill}{rgb}{0.000000,0.000000,0.000000}%
\pgfsetfillcolor{currentfill}%
\pgfsetlinewidth{0.803000pt}%
\definecolor{currentstroke}{rgb}{0.000000,0.000000,0.000000}%
\pgfsetstrokecolor{currentstroke}%
\pgfsetdash{}{0pt}%
\pgfsys@defobject{currentmarker}{\pgfqpoint{0.000000in}{-0.048611in}}{\pgfqpoint{0.000000in}{0.000000in}}{%
\pgfpathmoveto{\pgfqpoint{0.000000in}{0.000000in}}%
\pgfpathlineto{\pgfqpoint{0.000000in}{-0.048611in}}%
\pgfusepath{stroke,fill}%
}%
\begin{pgfscope}%
\pgfsys@transformshift{0.660278in}{0.585278in}%
\pgfsys@useobject{currentmarker}{}%
\end{pgfscope}%
\end{pgfscope}%
\begin{pgfscope}%
\definecolor{textcolor}{rgb}{0.000000,0.000000,0.000000}%
\pgfsetstrokecolor{textcolor}%
\pgfsetfillcolor{textcolor}%
\pgftext[x=0.660278in,y=0.488056in,,top]{\color{textcolor}\rmfamily\fontsize{10.000000}{12.000000}\selectfont 0}%
\end{pgfscope}%
\begin{pgfscope}%
\pgfpathrectangle{\pgfqpoint{0.660278in}{0.585278in}}{\pgfqpoint{5.120347in}{3.264722in}}%
\pgfusepath{clip}%
\pgfsetrectcap%
\pgfsetroundjoin%
\pgfsetlinewidth{0.803000pt}%
\definecolor{currentstroke}{rgb}{0.690196,0.690196,0.690196}%
\pgfsetstrokecolor{currentstroke}%
\pgfsetdash{}{0pt}%
\pgfpathmoveto{\pgfqpoint{1.684347in}{0.585278in}}%
\pgfpathlineto{\pgfqpoint{1.684347in}{3.850000in}}%
\pgfusepath{stroke}%
\end{pgfscope}%
\begin{pgfscope}%
\pgfsetbuttcap%
\pgfsetroundjoin%
\definecolor{currentfill}{rgb}{0.000000,0.000000,0.000000}%
\pgfsetfillcolor{currentfill}%
\pgfsetlinewidth{0.803000pt}%
\definecolor{currentstroke}{rgb}{0.000000,0.000000,0.000000}%
\pgfsetstrokecolor{currentstroke}%
\pgfsetdash{}{0pt}%
\pgfsys@defobject{currentmarker}{\pgfqpoint{0.000000in}{-0.048611in}}{\pgfqpoint{0.000000in}{0.000000in}}{%
\pgfpathmoveto{\pgfqpoint{0.000000in}{0.000000in}}%
\pgfpathlineto{\pgfqpoint{0.000000in}{-0.048611in}}%
\pgfusepath{stroke,fill}%
}%
\begin{pgfscope}%
\pgfsys@transformshift{1.684347in}{0.585278in}%
\pgfsys@useobject{currentmarker}{}%
\end{pgfscope}%
\end{pgfscope}%
\begin{pgfscope}%
\definecolor{textcolor}{rgb}{0.000000,0.000000,0.000000}%
\pgfsetstrokecolor{textcolor}%
\pgfsetfillcolor{textcolor}%
\pgftext[x=1.684347in,y=0.488056in,,top]{\color{textcolor}\rmfamily\fontsize{10.000000}{12.000000}\selectfont 2}%
\end{pgfscope}%
\begin{pgfscope}%
\pgfpathrectangle{\pgfqpoint{0.660278in}{0.585278in}}{\pgfqpoint{5.120347in}{3.264722in}}%
\pgfusepath{clip}%
\pgfsetrectcap%
\pgfsetroundjoin%
\pgfsetlinewidth{0.803000pt}%
\definecolor{currentstroke}{rgb}{0.690196,0.690196,0.690196}%
\pgfsetstrokecolor{currentstroke}%
\pgfsetdash{}{0pt}%
\pgfpathmoveto{\pgfqpoint{2.708417in}{0.585278in}}%
\pgfpathlineto{\pgfqpoint{2.708417in}{3.850000in}}%
\pgfusepath{stroke}%
\end{pgfscope}%
\begin{pgfscope}%
\pgfsetbuttcap%
\pgfsetroundjoin%
\definecolor{currentfill}{rgb}{0.000000,0.000000,0.000000}%
\pgfsetfillcolor{currentfill}%
\pgfsetlinewidth{0.803000pt}%
\definecolor{currentstroke}{rgb}{0.000000,0.000000,0.000000}%
\pgfsetstrokecolor{currentstroke}%
\pgfsetdash{}{0pt}%
\pgfsys@defobject{currentmarker}{\pgfqpoint{0.000000in}{-0.048611in}}{\pgfqpoint{0.000000in}{0.000000in}}{%
\pgfpathmoveto{\pgfqpoint{0.000000in}{0.000000in}}%
\pgfpathlineto{\pgfqpoint{0.000000in}{-0.048611in}}%
\pgfusepath{stroke,fill}%
}%
\begin{pgfscope}%
\pgfsys@transformshift{2.708417in}{0.585278in}%
\pgfsys@useobject{currentmarker}{}%
\end{pgfscope}%
\end{pgfscope}%
\begin{pgfscope}%
\definecolor{textcolor}{rgb}{0.000000,0.000000,0.000000}%
\pgfsetstrokecolor{textcolor}%
\pgfsetfillcolor{textcolor}%
\pgftext[x=2.708417in,y=0.488056in,,top]{\color{textcolor}\rmfamily\fontsize{10.000000}{12.000000}\selectfont 4}%
\end{pgfscope}%
\begin{pgfscope}%
\pgfpathrectangle{\pgfqpoint{0.660278in}{0.585278in}}{\pgfqpoint{5.120347in}{3.264722in}}%
\pgfusepath{clip}%
\pgfsetrectcap%
\pgfsetroundjoin%
\pgfsetlinewidth{0.803000pt}%
\definecolor{currentstroke}{rgb}{0.690196,0.690196,0.690196}%
\pgfsetstrokecolor{currentstroke}%
\pgfsetdash{}{0pt}%
\pgfpathmoveto{\pgfqpoint{3.732486in}{0.585278in}}%
\pgfpathlineto{\pgfqpoint{3.732486in}{3.850000in}}%
\pgfusepath{stroke}%
\end{pgfscope}%
\begin{pgfscope}%
\pgfsetbuttcap%
\pgfsetroundjoin%
\definecolor{currentfill}{rgb}{0.000000,0.000000,0.000000}%
\pgfsetfillcolor{currentfill}%
\pgfsetlinewidth{0.803000pt}%
\definecolor{currentstroke}{rgb}{0.000000,0.000000,0.000000}%
\pgfsetstrokecolor{currentstroke}%
\pgfsetdash{}{0pt}%
\pgfsys@defobject{currentmarker}{\pgfqpoint{0.000000in}{-0.048611in}}{\pgfqpoint{0.000000in}{0.000000in}}{%
\pgfpathmoveto{\pgfqpoint{0.000000in}{0.000000in}}%
\pgfpathlineto{\pgfqpoint{0.000000in}{-0.048611in}}%
\pgfusepath{stroke,fill}%
}%
\begin{pgfscope}%
\pgfsys@transformshift{3.732486in}{0.585278in}%
\pgfsys@useobject{currentmarker}{}%
\end{pgfscope}%
\end{pgfscope}%
\begin{pgfscope}%
\definecolor{textcolor}{rgb}{0.000000,0.000000,0.000000}%
\pgfsetstrokecolor{textcolor}%
\pgfsetfillcolor{textcolor}%
\pgftext[x=3.732486in,y=0.488056in,,top]{\color{textcolor}\rmfamily\fontsize{10.000000}{12.000000}\selectfont 6}%
\end{pgfscope}%
\begin{pgfscope}%
\pgfpathrectangle{\pgfqpoint{0.660278in}{0.585278in}}{\pgfqpoint{5.120347in}{3.264722in}}%
\pgfusepath{clip}%
\pgfsetrectcap%
\pgfsetroundjoin%
\pgfsetlinewidth{0.803000pt}%
\definecolor{currentstroke}{rgb}{0.690196,0.690196,0.690196}%
\pgfsetstrokecolor{currentstroke}%
\pgfsetdash{}{0pt}%
\pgfpathmoveto{\pgfqpoint{4.756556in}{0.585278in}}%
\pgfpathlineto{\pgfqpoint{4.756556in}{3.850000in}}%
\pgfusepath{stroke}%
\end{pgfscope}%
\begin{pgfscope}%
\pgfsetbuttcap%
\pgfsetroundjoin%
\definecolor{currentfill}{rgb}{0.000000,0.000000,0.000000}%
\pgfsetfillcolor{currentfill}%
\pgfsetlinewidth{0.803000pt}%
\definecolor{currentstroke}{rgb}{0.000000,0.000000,0.000000}%
\pgfsetstrokecolor{currentstroke}%
\pgfsetdash{}{0pt}%
\pgfsys@defobject{currentmarker}{\pgfqpoint{0.000000in}{-0.048611in}}{\pgfqpoint{0.000000in}{0.000000in}}{%
\pgfpathmoveto{\pgfqpoint{0.000000in}{0.000000in}}%
\pgfpathlineto{\pgfqpoint{0.000000in}{-0.048611in}}%
\pgfusepath{stroke,fill}%
}%
\begin{pgfscope}%
\pgfsys@transformshift{4.756556in}{0.585278in}%
\pgfsys@useobject{currentmarker}{}%
\end{pgfscope}%
\end{pgfscope}%
\begin{pgfscope}%
\definecolor{textcolor}{rgb}{0.000000,0.000000,0.000000}%
\pgfsetstrokecolor{textcolor}%
\pgfsetfillcolor{textcolor}%
\pgftext[x=4.756556in,y=0.488056in,,top]{\color{textcolor}\rmfamily\fontsize{10.000000}{12.000000}\selectfont 8}%
\end{pgfscope}%
\begin{pgfscope}%
\pgfpathrectangle{\pgfqpoint{0.660278in}{0.585278in}}{\pgfqpoint{5.120347in}{3.264722in}}%
\pgfusepath{clip}%
\pgfsetrectcap%
\pgfsetroundjoin%
\pgfsetlinewidth{0.803000pt}%
\definecolor{currentstroke}{rgb}{0.690196,0.690196,0.690196}%
\pgfsetstrokecolor{currentstroke}%
\pgfsetdash{}{0pt}%
\pgfpathmoveto{\pgfqpoint{5.780625in}{0.585278in}}%
\pgfpathlineto{\pgfqpoint{5.780625in}{3.850000in}}%
\pgfusepath{stroke}%
\end{pgfscope}%
\begin{pgfscope}%
\pgfsetbuttcap%
\pgfsetroundjoin%
\definecolor{currentfill}{rgb}{0.000000,0.000000,0.000000}%
\pgfsetfillcolor{currentfill}%
\pgfsetlinewidth{0.803000pt}%
\definecolor{currentstroke}{rgb}{0.000000,0.000000,0.000000}%
\pgfsetstrokecolor{currentstroke}%
\pgfsetdash{}{0pt}%
\pgfsys@defobject{currentmarker}{\pgfqpoint{0.000000in}{-0.048611in}}{\pgfqpoint{0.000000in}{0.000000in}}{%
\pgfpathmoveto{\pgfqpoint{0.000000in}{0.000000in}}%
\pgfpathlineto{\pgfqpoint{0.000000in}{-0.048611in}}%
\pgfusepath{stroke,fill}%
}%
\begin{pgfscope}%
\pgfsys@transformshift{5.780625in}{0.585278in}%
\pgfsys@useobject{currentmarker}{}%
\end{pgfscope}%
\end{pgfscope}%
\begin{pgfscope}%
\definecolor{textcolor}{rgb}{0.000000,0.000000,0.000000}%
\pgfsetstrokecolor{textcolor}%
\pgfsetfillcolor{textcolor}%
\pgftext[x=5.780625in,y=0.488056in,,top]{\color{textcolor}\rmfamily\fontsize{10.000000}{12.000000}\selectfont 10}%
\end{pgfscope}%
\begin{pgfscope}%
\definecolor{textcolor}{rgb}{0.000000,0.000000,0.000000}%
\pgfsetstrokecolor{textcolor}%
\pgfsetfillcolor{textcolor}%
\pgftext[x=3.220451in,y=0.309043in,,top]{\color{textcolor}\rmfamily\fontsize{10.000000}{12.000000}\selectfont Time (s)}%
\end{pgfscope}%
\begin{pgfscope}%
\pgfpathrectangle{\pgfqpoint{0.660278in}{0.585278in}}{\pgfqpoint{5.120347in}{3.264722in}}%
\pgfusepath{clip}%
\pgfsetrectcap%
\pgfsetroundjoin%
\pgfsetlinewidth{0.803000pt}%
\definecolor{currentstroke}{rgb}{0.690196,0.690196,0.690196}%
\pgfsetstrokecolor{currentstroke}%
\pgfsetdash{}{0pt}%
\pgfpathmoveto{\pgfqpoint{0.660278in}{1.120756in}}%
\pgfpathlineto{\pgfqpoint{5.780625in}{1.120756in}}%
\pgfusepath{stroke}%
\end{pgfscope}%
\begin{pgfscope}%
\pgfsetbuttcap%
\pgfsetroundjoin%
\definecolor{currentfill}{rgb}{0.000000,0.000000,0.000000}%
\pgfsetfillcolor{currentfill}%
\pgfsetlinewidth{0.803000pt}%
\definecolor{currentstroke}{rgb}{0.000000,0.000000,0.000000}%
\pgfsetstrokecolor{currentstroke}%
\pgfsetdash{}{0pt}%
\pgfsys@defobject{currentmarker}{\pgfqpoint{-0.048611in}{0.000000in}}{\pgfqpoint{-0.000000in}{0.000000in}}{%
\pgfpathmoveto{\pgfqpoint{-0.000000in}{0.000000in}}%
\pgfpathlineto{\pgfqpoint{-0.048611in}{0.000000in}}%
\pgfusepath{stroke,fill}%
}%
\begin{pgfscope}%
\pgfsys@transformshift{0.660278in}{1.120756in}%
\pgfsys@useobject{currentmarker}{}%
\end{pgfscope}%
\end{pgfscope}%
\begin{pgfscope}%
\definecolor{textcolor}{rgb}{0.000000,0.000000,0.000000}%
\pgfsetstrokecolor{textcolor}%
\pgfsetfillcolor{textcolor}%
\pgftext[x=0.316141in, y=1.072531in, left, base]{\color{textcolor}\rmfamily\fontsize{10.000000}{12.000000}\selectfont \ensuremath{-}30}%
\end{pgfscope}%
\begin{pgfscope}%
\pgfpathrectangle{\pgfqpoint{0.660278in}{0.585278in}}{\pgfqpoint{5.120347in}{3.264722in}}%
\pgfusepath{clip}%
\pgfsetrectcap%
\pgfsetroundjoin%
\pgfsetlinewidth{0.803000pt}%
\definecolor{currentstroke}{rgb}{0.690196,0.690196,0.690196}%
\pgfsetstrokecolor{currentstroke}%
\pgfsetdash{}{0pt}%
\pgfpathmoveto{\pgfqpoint{0.660278in}{1.732159in}}%
\pgfpathlineto{\pgfqpoint{5.780625in}{1.732159in}}%
\pgfusepath{stroke}%
\end{pgfscope}%
\begin{pgfscope}%
\pgfsetbuttcap%
\pgfsetroundjoin%
\definecolor{currentfill}{rgb}{0.000000,0.000000,0.000000}%
\pgfsetfillcolor{currentfill}%
\pgfsetlinewidth{0.803000pt}%
\definecolor{currentstroke}{rgb}{0.000000,0.000000,0.000000}%
\pgfsetstrokecolor{currentstroke}%
\pgfsetdash{}{0pt}%
\pgfsys@defobject{currentmarker}{\pgfqpoint{-0.048611in}{0.000000in}}{\pgfqpoint{-0.000000in}{0.000000in}}{%
\pgfpathmoveto{\pgfqpoint{-0.000000in}{0.000000in}}%
\pgfpathlineto{\pgfqpoint{-0.048611in}{0.000000in}}%
\pgfusepath{stroke,fill}%
}%
\begin{pgfscope}%
\pgfsys@transformshift{0.660278in}{1.732159in}%
\pgfsys@useobject{currentmarker}{}%
\end{pgfscope}%
\end{pgfscope}%
\begin{pgfscope}%
\definecolor{textcolor}{rgb}{0.000000,0.000000,0.000000}%
\pgfsetstrokecolor{textcolor}%
\pgfsetfillcolor{textcolor}%
\pgftext[x=0.316141in, y=1.683933in, left, base]{\color{textcolor}\rmfamily\fontsize{10.000000}{12.000000}\selectfont \ensuremath{-}20}%
\end{pgfscope}%
\begin{pgfscope}%
\pgfpathrectangle{\pgfqpoint{0.660278in}{0.585278in}}{\pgfqpoint{5.120347in}{3.264722in}}%
\pgfusepath{clip}%
\pgfsetrectcap%
\pgfsetroundjoin%
\pgfsetlinewidth{0.803000pt}%
\definecolor{currentstroke}{rgb}{0.690196,0.690196,0.690196}%
\pgfsetstrokecolor{currentstroke}%
\pgfsetdash{}{0pt}%
\pgfpathmoveto{\pgfqpoint{0.660278in}{2.343561in}}%
\pgfpathlineto{\pgfqpoint{5.780625in}{2.343561in}}%
\pgfusepath{stroke}%
\end{pgfscope}%
\begin{pgfscope}%
\pgfsetbuttcap%
\pgfsetroundjoin%
\definecolor{currentfill}{rgb}{0.000000,0.000000,0.000000}%
\pgfsetfillcolor{currentfill}%
\pgfsetlinewidth{0.803000pt}%
\definecolor{currentstroke}{rgb}{0.000000,0.000000,0.000000}%
\pgfsetstrokecolor{currentstroke}%
\pgfsetdash{}{0pt}%
\pgfsys@defobject{currentmarker}{\pgfqpoint{-0.048611in}{0.000000in}}{\pgfqpoint{-0.000000in}{0.000000in}}{%
\pgfpathmoveto{\pgfqpoint{-0.000000in}{0.000000in}}%
\pgfpathlineto{\pgfqpoint{-0.048611in}{0.000000in}}%
\pgfusepath{stroke,fill}%
}%
\begin{pgfscope}%
\pgfsys@transformshift{0.660278in}{2.343561in}%
\pgfsys@useobject{currentmarker}{}%
\end{pgfscope}%
\end{pgfscope}%
\begin{pgfscope}%
\definecolor{textcolor}{rgb}{0.000000,0.000000,0.000000}%
\pgfsetstrokecolor{textcolor}%
\pgfsetfillcolor{textcolor}%
\pgftext[x=0.316141in, y=2.295336in, left, base]{\color{textcolor}\rmfamily\fontsize{10.000000}{12.000000}\selectfont \ensuremath{-}10}%
\end{pgfscope}%
\begin{pgfscope}%
\pgfpathrectangle{\pgfqpoint{0.660278in}{0.585278in}}{\pgfqpoint{5.120347in}{3.264722in}}%
\pgfusepath{clip}%
\pgfsetrectcap%
\pgfsetroundjoin%
\pgfsetlinewidth{0.803000pt}%
\definecolor{currentstroke}{rgb}{0.690196,0.690196,0.690196}%
\pgfsetstrokecolor{currentstroke}%
\pgfsetdash{}{0pt}%
\pgfpathmoveto{\pgfqpoint{0.660278in}{2.954964in}}%
\pgfpathlineto{\pgfqpoint{5.780625in}{2.954964in}}%
\pgfusepath{stroke}%
\end{pgfscope}%
\begin{pgfscope}%
\pgfsetbuttcap%
\pgfsetroundjoin%
\definecolor{currentfill}{rgb}{0.000000,0.000000,0.000000}%
\pgfsetfillcolor{currentfill}%
\pgfsetlinewidth{0.803000pt}%
\definecolor{currentstroke}{rgb}{0.000000,0.000000,0.000000}%
\pgfsetstrokecolor{currentstroke}%
\pgfsetdash{}{0pt}%
\pgfsys@defobject{currentmarker}{\pgfqpoint{-0.048611in}{0.000000in}}{\pgfqpoint{-0.000000in}{0.000000in}}{%
\pgfpathmoveto{\pgfqpoint{-0.000000in}{0.000000in}}%
\pgfpathlineto{\pgfqpoint{-0.048611in}{0.000000in}}%
\pgfusepath{stroke,fill}%
}%
\begin{pgfscope}%
\pgfsys@transformshift{0.660278in}{2.954964in}%
\pgfsys@useobject{currentmarker}{}%
\end{pgfscope}%
\end{pgfscope}%
\begin{pgfscope}%
\definecolor{textcolor}{rgb}{0.000000,0.000000,0.000000}%
\pgfsetstrokecolor{textcolor}%
\pgfsetfillcolor{textcolor}%
\pgftext[x=0.493611in, y=2.906739in, left, base]{\color{textcolor}\rmfamily\fontsize{10.000000}{12.000000}\selectfont 0}%
\end{pgfscope}%
\begin{pgfscope}%
\pgfpathrectangle{\pgfqpoint{0.660278in}{0.585278in}}{\pgfqpoint{5.120347in}{3.264722in}}%
\pgfusepath{clip}%
\pgfsetrectcap%
\pgfsetroundjoin%
\pgfsetlinewidth{0.803000pt}%
\definecolor{currentstroke}{rgb}{0.690196,0.690196,0.690196}%
\pgfsetstrokecolor{currentstroke}%
\pgfsetdash{}{0pt}%
\pgfpathmoveto{\pgfqpoint{0.660278in}{3.566366in}}%
\pgfpathlineto{\pgfqpoint{5.780625in}{3.566366in}}%
\pgfusepath{stroke}%
\end{pgfscope}%
\begin{pgfscope}%
\pgfsetbuttcap%
\pgfsetroundjoin%
\definecolor{currentfill}{rgb}{0.000000,0.000000,0.000000}%
\pgfsetfillcolor{currentfill}%
\pgfsetlinewidth{0.803000pt}%
\definecolor{currentstroke}{rgb}{0.000000,0.000000,0.000000}%
\pgfsetstrokecolor{currentstroke}%
\pgfsetdash{}{0pt}%
\pgfsys@defobject{currentmarker}{\pgfqpoint{-0.048611in}{0.000000in}}{\pgfqpoint{-0.000000in}{0.000000in}}{%
\pgfpathmoveto{\pgfqpoint{-0.000000in}{0.000000in}}%
\pgfpathlineto{\pgfqpoint{-0.048611in}{0.000000in}}%
\pgfusepath{stroke,fill}%
}%
\begin{pgfscope}%
\pgfsys@transformshift{0.660278in}{3.566366in}%
\pgfsys@useobject{currentmarker}{}%
\end{pgfscope}%
\end{pgfscope}%
\begin{pgfscope}%
\definecolor{textcolor}{rgb}{0.000000,0.000000,0.000000}%
\pgfsetstrokecolor{textcolor}%
\pgfsetfillcolor{textcolor}%
\pgftext[x=0.424166in, y=3.518141in, left, base]{\color{textcolor}\rmfamily\fontsize{10.000000}{12.000000}\selectfont 10}%
\end{pgfscope}%
\begin{pgfscope}%
\definecolor{textcolor}{rgb}{0.000000,0.000000,0.000000}%
\pgfsetstrokecolor{textcolor}%
\pgfsetfillcolor{textcolor}%
\pgftext[x=0.260586in,y=2.217639in,,bottom,rotate=90.000000]{\color{textcolor}\rmfamily\fontsize{10.000000}{12.000000}\selectfont Input Force (N)}%
\end{pgfscope}%
\begin{pgfscope}%
\pgfpathrectangle{\pgfqpoint{0.660278in}{0.585278in}}{\pgfqpoint{5.120347in}{3.264722in}}%
\pgfusepath{clip}%
\pgfsetrectcap%
\pgfsetroundjoin%
\pgfsetlinewidth{0.803000pt}%
\definecolor{currentstroke}{rgb}{0.000000,0.000000,1.000000}%
\pgfsetstrokecolor{currentstroke}%
\pgfsetdash{}{0pt}%
\pgfpathmoveto{\pgfqpoint{0.660278in}{2.958001in}}%
\pgfpathlineto{\pgfqpoint{0.685367in}{2.958001in}}%
\pgfpathlineto{\pgfqpoint{0.686904in}{2.954118in}}%
\pgfpathlineto{\pgfqpoint{0.710969in}{2.954118in}}%
\pgfpathlineto{\pgfqpoint{0.712505in}{3.701604in}}%
\pgfpathlineto{\pgfqpoint{0.736571in}{3.701604in}}%
\pgfpathlineto{\pgfqpoint{0.738107in}{2.024774in}}%
\pgfpathlineto{\pgfqpoint{0.762173in}{2.024774in}}%
\pgfpathlineto{\pgfqpoint{0.763709in}{2.390157in}}%
\pgfpathlineto{\pgfqpoint{0.787774in}{2.390157in}}%
\pgfpathlineto{\pgfqpoint{0.789311in}{0.733674in}}%
\pgfpathlineto{\pgfqpoint{0.813376in}{0.733674in}}%
\pgfpathlineto{\pgfqpoint{0.814912in}{1.446775in}}%
\pgfpathlineto{\pgfqpoint{0.838978in}{1.446775in}}%
\pgfpathlineto{\pgfqpoint{0.840514in}{1.938679in}}%
\pgfpathlineto{\pgfqpoint{0.864580in}{1.938679in}}%
\pgfpathlineto{\pgfqpoint{0.866116in}{2.377902in}}%
\pgfpathlineto{\pgfqpoint{0.890181in}{2.377902in}}%
\pgfpathlineto{\pgfqpoint{0.891717in}{2.685206in}}%
\pgfpathlineto{\pgfqpoint{0.915783in}{2.685206in}}%
\pgfpathlineto{\pgfqpoint{0.917319in}{2.814268in}}%
\pgfpathlineto{\pgfqpoint{0.941385in}{2.814268in}}%
\pgfpathlineto{\pgfqpoint{0.942921in}{2.942924in}}%
\pgfpathlineto{\pgfqpoint{0.966987in}{2.942924in}}%
\pgfpathlineto{\pgfqpoint{0.968523in}{2.936264in}}%
\pgfpathlineto{\pgfqpoint{0.992588in}{2.936264in}}%
\pgfpathlineto{\pgfqpoint{0.994124in}{2.805529in}}%
\pgfpathlineto{\pgfqpoint{1.018190in}{2.805529in}}%
\pgfpathlineto{\pgfqpoint{1.019726in}{2.753895in}}%
\pgfpathlineto{\pgfqpoint{1.043792in}{2.753895in}}%
\pgfpathlineto{\pgfqpoint{1.045328in}{2.861632in}}%
\pgfpathlineto{\pgfqpoint{1.069394in}{2.861632in}}%
\pgfpathlineto{\pgfqpoint{1.070930in}{2.920131in}}%
\pgfpathlineto{\pgfqpoint{1.094995in}{2.920131in}}%
\pgfpathlineto{\pgfqpoint{1.096531in}{3.015637in}}%
\pgfpathlineto{\pgfqpoint{1.120597in}{3.015637in}}%
\pgfpathlineto{\pgfqpoint{1.122133in}{3.105371in}}%
\pgfpathlineto{\pgfqpoint{1.146199in}{3.105371in}}%
\pgfpathlineto{\pgfqpoint{1.147735in}{3.181092in}}%
\pgfpathlineto{\pgfqpoint{1.171800in}{3.181092in}}%
\pgfpathlineto{\pgfqpoint{1.173337in}{3.252245in}}%
\pgfpathlineto{\pgfqpoint{1.197402in}{3.252245in}}%
\pgfpathlineto{\pgfqpoint{1.198938in}{3.276476in}}%
\pgfpathlineto{\pgfqpoint{1.223004in}{3.276476in}}%
\pgfpathlineto{\pgfqpoint{1.224540in}{3.308258in}}%
\pgfpathlineto{\pgfqpoint{1.248606in}{3.308258in}}%
\pgfpathlineto{\pgfqpoint{1.250142in}{3.319314in}}%
\pgfpathlineto{\pgfqpoint{1.274207in}{3.319314in}}%
\pgfpathlineto{\pgfqpoint{1.275744in}{3.341088in}}%
\pgfpathlineto{\pgfqpoint{1.299809in}{3.341088in}}%
\pgfpathlineto{\pgfqpoint{1.301345in}{3.353568in}}%
\pgfpathlineto{\pgfqpoint{1.325411in}{3.353568in}}%
\pgfpathlineto{\pgfqpoint{1.326947in}{3.346270in}}%
\pgfpathlineto{\pgfqpoint{1.351013in}{3.346270in}}%
\pgfpathlineto{\pgfqpoint{1.352549in}{3.355274in}}%
\pgfpathlineto{\pgfqpoint{1.376614in}{3.355274in}}%
\pgfpathlineto{\pgfqpoint{1.378150in}{3.341123in}}%
\pgfpathlineto{\pgfqpoint{1.402216in}{3.341123in}}%
\pgfpathlineto{\pgfqpoint{1.403752in}{3.335127in}}%
\pgfpathlineto{\pgfqpoint{1.427818in}{3.335127in}}%
\pgfpathlineto{\pgfqpoint{1.429354in}{3.322265in}}%
\pgfpathlineto{\pgfqpoint{1.453420in}{3.322265in}}%
\pgfpathlineto{\pgfqpoint{1.454956in}{3.325501in}}%
\pgfpathlineto{\pgfqpoint{1.479021in}{3.325501in}}%
\pgfpathlineto{\pgfqpoint{1.480557in}{3.294571in}}%
\pgfpathlineto{\pgfqpoint{1.504623in}{3.294571in}}%
\pgfpathlineto{\pgfqpoint{1.506159in}{3.273804in}}%
\pgfpathlineto{\pgfqpoint{1.555827in}{3.272996in}}%
\pgfpathlineto{\pgfqpoint{1.557363in}{3.242348in}}%
\pgfpathlineto{\pgfqpoint{1.581428in}{3.242348in}}%
\pgfpathlineto{\pgfqpoint{1.582964in}{3.238586in}}%
\pgfpathlineto{\pgfqpoint{1.607030in}{3.238586in}}%
\pgfpathlineto{\pgfqpoint{1.608566in}{3.208429in}}%
\pgfpathlineto{\pgfqpoint{1.632632in}{3.208429in}}%
\pgfpathlineto{\pgfqpoint{1.634168in}{3.201278in}}%
\pgfpathlineto{\pgfqpoint{1.658233in}{3.201278in}}%
\pgfpathlineto{\pgfqpoint{1.659770in}{3.197310in}}%
\pgfpathlineto{\pgfqpoint{1.683835in}{3.197310in}}%
\pgfpathlineto{\pgfqpoint{1.685371in}{3.183099in}}%
\pgfpathlineto{\pgfqpoint{1.709437in}{3.183099in}}%
\pgfpathlineto{\pgfqpoint{1.710973in}{3.161448in}}%
\pgfpathlineto{\pgfqpoint{1.735039in}{3.161448in}}%
\pgfpathlineto{\pgfqpoint{1.736575in}{3.143373in}}%
\pgfpathlineto{\pgfqpoint{1.760640in}{3.143373in}}%
\pgfpathlineto{\pgfqpoint{1.762177in}{3.128464in}}%
\pgfpathlineto{\pgfqpoint{1.786242in}{3.128464in}}%
\pgfpathlineto{\pgfqpoint{1.787778in}{3.107436in}}%
\pgfpathlineto{\pgfqpoint{1.811844in}{3.107436in}}%
\pgfpathlineto{\pgfqpoint{1.813380in}{3.103343in}}%
\pgfpathlineto{\pgfqpoint{1.837446in}{3.103343in}}%
\pgfpathlineto{\pgfqpoint{1.838982in}{3.094364in}}%
\pgfpathlineto{\pgfqpoint{1.863047in}{3.094364in}}%
\pgfpathlineto{\pgfqpoint{1.864583in}{3.091233in}}%
\pgfpathlineto{\pgfqpoint{1.888649in}{3.091233in}}%
\pgfpathlineto{\pgfqpoint{1.890185in}{3.071141in}}%
\pgfpathlineto{\pgfqpoint{1.914251in}{3.071141in}}%
\pgfpathlineto{\pgfqpoint{1.915787in}{3.045694in}}%
\pgfpathlineto{\pgfqpoint{1.939853in}{3.045694in}}%
\pgfpathlineto{\pgfqpoint{1.941389in}{3.052271in}}%
\pgfpathlineto{\pgfqpoint{1.965454in}{3.052271in}}%
\pgfpathlineto{\pgfqpoint{1.966990in}{3.035995in}}%
\pgfpathlineto{\pgfqpoint{1.991056in}{3.035995in}}%
\pgfpathlineto{\pgfqpoint{1.992592in}{3.024914in}}%
\pgfpathlineto{\pgfqpoint{2.042259in}{3.024739in}}%
\pgfpathlineto{\pgfqpoint{2.043796in}{3.016275in}}%
\pgfpathlineto{\pgfqpoint{2.067861in}{3.016275in}}%
\pgfpathlineto{\pgfqpoint{2.069397in}{3.004469in}}%
\pgfpathlineto{\pgfqpoint{2.093463in}{3.004469in}}%
\pgfpathlineto{\pgfqpoint{2.094999in}{2.983885in}}%
\pgfpathlineto{\pgfqpoint{2.144666in}{2.982940in}}%
\pgfpathlineto{\pgfqpoint{2.146203in}{2.981169in}}%
\pgfpathlineto{\pgfqpoint{2.170268in}{2.981169in}}%
\pgfpathlineto{\pgfqpoint{2.171804in}{2.977986in}}%
\pgfpathlineto{\pgfqpoint{2.195870in}{2.977986in}}%
\pgfpathlineto{\pgfqpoint{2.197406in}{2.960914in}}%
\pgfpathlineto{\pgfqpoint{2.221472in}{2.960914in}}%
\pgfpathlineto{\pgfqpoint{2.223008in}{2.959214in}}%
\pgfpathlineto{\pgfqpoint{2.247073in}{2.959214in}}%
\pgfpathlineto{\pgfqpoint{2.248609in}{2.949344in}}%
\pgfpathlineto{\pgfqpoint{2.272675in}{2.949344in}}%
\pgfpathlineto{\pgfqpoint{2.274211in}{2.947171in}}%
\pgfpathlineto{\pgfqpoint{2.298277in}{2.947171in}}%
\pgfpathlineto{\pgfqpoint{2.299813in}{2.957791in}}%
\pgfpathlineto{\pgfqpoint{2.323879in}{2.957791in}}%
\pgfpathlineto{\pgfqpoint{2.325415in}{2.953992in}}%
\pgfpathlineto{\pgfqpoint{2.349480in}{2.953992in}}%
\pgfpathlineto{\pgfqpoint{2.351016in}{2.937659in}}%
\pgfpathlineto{\pgfqpoint{2.375082in}{2.937659in}}%
\pgfpathlineto{\pgfqpoint{2.376618in}{2.941377in}}%
\pgfpathlineto{\pgfqpoint{2.400684in}{2.941377in}}%
\pgfpathlineto{\pgfqpoint{2.402220in}{2.932060in}}%
\pgfpathlineto{\pgfqpoint{2.426286in}{2.932060in}}%
\pgfpathlineto{\pgfqpoint{2.427822in}{2.922781in}}%
\pgfpathlineto{\pgfqpoint{2.451887in}{2.922781in}}%
\pgfpathlineto{\pgfqpoint{2.453423in}{2.929129in}}%
\pgfpathlineto{\pgfqpoint{2.477489in}{2.929129in}}%
\pgfpathlineto{\pgfqpoint{2.479025in}{2.933449in}}%
\pgfpathlineto{\pgfqpoint{2.503091in}{2.933449in}}%
\pgfpathlineto{\pgfqpoint{2.504627in}{2.918111in}}%
\pgfpathlineto{\pgfqpoint{2.528692in}{2.918111in}}%
\pgfpathlineto{\pgfqpoint{2.530229in}{2.927401in}}%
\pgfpathlineto{\pgfqpoint{2.554294in}{2.927401in}}%
\pgfpathlineto{\pgfqpoint{2.555830in}{2.897157in}}%
\pgfpathlineto{\pgfqpoint{2.579896in}{2.897157in}}%
\pgfpathlineto{\pgfqpoint{2.581432in}{2.925294in}}%
\pgfpathlineto{\pgfqpoint{2.605498in}{2.925294in}}%
\pgfpathlineto{\pgfqpoint{2.607034in}{2.917284in}}%
\pgfpathlineto{\pgfqpoint{2.631099in}{2.917284in}}%
\pgfpathlineto{\pgfqpoint{2.632636in}{2.913939in}}%
\pgfpathlineto{\pgfqpoint{2.656701in}{2.913939in}}%
\pgfpathlineto{\pgfqpoint{2.658237in}{2.916518in}}%
\pgfpathlineto{\pgfqpoint{2.682303in}{2.916518in}}%
\pgfpathlineto{\pgfqpoint{2.683839in}{2.902791in}}%
\pgfpathlineto{\pgfqpoint{2.707905in}{2.902791in}}%
\pgfpathlineto{\pgfqpoint{2.709441in}{2.918245in}}%
\pgfpathlineto{\pgfqpoint{2.733506in}{2.918245in}}%
\pgfpathlineto{\pgfqpoint{2.735042in}{2.920839in}}%
\pgfpathlineto{\pgfqpoint{2.759108in}{2.920839in}}%
\pgfpathlineto{\pgfqpoint{2.760644in}{2.925403in}}%
\pgfpathlineto{\pgfqpoint{2.784710in}{2.925403in}}%
\pgfpathlineto{\pgfqpoint{2.786246in}{2.908362in}}%
\pgfpathlineto{\pgfqpoint{2.835913in}{2.908155in}}%
\pgfpathlineto{\pgfqpoint{2.837449in}{2.912051in}}%
\pgfpathlineto{\pgfqpoint{2.861515in}{2.912051in}}%
\pgfpathlineto{\pgfqpoint{2.863051in}{2.921655in}}%
\pgfpathlineto{\pgfqpoint{2.887117in}{2.921655in}}%
\pgfpathlineto{\pgfqpoint{2.888653in}{2.915125in}}%
\pgfpathlineto{\pgfqpoint{2.912719in}{2.915125in}}%
\pgfpathlineto{\pgfqpoint{2.914255in}{2.909046in}}%
\pgfpathlineto{\pgfqpoint{2.938320in}{2.909046in}}%
\pgfpathlineto{\pgfqpoint{2.939856in}{2.917093in}}%
\pgfpathlineto{\pgfqpoint{2.963922in}{2.917093in}}%
\pgfpathlineto{\pgfqpoint{2.965458in}{2.913238in}}%
\pgfpathlineto{\pgfqpoint{2.989524in}{2.913238in}}%
\pgfpathlineto{\pgfqpoint{2.991060in}{2.918911in}}%
\pgfpathlineto{\pgfqpoint{3.015125in}{2.918911in}}%
\pgfpathlineto{\pgfqpoint{3.016662in}{2.906414in}}%
\pgfpathlineto{\pgfqpoint{3.040727in}{2.906414in}}%
\pgfpathlineto{\pgfqpoint{3.042263in}{2.912108in}}%
\pgfpathlineto{\pgfqpoint{3.066329in}{2.912108in}}%
\pgfpathlineto{\pgfqpoint{3.067865in}{2.913510in}}%
\pgfpathlineto{\pgfqpoint{3.091931in}{2.913510in}}%
\pgfpathlineto{\pgfqpoint{3.093467in}{2.908103in}}%
\pgfpathlineto{\pgfqpoint{3.117532in}{2.908103in}}%
\pgfpathlineto{\pgfqpoint{3.119069in}{2.923697in}}%
\pgfpathlineto{\pgfqpoint{3.143134in}{2.923697in}}%
\pgfpathlineto{\pgfqpoint{3.144670in}{2.921302in}}%
\pgfpathlineto{\pgfqpoint{3.168736in}{2.921302in}}%
\pgfpathlineto{\pgfqpoint{3.170272in}{2.918838in}}%
\pgfpathlineto{\pgfqpoint{3.194338in}{2.918838in}}%
\pgfpathlineto{\pgfqpoint{3.195874in}{2.917339in}}%
\pgfpathlineto{\pgfqpoint{3.219939in}{2.917339in}}%
\pgfpathlineto{\pgfqpoint{3.221475in}{2.910996in}}%
\pgfpathlineto{\pgfqpoint{3.245541in}{2.910996in}}%
\pgfpathlineto{\pgfqpoint{3.247077in}{2.921255in}}%
\pgfpathlineto{\pgfqpoint{3.271143in}{2.921255in}}%
\pgfpathlineto{\pgfqpoint{3.272679in}{2.922974in}}%
\pgfpathlineto{\pgfqpoint{3.296745in}{2.922974in}}%
\pgfpathlineto{\pgfqpoint{3.298281in}{2.920278in}}%
\pgfpathlineto{\pgfqpoint{3.322346in}{2.920278in}}%
\pgfpathlineto{\pgfqpoint{3.323882in}{2.926669in}}%
\pgfpathlineto{\pgfqpoint{3.347948in}{2.926669in}}%
\pgfpathlineto{\pgfqpoint{3.349484in}{2.929587in}}%
\pgfpathlineto{\pgfqpoint{3.373550in}{2.929587in}}%
\pgfpathlineto{\pgfqpoint{3.375086in}{2.937217in}}%
\pgfpathlineto{\pgfqpoint{3.399152in}{2.937217in}}%
\pgfpathlineto{\pgfqpoint{3.400688in}{2.920522in}}%
\pgfpathlineto{\pgfqpoint{3.424753in}{2.920522in}}%
\pgfpathlineto{\pgfqpoint{3.426289in}{2.922663in}}%
\pgfpathlineto{\pgfqpoint{3.475957in}{2.921724in}}%
\pgfpathlineto{\pgfqpoint{3.477493in}{2.911870in}}%
\pgfpathlineto{\pgfqpoint{3.501558in}{2.911870in}}%
\pgfpathlineto{\pgfqpoint{3.503095in}{2.929902in}}%
\pgfpathlineto{\pgfqpoint{3.552762in}{2.930501in}}%
\pgfpathlineto{\pgfqpoint{3.554298in}{2.944836in}}%
\pgfpathlineto{\pgfqpoint{3.578364in}{2.944836in}}%
\pgfpathlineto{\pgfqpoint{3.579900in}{2.921952in}}%
\pgfpathlineto{\pgfqpoint{3.603965in}{2.921952in}}%
\pgfpathlineto{\pgfqpoint{3.605502in}{2.927750in}}%
\pgfpathlineto{\pgfqpoint{3.655169in}{2.927145in}}%
\pgfpathlineto{\pgfqpoint{3.656705in}{2.922363in}}%
\pgfpathlineto{\pgfqpoint{3.680771in}{2.922363in}}%
\pgfpathlineto{\pgfqpoint{3.682307in}{2.941130in}}%
\pgfpathlineto{\pgfqpoint{3.706372in}{2.941130in}}%
\pgfpathlineto{\pgfqpoint{3.707908in}{2.936337in}}%
\pgfpathlineto{\pgfqpoint{3.757576in}{2.935671in}}%
\pgfpathlineto{\pgfqpoint{3.759112in}{2.924315in}}%
\pgfpathlineto{\pgfqpoint{3.783178in}{2.924315in}}%
\pgfpathlineto{\pgfqpoint{3.784714in}{2.942524in}}%
\pgfpathlineto{\pgfqpoint{3.808779in}{2.942524in}}%
\pgfpathlineto{\pgfqpoint{3.810315in}{2.922336in}}%
\pgfpathlineto{\pgfqpoint{3.834381in}{2.922336in}}%
\pgfpathlineto{\pgfqpoint{3.835917in}{2.940738in}}%
\pgfpathlineto{\pgfqpoint{3.859983in}{2.940738in}}%
\pgfpathlineto{\pgfqpoint{3.861519in}{2.949471in}}%
\pgfpathlineto{\pgfqpoint{3.885584in}{2.949471in}}%
\pgfpathlineto{\pgfqpoint{3.887121in}{2.924522in}}%
\pgfpathlineto{\pgfqpoint{3.911186in}{2.924522in}}%
\pgfpathlineto{\pgfqpoint{3.912722in}{2.932713in}}%
\pgfpathlineto{\pgfqpoint{3.936788in}{2.932713in}}%
\pgfpathlineto{\pgfqpoint{3.938324in}{2.940287in}}%
\pgfpathlineto{\pgfqpoint{3.962390in}{2.940287in}}%
\pgfpathlineto{\pgfqpoint{3.963926in}{2.937201in}}%
\pgfpathlineto{\pgfqpoint{3.987991in}{2.937201in}}%
\pgfpathlineto{\pgfqpoint{3.989528in}{2.933122in}}%
\pgfpathlineto{\pgfqpoint{4.013593in}{2.933122in}}%
\pgfpathlineto{\pgfqpoint{4.015129in}{2.948404in}}%
\pgfpathlineto{\pgfqpoint{4.039195in}{2.948404in}}%
\pgfpathlineto{\pgfqpoint{4.040731in}{2.939670in}}%
\pgfpathlineto{\pgfqpoint{4.064797in}{2.939670in}}%
\pgfpathlineto{\pgfqpoint{4.066333in}{2.951696in}}%
\pgfpathlineto{\pgfqpoint{4.090398in}{2.951696in}}%
\pgfpathlineto{\pgfqpoint{4.091934in}{2.940158in}}%
\pgfpathlineto{\pgfqpoint{4.116000in}{2.940158in}}%
\pgfpathlineto{\pgfqpoint{4.117536in}{2.957030in}}%
\pgfpathlineto{\pgfqpoint{4.141602in}{2.957030in}}%
\pgfpathlineto{\pgfqpoint{4.143138in}{2.937246in}}%
\pgfpathlineto{\pgfqpoint{4.167204in}{2.937246in}}%
\pgfpathlineto{\pgfqpoint{4.168740in}{2.943871in}}%
\pgfpathlineto{\pgfqpoint{4.192805in}{2.943871in}}%
\pgfpathlineto{\pgfqpoint{4.194341in}{2.951319in}}%
\pgfpathlineto{\pgfqpoint{4.218407in}{2.951319in}}%
\pgfpathlineto{\pgfqpoint{4.219943in}{2.935511in}}%
\pgfpathlineto{\pgfqpoint{4.244009in}{2.935511in}}%
\pgfpathlineto{\pgfqpoint{4.245545in}{2.948887in}}%
\pgfpathlineto{\pgfqpoint{4.269611in}{2.948887in}}%
\pgfpathlineto{\pgfqpoint{4.271147in}{2.954966in}}%
\pgfpathlineto{\pgfqpoint{4.295212in}{2.954966in}}%
\pgfpathlineto{\pgfqpoint{4.296748in}{2.932811in}}%
\pgfpathlineto{\pgfqpoint{4.320814in}{2.932811in}}%
\pgfpathlineto{\pgfqpoint{4.322350in}{2.950662in}}%
\pgfpathlineto{\pgfqpoint{4.372017in}{2.949966in}}%
\pgfpathlineto{\pgfqpoint{4.373554in}{2.951816in}}%
\pgfpathlineto{\pgfqpoint{4.397619in}{2.951816in}}%
\pgfpathlineto{\pgfqpoint{4.399155in}{2.937170in}}%
\pgfpathlineto{\pgfqpoint{4.423221in}{2.937170in}}%
\pgfpathlineto{\pgfqpoint{4.424757in}{2.941807in}}%
\pgfpathlineto{\pgfqpoint{4.448823in}{2.941807in}}%
\pgfpathlineto{\pgfqpoint{4.450359in}{2.957203in}}%
\pgfpathlineto{\pgfqpoint{4.474424in}{2.957203in}}%
\pgfpathlineto{\pgfqpoint{4.475961in}{2.952866in}}%
\pgfpathlineto{\pgfqpoint{4.500026in}{2.952866in}}%
\pgfpathlineto{\pgfqpoint{4.501562in}{2.950915in}}%
\pgfpathlineto{\pgfqpoint{4.551230in}{2.950876in}}%
\pgfpathlineto{\pgfqpoint{4.552766in}{2.944083in}}%
\pgfpathlineto{\pgfqpoint{4.576831in}{2.944083in}}%
\pgfpathlineto{\pgfqpoint{4.578367in}{2.952034in}}%
\pgfpathlineto{\pgfqpoint{4.628035in}{2.951806in}}%
\pgfpathlineto{\pgfqpoint{4.629571in}{2.945144in}}%
\pgfpathlineto{\pgfqpoint{4.653637in}{2.945144in}}%
\pgfpathlineto{\pgfqpoint{4.655173in}{2.963253in}}%
\pgfpathlineto{\pgfqpoint{4.679238in}{2.963253in}}%
\pgfpathlineto{\pgfqpoint{4.680774in}{2.949599in}}%
\pgfpathlineto{\pgfqpoint{4.704840in}{2.949599in}}%
\pgfpathlineto{\pgfqpoint{4.706376in}{2.939425in}}%
\pgfpathlineto{\pgfqpoint{4.730442in}{2.939425in}}%
\pgfpathlineto{\pgfqpoint{4.731978in}{2.955208in}}%
\pgfpathlineto{\pgfqpoint{4.756044in}{2.955208in}}%
\pgfpathlineto{\pgfqpoint{4.757580in}{2.943767in}}%
\pgfpathlineto{\pgfqpoint{4.781645in}{2.943767in}}%
\pgfpathlineto{\pgfqpoint{4.783181in}{2.957864in}}%
\pgfpathlineto{\pgfqpoint{4.832849in}{2.958003in}}%
\pgfpathlineto{\pgfqpoint{4.834385in}{2.943197in}}%
\pgfpathlineto{\pgfqpoint{4.858450in}{2.943197in}}%
\pgfpathlineto{\pgfqpoint{4.859987in}{2.957497in}}%
\pgfpathlineto{\pgfqpoint{4.884052in}{2.957497in}}%
\pgfpathlineto{\pgfqpoint{4.885588in}{2.951967in}}%
\pgfpathlineto{\pgfqpoint{4.909654in}{2.951967in}}%
\pgfpathlineto{\pgfqpoint{4.911190in}{2.954302in}}%
\pgfpathlineto{\pgfqpoint{4.935256in}{2.954302in}}%
\pgfpathlineto{\pgfqpoint{4.936792in}{2.959714in}}%
\pgfpathlineto{\pgfqpoint{4.960857in}{2.959714in}}%
\pgfpathlineto{\pgfqpoint{4.962394in}{2.942792in}}%
\pgfpathlineto{\pgfqpoint{5.012061in}{2.942546in}}%
\pgfpathlineto{\pgfqpoint{5.013597in}{2.945450in}}%
\pgfpathlineto{\pgfqpoint{5.037663in}{2.945450in}}%
\pgfpathlineto{\pgfqpoint{5.039199in}{2.949235in}}%
\pgfpathlineto{\pgfqpoint{5.063264in}{2.949235in}}%
\pgfpathlineto{\pgfqpoint{5.064800in}{2.956195in}}%
\pgfpathlineto{\pgfqpoint{5.088866in}{2.956195in}}%
\pgfpathlineto{\pgfqpoint{5.090402in}{2.958704in}}%
\pgfpathlineto{\pgfqpoint{5.114468in}{2.958704in}}%
\pgfpathlineto{\pgfqpoint{5.116004in}{2.957133in}}%
\pgfpathlineto{\pgfqpoint{5.140070in}{2.957133in}}%
\pgfpathlineto{\pgfqpoint{5.141606in}{2.959763in}}%
\pgfpathlineto{\pgfqpoint{5.165671in}{2.959763in}}%
\pgfpathlineto{\pgfqpoint{5.167207in}{2.952712in}}%
\pgfpathlineto{\pgfqpoint{5.191273in}{2.952712in}}%
\pgfpathlineto{\pgfqpoint{5.192809in}{2.962051in}}%
\pgfpathlineto{\pgfqpoint{5.216875in}{2.962051in}}%
\pgfpathlineto{\pgfqpoint{5.218411in}{2.948146in}}%
\pgfpathlineto{\pgfqpoint{5.242477in}{2.948146in}}%
\pgfpathlineto{\pgfqpoint{5.244013in}{2.968262in}}%
\pgfpathlineto{\pgfqpoint{5.268078in}{2.968262in}}%
\pgfpathlineto{\pgfqpoint{5.269614in}{2.949074in}}%
\pgfpathlineto{\pgfqpoint{5.293680in}{2.949074in}}%
\pgfpathlineto{\pgfqpoint{5.295216in}{2.940607in}}%
\pgfpathlineto{\pgfqpoint{5.319282in}{2.940607in}}%
\pgfpathlineto{\pgfqpoint{5.320818in}{2.943804in}}%
\pgfpathlineto{\pgfqpoint{5.344883in}{2.943804in}}%
\pgfpathlineto{\pgfqpoint{5.346420in}{2.957483in}}%
\pgfpathlineto{\pgfqpoint{5.370485in}{2.957483in}}%
\pgfpathlineto{\pgfqpoint{5.372021in}{2.952307in}}%
\pgfpathlineto{\pgfqpoint{5.396087in}{2.952307in}}%
\pgfpathlineto{\pgfqpoint{5.397623in}{2.959279in}}%
\pgfpathlineto{\pgfqpoint{5.421689in}{2.959279in}}%
\pgfpathlineto{\pgfqpoint{5.423225in}{2.956359in}}%
\pgfpathlineto{\pgfqpoint{5.447290in}{2.956359in}}%
\pgfpathlineto{\pgfqpoint{5.448827in}{2.952469in}}%
\pgfpathlineto{\pgfqpoint{5.472892in}{2.952469in}}%
\pgfpathlineto{\pgfqpoint{5.474428in}{2.948750in}}%
\pgfpathlineto{\pgfqpoint{5.498494in}{2.948750in}}%
\pgfpathlineto{\pgfqpoint{5.500030in}{2.947520in}}%
\pgfpathlineto{\pgfqpoint{5.524096in}{2.947520in}}%
\pgfpathlineto{\pgfqpoint{5.525632in}{2.958126in}}%
\pgfpathlineto{\pgfqpoint{5.549697in}{2.958126in}}%
\pgfpathlineto{\pgfqpoint{5.551233in}{2.966418in}}%
\pgfpathlineto{\pgfqpoint{5.575299in}{2.966418in}}%
\pgfpathlineto{\pgfqpoint{5.576835in}{2.959250in}}%
\pgfpathlineto{\pgfqpoint{5.600901in}{2.959250in}}%
\pgfpathlineto{\pgfqpoint{5.602437in}{2.949418in}}%
\pgfpathlineto{\pgfqpoint{5.626503in}{2.949418in}}%
\pgfpathlineto{\pgfqpoint{5.628039in}{2.961243in}}%
\pgfpathlineto{\pgfqpoint{5.677706in}{2.961237in}}%
\pgfpathlineto{\pgfqpoint{5.679242in}{2.951871in}}%
\pgfpathlineto{\pgfqpoint{5.703308in}{2.951871in}}%
\pgfpathlineto{\pgfqpoint{5.704844in}{2.960260in}}%
\pgfpathlineto{\pgfqpoint{5.728909in}{2.960260in}}%
\pgfpathlineto{\pgfqpoint{5.730446in}{2.958533in}}%
\pgfpathlineto{\pgfqpoint{5.754511in}{2.958533in}}%
\pgfpathlineto{\pgfqpoint{5.756047in}{2.950517in}}%
\pgfpathlineto{\pgfqpoint{5.780113in}{2.950517in}}%
\pgfpathlineto{\pgfqpoint{5.780113in}{2.950517in}}%
\pgfusepath{stroke}%
\end{pgfscope}%
\begin{pgfscope}%
\pgfsetrectcap%
\pgfsetmiterjoin%
\pgfsetlinewidth{0.803000pt}%
\definecolor{currentstroke}{rgb}{0.000000,0.000000,0.000000}%
\pgfsetstrokecolor{currentstroke}%
\pgfsetdash{}{0pt}%
\pgfpathmoveto{\pgfqpoint{0.660278in}{0.585278in}}%
\pgfpathlineto{\pgfqpoint{0.660278in}{3.850000in}}%
\pgfusepath{stroke}%
\end{pgfscope}%
\begin{pgfscope}%
\pgfsetrectcap%
\pgfsetmiterjoin%
\pgfsetlinewidth{0.803000pt}%
\definecolor{currentstroke}{rgb}{0.000000,0.000000,0.000000}%
\pgfsetstrokecolor{currentstroke}%
\pgfsetdash{}{0pt}%
\pgfpathmoveto{\pgfqpoint{5.780625in}{0.585278in}}%
\pgfpathlineto{\pgfqpoint{5.780625in}{3.850000in}}%
\pgfusepath{stroke}%
\end{pgfscope}%
\begin{pgfscope}%
\pgfsetrectcap%
\pgfsetmiterjoin%
\pgfsetlinewidth{0.803000pt}%
\definecolor{currentstroke}{rgb}{0.000000,0.000000,0.000000}%
\pgfsetstrokecolor{currentstroke}%
\pgfsetdash{}{0pt}%
\pgfpathmoveto{\pgfqpoint{0.660278in}{0.585278in}}%
\pgfpathlineto{\pgfqpoint{5.780625in}{0.585278in}}%
\pgfusepath{stroke}%
\end{pgfscope}%
\begin{pgfscope}%
\pgfsetrectcap%
\pgfsetmiterjoin%
\pgfsetlinewidth{0.803000pt}%
\definecolor{currentstroke}{rgb}{0.000000,0.000000,0.000000}%
\pgfsetstrokecolor{currentstroke}%
\pgfsetdash{}{0pt}%
\pgfpathmoveto{\pgfqpoint{0.660278in}{3.850000in}}%
\pgfpathlineto{\pgfqpoint{5.780625in}{3.850000in}}%
\pgfusepath{stroke}%
\end{pgfscope}%
\end{pgfpicture}%
\makeatother%
\endgroup%

  \caption{Control input force over time.}
  \label{fig:input}
\end{figure}

\subsection{Controller Comparison}

Figure~\ref{fig:comparison3way} presents a three-way comparison of the state norms for different control strategies: DD-SDLQR (data-driven sampled-data LQR), DD-LQR (data-driven continuous-time LQR), and SD-LQR (sampled-data LQR with known model).

\begin{figure}[htbp]
  \centering
  %% Creator: Matplotlib, PGF backend
%%
%% To include the figure in your LaTeX document, write
%%   \input{<filename>.pgf}
%%
%% Make sure the required packages are loaded in your preamble
%%   \usepackage{pgf}
%%
%% Also ensure that all the required font packages are loaded; for instance,
%% the lmodern package is sometimes necessary when using math font.
%%   \usepackage{lmodern}
%%
%% Figures using additional raster images can only be included by \input if
%% they are in the same directory as the main LaTeX file. For loading figures
%% from other directories you can use the `import` package
%%   \usepackage{import}
%%
%% and then include the figures with
%%   \import{<path to file>}{<filename>.pgf}
%%
%% Matplotlib used the following preamble
%%   
%%   \makeatletter\@ifpackageloaded{underscore}{}{\usepackage[strings]{underscore}}\makeatother
%%
\begingroup%
\makeatletter%
\begin{pgfpicture}%
\pgfpathrectangle{\pgfpointorigin}{\pgfqpoint{6.000000in}{4.000000in}}%
\pgfusepath{use as bounding box, clip}%
\begin{pgfscope}%
\pgfsetbuttcap%
\pgfsetmiterjoin%
\definecolor{currentfill}{rgb}{1.000000,1.000000,1.000000}%
\pgfsetfillcolor{currentfill}%
\pgfsetlinewidth{0.000000pt}%
\definecolor{currentstroke}{rgb}{1.000000,1.000000,1.000000}%
\pgfsetstrokecolor{currentstroke}%
\pgfsetdash{}{0pt}%
\pgfpathmoveto{\pgfqpoint{0.000000in}{0.000000in}}%
\pgfpathlineto{\pgfqpoint{6.000000in}{0.000000in}}%
\pgfpathlineto{\pgfqpoint{6.000000in}{4.000000in}}%
\pgfpathlineto{\pgfqpoint{0.000000in}{4.000000in}}%
\pgfpathlineto{\pgfqpoint{0.000000in}{0.000000in}}%
\pgfpathclose%
\pgfusepath{fill}%
\end{pgfscope}%
\begin{pgfscope}%
\pgfsetbuttcap%
\pgfsetmiterjoin%
\definecolor{currentfill}{rgb}{1.000000,1.000000,1.000000}%
\pgfsetfillcolor{currentfill}%
\pgfsetlinewidth{0.000000pt}%
\definecolor{currentstroke}{rgb}{0.000000,0.000000,0.000000}%
\pgfsetstrokecolor{currentstroke}%
\pgfsetstrokeopacity{0.000000}%
\pgfsetdash{}{0pt}%
\pgfpathmoveto{\pgfqpoint{0.512778in}{0.585278in}}%
\pgfpathlineto{\pgfqpoint{5.780625in}{0.585278in}}%
\pgfpathlineto{\pgfqpoint{5.780625in}{3.850000in}}%
\pgfpathlineto{\pgfqpoint{0.512778in}{3.850000in}}%
\pgfpathlineto{\pgfqpoint{0.512778in}{0.585278in}}%
\pgfpathclose%
\pgfusepath{fill}%
\end{pgfscope}%
\begin{pgfscope}%
\pgfpathrectangle{\pgfqpoint{0.512778in}{0.585278in}}{\pgfqpoint{5.267847in}{3.264722in}}%
\pgfusepath{clip}%
\pgfsetrectcap%
\pgfsetroundjoin%
\pgfsetlinewidth{0.803000pt}%
\definecolor{currentstroke}{rgb}{0.690196,0.690196,0.690196}%
\pgfsetstrokecolor{currentstroke}%
\pgfsetstrokeopacity{0.300000}%
\pgfsetdash{}{0pt}%
\pgfpathmoveto{\pgfqpoint{0.512778in}{0.585278in}}%
\pgfpathlineto{\pgfqpoint{0.512778in}{3.850000in}}%
\pgfusepath{stroke}%
\end{pgfscope}%
\begin{pgfscope}%
\pgfsetbuttcap%
\pgfsetroundjoin%
\definecolor{currentfill}{rgb}{0.000000,0.000000,0.000000}%
\pgfsetfillcolor{currentfill}%
\pgfsetlinewidth{0.803000pt}%
\definecolor{currentstroke}{rgb}{0.000000,0.000000,0.000000}%
\pgfsetstrokecolor{currentstroke}%
\pgfsetdash{}{0pt}%
\pgfsys@defobject{currentmarker}{\pgfqpoint{0.000000in}{-0.048611in}}{\pgfqpoint{0.000000in}{0.000000in}}{%
\pgfpathmoveto{\pgfqpoint{0.000000in}{0.000000in}}%
\pgfpathlineto{\pgfqpoint{0.000000in}{-0.048611in}}%
\pgfusepath{stroke,fill}%
}%
\begin{pgfscope}%
\pgfsys@transformshift{0.512778in}{0.585278in}%
\pgfsys@useobject{currentmarker}{}%
\end{pgfscope}%
\end{pgfscope}%
\begin{pgfscope}%
\definecolor{textcolor}{rgb}{0.000000,0.000000,0.000000}%
\pgfsetstrokecolor{textcolor}%
\pgfsetfillcolor{textcolor}%
\pgftext[x=0.512778in,y=0.488056in,,top]{\color{textcolor}\rmfamily\fontsize{10.000000}{12.000000}\selectfont 0}%
\end{pgfscope}%
\begin{pgfscope}%
\pgfpathrectangle{\pgfqpoint{0.512778in}{0.585278in}}{\pgfqpoint{5.267847in}{3.264722in}}%
\pgfusepath{clip}%
\pgfsetrectcap%
\pgfsetroundjoin%
\pgfsetlinewidth{0.803000pt}%
\definecolor{currentstroke}{rgb}{0.690196,0.690196,0.690196}%
\pgfsetstrokecolor{currentstroke}%
\pgfsetstrokeopacity{0.300000}%
\pgfsetdash{}{0pt}%
\pgfpathmoveto{\pgfqpoint{1.566347in}{0.585278in}}%
\pgfpathlineto{\pgfqpoint{1.566347in}{3.850000in}}%
\pgfusepath{stroke}%
\end{pgfscope}%
\begin{pgfscope}%
\pgfsetbuttcap%
\pgfsetroundjoin%
\definecolor{currentfill}{rgb}{0.000000,0.000000,0.000000}%
\pgfsetfillcolor{currentfill}%
\pgfsetlinewidth{0.803000pt}%
\definecolor{currentstroke}{rgb}{0.000000,0.000000,0.000000}%
\pgfsetstrokecolor{currentstroke}%
\pgfsetdash{}{0pt}%
\pgfsys@defobject{currentmarker}{\pgfqpoint{0.000000in}{-0.048611in}}{\pgfqpoint{0.000000in}{0.000000in}}{%
\pgfpathmoveto{\pgfqpoint{0.000000in}{0.000000in}}%
\pgfpathlineto{\pgfqpoint{0.000000in}{-0.048611in}}%
\pgfusepath{stroke,fill}%
}%
\begin{pgfscope}%
\pgfsys@transformshift{1.566347in}{0.585278in}%
\pgfsys@useobject{currentmarker}{}%
\end{pgfscope}%
\end{pgfscope}%
\begin{pgfscope}%
\definecolor{textcolor}{rgb}{0.000000,0.000000,0.000000}%
\pgfsetstrokecolor{textcolor}%
\pgfsetfillcolor{textcolor}%
\pgftext[x=1.566347in,y=0.488056in,,top]{\color{textcolor}\rmfamily\fontsize{10.000000}{12.000000}\selectfont 2}%
\end{pgfscope}%
\begin{pgfscope}%
\pgfpathrectangle{\pgfqpoint{0.512778in}{0.585278in}}{\pgfqpoint{5.267847in}{3.264722in}}%
\pgfusepath{clip}%
\pgfsetrectcap%
\pgfsetroundjoin%
\pgfsetlinewidth{0.803000pt}%
\definecolor{currentstroke}{rgb}{0.690196,0.690196,0.690196}%
\pgfsetstrokecolor{currentstroke}%
\pgfsetstrokeopacity{0.300000}%
\pgfsetdash{}{0pt}%
\pgfpathmoveto{\pgfqpoint{2.619917in}{0.585278in}}%
\pgfpathlineto{\pgfqpoint{2.619917in}{3.850000in}}%
\pgfusepath{stroke}%
\end{pgfscope}%
\begin{pgfscope}%
\pgfsetbuttcap%
\pgfsetroundjoin%
\definecolor{currentfill}{rgb}{0.000000,0.000000,0.000000}%
\pgfsetfillcolor{currentfill}%
\pgfsetlinewidth{0.803000pt}%
\definecolor{currentstroke}{rgb}{0.000000,0.000000,0.000000}%
\pgfsetstrokecolor{currentstroke}%
\pgfsetdash{}{0pt}%
\pgfsys@defobject{currentmarker}{\pgfqpoint{0.000000in}{-0.048611in}}{\pgfqpoint{0.000000in}{0.000000in}}{%
\pgfpathmoveto{\pgfqpoint{0.000000in}{0.000000in}}%
\pgfpathlineto{\pgfqpoint{0.000000in}{-0.048611in}}%
\pgfusepath{stroke,fill}%
}%
\begin{pgfscope}%
\pgfsys@transformshift{2.619917in}{0.585278in}%
\pgfsys@useobject{currentmarker}{}%
\end{pgfscope}%
\end{pgfscope}%
\begin{pgfscope}%
\definecolor{textcolor}{rgb}{0.000000,0.000000,0.000000}%
\pgfsetstrokecolor{textcolor}%
\pgfsetfillcolor{textcolor}%
\pgftext[x=2.619917in,y=0.488056in,,top]{\color{textcolor}\rmfamily\fontsize{10.000000}{12.000000}\selectfont 4}%
\end{pgfscope}%
\begin{pgfscope}%
\pgfpathrectangle{\pgfqpoint{0.512778in}{0.585278in}}{\pgfqpoint{5.267847in}{3.264722in}}%
\pgfusepath{clip}%
\pgfsetrectcap%
\pgfsetroundjoin%
\pgfsetlinewidth{0.803000pt}%
\definecolor{currentstroke}{rgb}{0.690196,0.690196,0.690196}%
\pgfsetstrokecolor{currentstroke}%
\pgfsetstrokeopacity{0.300000}%
\pgfsetdash{}{0pt}%
\pgfpathmoveto{\pgfqpoint{3.673486in}{0.585278in}}%
\pgfpathlineto{\pgfqpoint{3.673486in}{3.850000in}}%
\pgfusepath{stroke}%
\end{pgfscope}%
\begin{pgfscope}%
\pgfsetbuttcap%
\pgfsetroundjoin%
\definecolor{currentfill}{rgb}{0.000000,0.000000,0.000000}%
\pgfsetfillcolor{currentfill}%
\pgfsetlinewidth{0.803000pt}%
\definecolor{currentstroke}{rgb}{0.000000,0.000000,0.000000}%
\pgfsetstrokecolor{currentstroke}%
\pgfsetdash{}{0pt}%
\pgfsys@defobject{currentmarker}{\pgfqpoint{0.000000in}{-0.048611in}}{\pgfqpoint{0.000000in}{0.000000in}}{%
\pgfpathmoveto{\pgfqpoint{0.000000in}{0.000000in}}%
\pgfpathlineto{\pgfqpoint{0.000000in}{-0.048611in}}%
\pgfusepath{stroke,fill}%
}%
\begin{pgfscope}%
\pgfsys@transformshift{3.673486in}{0.585278in}%
\pgfsys@useobject{currentmarker}{}%
\end{pgfscope}%
\end{pgfscope}%
\begin{pgfscope}%
\definecolor{textcolor}{rgb}{0.000000,0.000000,0.000000}%
\pgfsetstrokecolor{textcolor}%
\pgfsetfillcolor{textcolor}%
\pgftext[x=3.673486in,y=0.488056in,,top]{\color{textcolor}\rmfamily\fontsize{10.000000}{12.000000}\selectfont 6}%
\end{pgfscope}%
\begin{pgfscope}%
\pgfpathrectangle{\pgfqpoint{0.512778in}{0.585278in}}{\pgfqpoint{5.267847in}{3.264722in}}%
\pgfusepath{clip}%
\pgfsetrectcap%
\pgfsetroundjoin%
\pgfsetlinewidth{0.803000pt}%
\definecolor{currentstroke}{rgb}{0.690196,0.690196,0.690196}%
\pgfsetstrokecolor{currentstroke}%
\pgfsetstrokeopacity{0.300000}%
\pgfsetdash{}{0pt}%
\pgfpathmoveto{\pgfqpoint{4.727056in}{0.585278in}}%
\pgfpathlineto{\pgfqpoint{4.727056in}{3.850000in}}%
\pgfusepath{stroke}%
\end{pgfscope}%
\begin{pgfscope}%
\pgfsetbuttcap%
\pgfsetroundjoin%
\definecolor{currentfill}{rgb}{0.000000,0.000000,0.000000}%
\pgfsetfillcolor{currentfill}%
\pgfsetlinewidth{0.803000pt}%
\definecolor{currentstroke}{rgb}{0.000000,0.000000,0.000000}%
\pgfsetstrokecolor{currentstroke}%
\pgfsetdash{}{0pt}%
\pgfsys@defobject{currentmarker}{\pgfqpoint{0.000000in}{-0.048611in}}{\pgfqpoint{0.000000in}{0.000000in}}{%
\pgfpathmoveto{\pgfqpoint{0.000000in}{0.000000in}}%
\pgfpathlineto{\pgfqpoint{0.000000in}{-0.048611in}}%
\pgfusepath{stroke,fill}%
}%
\begin{pgfscope}%
\pgfsys@transformshift{4.727056in}{0.585278in}%
\pgfsys@useobject{currentmarker}{}%
\end{pgfscope}%
\end{pgfscope}%
\begin{pgfscope}%
\definecolor{textcolor}{rgb}{0.000000,0.000000,0.000000}%
\pgfsetstrokecolor{textcolor}%
\pgfsetfillcolor{textcolor}%
\pgftext[x=4.727056in,y=0.488056in,,top]{\color{textcolor}\rmfamily\fontsize{10.000000}{12.000000}\selectfont 8}%
\end{pgfscope}%
\begin{pgfscope}%
\pgfpathrectangle{\pgfqpoint{0.512778in}{0.585278in}}{\pgfqpoint{5.267847in}{3.264722in}}%
\pgfusepath{clip}%
\pgfsetrectcap%
\pgfsetroundjoin%
\pgfsetlinewidth{0.803000pt}%
\definecolor{currentstroke}{rgb}{0.690196,0.690196,0.690196}%
\pgfsetstrokecolor{currentstroke}%
\pgfsetstrokeopacity{0.300000}%
\pgfsetdash{}{0pt}%
\pgfpathmoveto{\pgfqpoint{5.780625in}{0.585278in}}%
\pgfpathlineto{\pgfqpoint{5.780625in}{3.850000in}}%
\pgfusepath{stroke}%
\end{pgfscope}%
\begin{pgfscope}%
\pgfsetbuttcap%
\pgfsetroundjoin%
\definecolor{currentfill}{rgb}{0.000000,0.000000,0.000000}%
\pgfsetfillcolor{currentfill}%
\pgfsetlinewidth{0.803000pt}%
\definecolor{currentstroke}{rgb}{0.000000,0.000000,0.000000}%
\pgfsetstrokecolor{currentstroke}%
\pgfsetdash{}{0pt}%
\pgfsys@defobject{currentmarker}{\pgfqpoint{0.000000in}{-0.048611in}}{\pgfqpoint{0.000000in}{0.000000in}}{%
\pgfpathmoveto{\pgfqpoint{0.000000in}{0.000000in}}%
\pgfpathlineto{\pgfqpoint{0.000000in}{-0.048611in}}%
\pgfusepath{stroke,fill}%
}%
\begin{pgfscope}%
\pgfsys@transformshift{5.780625in}{0.585278in}%
\pgfsys@useobject{currentmarker}{}%
\end{pgfscope}%
\end{pgfscope}%
\begin{pgfscope}%
\definecolor{textcolor}{rgb}{0.000000,0.000000,0.000000}%
\pgfsetstrokecolor{textcolor}%
\pgfsetfillcolor{textcolor}%
\pgftext[x=5.780625in,y=0.488056in,,top]{\color{textcolor}\rmfamily\fontsize{10.000000}{12.000000}\selectfont 10}%
\end{pgfscope}%
\begin{pgfscope}%
\definecolor{textcolor}{rgb}{0.000000,0.000000,0.000000}%
\pgfsetstrokecolor{textcolor}%
\pgfsetfillcolor{textcolor}%
\pgftext[x=3.146701in,y=0.309043in,,top]{\color{textcolor}\rmfamily\fontsize{10.000000}{12.000000}\selectfont Time (s)}%
\end{pgfscope}%
\begin{pgfscope}%
\pgfpathrectangle{\pgfqpoint{0.512778in}{0.585278in}}{\pgfqpoint{5.267847in}{3.264722in}}%
\pgfusepath{clip}%
\pgfsetrectcap%
\pgfsetroundjoin%
\pgfsetlinewidth{0.803000pt}%
\definecolor{currentstroke}{rgb}{0.690196,0.690196,0.690196}%
\pgfsetstrokecolor{currentstroke}%
\pgfsetstrokeopacity{0.300000}%
\pgfsetdash{}{0pt}%
\pgfpathmoveto{\pgfqpoint{0.512778in}{0.730812in}}%
\pgfpathlineto{\pgfqpoint{5.780625in}{0.730812in}}%
\pgfusepath{stroke}%
\end{pgfscope}%
\begin{pgfscope}%
\pgfsetbuttcap%
\pgfsetroundjoin%
\definecolor{currentfill}{rgb}{0.000000,0.000000,0.000000}%
\pgfsetfillcolor{currentfill}%
\pgfsetlinewidth{0.803000pt}%
\definecolor{currentstroke}{rgb}{0.000000,0.000000,0.000000}%
\pgfsetstrokecolor{currentstroke}%
\pgfsetdash{}{0pt}%
\pgfsys@defobject{currentmarker}{\pgfqpoint{-0.048611in}{0.000000in}}{\pgfqpoint{-0.000000in}{0.000000in}}{%
\pgfpathmoveto{\pgfqpoint{-0.000000in}{0.000000in}}%
\pgfpathlineto{\pgfqpoint{-0.048611in}{0.000000in}}%
\pgfusepath{stroke,fill}%
}%
\begin{pgfscope}%
\pgfsys@transformshift{0.512778in}{0.730812in}%
\pgfsys@useobject{currentmarker}{}%
\end{pgfscope}%
\end{pgfscope}%
\begin{pgfscope}%
\definecolor{textcolor}{rgb}{0.000000,0.000000,0.000000}%
\pgfsetstrokecolor{textcolor}%
\pgfsetfillcolor{textcolor}%
\pgftext[x=0.346111in, y=0.682586in, left, base]{\color{textcolor}\rmfamily\fontsize{10.000000}{12.000000}\selectfont 0}%
\end{pgfscope}%
\begin{pgfscope}%
\pgfpathrectangle{\pgfqpoint{0.512778in}{0.585278in}}{\pgfqpoint{5.267847in}{3.264722in}}%
\pgfusepath{clip}%
\pgfsetrectcap%
\pgfsetroundjoin%
\pgfsetlinewidth{0.803000pt}%
\definecolor{currentstroke}{rgb}{0.690196,0.690196,0.690196}%
\pgfsetstrokecolor{currentstroke}%
\pgfsetstrokeopacity{0.300000}%
\pgfsetdash{}{0pt}%
\pgfpathmoveto{\pgfqpoint{0.512778in}{1.286632in}}%
\pgfpathlineto{\pgfqpoint{5.780625in}{1.286632in}}%
\pgfusepath{stroke}%
\end{pgfscope}%
\begin{pgfscope}%
\pgfsetbuttcap%
\pgfsetroundjoin%
\definecolor{currentfill}{rgb}{0.000000,0.000000,0.000000}%
\pgfsetfillcolor{currentfill}%
\pgfsetlinewidth{0.803000pt}%
\definecolor{currentstroke}{rgb}{0.000000,0.000000,0.000000}%
\pgfsetstrokecolor{currentstroke}%
\pgfsetdash{}{0pt}%
\pgfsys@defobject{currentmarker}{\pgfqpoint{-0.048611in}{0.000000in}}{\pgfqpoint{-0.000000in}{0.000000in}}{%
\pgfpathmoveto{\pgfqpoint{-0.000000in}{0.000000in}}%
\pgfpathlineto{\pgfqpoint{-0.048611in}{0.000000in}}%
\pgfusepath{stroke,fill}%
}%
\begin{pgfscope}%
\pgfsys@transformshift{0.512778in}{1.286632in}%
\pgfsys@useobject{currentmarker}{}%
\end{pgfscope}%
\end{pgfscope}%
\begin{pgfscope}%
\definecolor{textcolor}{rgb}{0.000000,0.000000,0.000000}%
\pgfsetstrokecolor{textcolor}%
\pgfsetfillcolor{textcolor}%
\pgftext[x=0.346111in, y=1.238407in, left, base]{\color{textcolor}\rmfamily\fontsize{10.000000}{12.000000}\selectfont 1}%
\end{pgfscope}%
\begin{pgfscope}%
\pgfpathrectangle{\pgfqpoint{0.512778in}{0.585278in}}{\pgfqpoint{5.267847in}{3.264722in}}%
\pgfusepath{clip}%
\pgfsetrectcap%
\pgfsetroundjoin%
\pgfsetlinewidth{0.803000pt}%
\definecolor{currentstroke}{rgb}{0.690196,0.690196,0.690196}%
\pgfsetstrokecolor{currentstroke}%
\pgfsetstrokeopacity{0.300000}%
\pgfsetdash{}{0pt}%
\pgfpathmoveto{\pgfqpoint{0.512778in}{1.842452in}}%
\pgfpathlineto{\pgfqpoint{5.780625in}{1.842452in}}%
\pgfusepath{stroke}%
\end{pgfscope}%
\begin{pgfscope}%
\pgfsetbuttcap%
\pgfsetroundjoin%
\definecolor{currentfill}{rgb}{0.000000,0.000000,0.000000}%
\pgfsetfillcolor{currentfill}%
\pgfsetlinewidth{0.803000pt}%
\definecolor{currentstroke}{rgb}{0.000000,0.000000,0.000000}%
\pgfsetstrokecolor{currentstroke}%
\pgfsetdash{}{0pt}%
\pgfsys@defobject{currentmarker}{\pgfqpoint{-0.048611in}{0.000000in}}{\pgfqpoint{-0.000000in}{0.000000in}}{%
\pgfpathmoveto{\pgfqpoint{-0.000000in}{0.000000in}}%
\pgfpathlineto{\pgfqpoint{-0.048611in}{0.000000in}}%
\pgfusepath{stroke,fill}%
}%
\begin{pgfscope}%
\pgfsys@transformshift{0.512778in}{1.842452in}%
\pgfsys@useobject{currentmarker}{}%
\end{pgfscope}%
\end{pgfscope}%
\begin{pgfscope}%
\definecolor{textcolor}{rgb}{0.000000,0.000000,0.000000}%
\pgfsetstrokecolor{textcolor}%
\pgfsetfillcolor{textcolor}%
\pgftext[x=0.346111in, y=1.794227in, left, base]{\color{textcolor}\rmfamily\fontsize{10.000000}{12.000000}\selectfont 2}%
\end{pgfscope}%
\begin{pgfscope}%
\pgfpathrectangle{\pgfqpoint{0.512778in}{0.585278in}}{\pgfqpoint{5.267847in}{3.264722in}}%
\pgfusepath{clip}%
\pgfsetrectcap%
\pgfsetroundjoin%
\pgfsetlinewidth{0.803000pt}%
\definecolor{currentstroke}{rgb}{0.690196,0.690196,0.690196}%
\pgfsetstrokecolor{currentstroke}%
\pgfsetstrokeopacity{0.300000}%
\pgfsetdash{}{0pt}%
\pgfpathmoveto{\pgfqpoint{0.512778in}{2.398273in}}%
\pgfpathlineto{\pgfqpoint{5.780625in}{2.398273in}}%
\pgfusepath{stroke}%
\end{pgfscope}%
\begin{pgfscope}%
\pgfsetbuttcap%
\pgfsetroundjoin%
\definecolor{currentfill}{rgb}{0.000000,0.000000,0.000000}%
\pgfsetfillcolor{currentfill}%
\pgfsetlinewidth{0.803000pt}%
\definecolor{currentstroke}{rgb}{0.000000,0.000000,0.000000}%
\pgfsetstrokecolor{currentstroke}%
\pgfsetdash{}{0pt}%
\pgfsys@defobject{currentmarker}{\pgfqpoint{-0.048611in}{0.000000in}}{\pgfqpoint{-0.000000in}{0.000000in}}{%
\pgfpathmoveto{\pgfqpoint{-0.000000in}{0.000000in}}%
\pgfpathlineto{\pgfqpoint{-0.048611in}{0.000000in}}%
\pgfusepath{stroke,fill}%
}%
\begin{pgfscope}%
\pgfsys@transformshift{0.512778in}{2.398273in}%
\pgfsys@useobject{currentmarker}{}%
\end{pgfscope}%
\end{pgfscope}%
\begin{pgfscope}%
\definecolor{textcolor}{rgb}{0.000000,0.000000,0.000000}%
\pgfsetstrokecolor{textcolor}%
\pgfsetfillcolor{textcolor}%
\pgftext[x=0.346111in, y=2.350048in, left, base]{\color{textcolor}\rmfamily\fontsize{10.000000}{12.000000}\selectfont 3}%
\end{pgfscope}%
\begin{pgfscope}%
\pgfpathrectangle{\pgfqpoint{0.512778in}{0.585278in}}{\pgfqpoint{5.267847in}{3.264722in}}%
\pgfusepath{clip}%
\pgfsetrectcap%
\pgfsetroundjoin%
\pgfsetlinewidth{0.803000pt}%
\definecolor{currentstroke}{rgb}{0.690196,0.690196,0.690196}%
\pgfsetstrokecolor{currentstroke}%
\pgfsetstrokeopacity{0.300000}%
\pgfsetdash{}{0pt}%
\pgfpathmoveto{\pgfqpoint{0.512778in}{2.954093in}}%
\pgfpathlineto{\pgfqpoint{5.780625in}{2.954093in}}%
\pgfusepath{stroke}%
\end{pgfscope}%
\begin{pgfscope}%
\pgfsetbuttcap%
\pgfsetroundjoin%
\definecolor{currentfill}{rgb}{0.000000,0.000000,0.000000}%
\pgfsetfillcolor{currentfill}%
\pgfsetlinewidth{0.803000pt}%
\definecolor{currentstroke}{rgb}{0.000000,0.000000,0.000000}%
\pgfsetstrokecolor{currentstroke}%
\pgfsetdash{}{0pt}%
\pgfsys@defobject{currentmarker}{\pgfqpoint{-0.048611in}{0.000000in}}{\pgfqpoint{-0.000000in}{0.000000in}}{%
\pgfpathmoveto{\pgfqpoint{-0.000000in}{0.000000in}}%
\pgfpathlineto{\pgfqpoint{-0.048611in}{0.000000in}}%
\pgfusepath{stroke,fill}%
}%
\begin{pgfscope}%
\pgfsys@transformshift{0.512778in}{2.954093in}%
\pgfsys@useobject{currentmarker}{}%
\end{pgfscope}%
\end{pgfscope}%
\begin{pgfscope}%
\definecolor{textcolor}{rgb}{0.000000,0.000000,0.000000}%
\pgfsetstrokecolor{textcolor}%
\pgfsetfillcolor{textcolor}%
\pgftext[x=0.346111in, y=2.905868in, left, base]{\color{textcolor}\rmfamily\fontsize{10.000000}{12.000000}\selectfont 4}%
\end{pgfscope}%
\begin{pgfscope}%
\pgfpathrectangle{\pgfqpoint{0.512778in}{0.585278in}}{\pgfqpoint{5.267847in}{3.264722in}}%
\pgfusepath{clip}%
\pgfsetrectcap%
\pgfsetroundjoin%
\pgfsetlinewidth{0.803000pt}%
\definecolor{currentstroke}{rgb}{0.690196,0.690196,0.690196}%
\pgfsetstrokecolor{currentstroke}%
\pgfsetstrokeopacity{0.300000}%
\pgfsetdash{}{0pt}%
\pgfpathmoveto{\pgfqpoint{0.512778in}{3.509914in}}%
\pgfpathlineto{\pgfqpoint{5.780625in}{3.509914in}}%
\pgfusepath{stroke}%
\end{pgfscope}%
\begin{pgfscope}%
\pgfsetbuttcap%
\pgfsetroundjoin%
\definecolor{currentfill}{rgb}{0.000000,0.000000,0.000000}%
\pgfsetfillcolor{currentfill}%
\pgfsetlinewidth{0.803000pt}%
\definecolor{currentstroke}{rgb}{0.000000,0.000000,0.000000}%
\pgfsetstrokecolor{currentstroke}%
\pgfsetdash{}{0pt}%
\pgfsys@defobject{currentmarker}{\pgfqpoint{-0.048611in}{0.000000in}}{\pgfqpoint{-0.000000in}{0.000000in}}{%
\pgfpathmoveto{\pgfqpoint{-0.000000in}{0.000000in}}%
\pgfpathlineto{\pgfqpoint{-0.048611in}{0.000000in}}%
\pgfusepath{stroke,fill}%
}%
\begin{pgfscope}%
\pgfsys@transformshift{0.512778in}{3.509914in}%
\pgfsys@useobject{currentmarker}{}%
\end{pgfscope}%
\end{pgfscope}%
\begin{pgfscope}%
\definecolor{textcolor}{rgb}{0.000000,0.000000,0.000000}%
\pgfsetstrokecolor{textcolor}%
\pgfsetfillcolor{textcolor}%
\pgftext[x=0.346111in, y=3.461688in, left, base]{\color{textcolor}\rmfamily\fontsize{10.000000}{12.000000}\selectfont 5}%
\end{pgfscope}%
\begin{pgfscope}%
\definecolor{textcolor}{rgb}{0.000000,0.000000,0.000000}%
\pgfsetstrokecolor{textcolor}%
\pgfsetfillcolor{textcolor}%
\pgftext[x=0.290555in,y=2.217639in,,bottom,rotate=90.000000]{\color{textcolor}\rmfamily\fontsize{10.000000}{12.000000}\selectfont State Norm}%
\end{pgfscope}%
\begin{pgfscope}%
\pgfpathrectangle{\pgfqpoint{0.512778in}{0.585278in}}{\pgfqpoint{5.267847in}{3.264722in}}%
\pgfusepath{clip}%
\pgfsetrectcap%
\pgfsetroundjoin%
\pgfsetlinewidth{1.204500pt}%
\definecolor{currentstroke}{rgb}{0.000000,0.000000,1.000000}%
\pgfsetstrokecolor{currentstroke}%
\pgfsetstrokeopacity{0.800000}%
\pgfsetdash{}{0pt}%
\pgfpathmoveto{\pgfqpoint{0.512778in}{1.352238in}}%
\pgfpathlineto{\pgfqpoint{0.518572in}{1.353449in}}%
\pgfpathlineto{\pgfqpoint{0.524367in}{1.357075in}}%
\pgfpathlineto{\pgfqpoint{0.530688in}{1.363761in}}%
\pgfpathlineto{\pgfqpoint{0.538063in}{1.375099in}}%
\pgfpathlineto{\pgfqpoint{0.546492in}{1.392412in}}%
\pgfpathlineto{\pgfqpoint{0.555974in}{1.417328in}}%
\pgfpathlineto{\pgfqpoint{0.565456in}{1.447786in}}%
\pgfpathlineto{\pgfqpoint{0.571778in}{1.526965in}}%
\pgfpathlineto{\pgfqpoint{0.580206in}{1.663714in}}%
\pgfpathlineto{\pgfqpoint{0.591795in}{1.889764in}}%
\pgfpathlineto{\pgfqpoint{0.592849in}{1.878158in}}%
\pgfpathlineto{\pgfqpoint{0.604438in}{1.763671in}}%
\pgfpathlineto{\pgfqpoint{0.613920in}{1.690775in}}%
\pgfpathlineto{\pgfqpoint{0.618662in}{1.664852in}}%
\pgfpathlineto{\pgfqpoint{0.624456in}{1.661007in}}%
\pgfpathlineto{\pgfqpoint{0.629724in}{1.660010in}}%
\pgfpathlineto{\pgfqpoint{0.634992in}{1.661360in}}%
\pgfpathlineto{\pgfqpoint{0.640260in}{1.665013in}}%
\pgfpathlineto{\pgfqpoint{0.644474in}{1.669555in}}%
\pgfpathlineto{\pgfqpoint{0.647635in}{1.644679in}}%
\pgfpathlineto{\pgfqpoint{0.649742in}{1.641092in}}%
\pgfpathlineto{\pgfqpoint{0.650795in}{1.643266in}}%
\pgfpathlineto{\pgfqpoint{0.652903in}{1.655442in}}%
\pgfpathlineto{\pgfqpoint{0.656063in}{1.692337in}}%
\pgfpathlineto{\pgfqpoint{0.660278in}{1.772182in}}%
\pgfpathlineto{\pgfqpoint{0.666599in}{1.942423in}}%
\pgfpathlineto{\pgfqpoint{0.675554in}{2.187762in}}%
\pgfpathlineto{\pgfqpoint{0.692938in}{2.620370in}}%
\pgfpathlineto{\pgfqpoint{0.699786in}{2.774499in}}%
\pgfpathlineto{\pgfqpoint{0.726652in}{3.200599in}}%
\pgfpathlineto{\pgfqpoint{0.752465in}{3.398032in}}%
\pgfpathlineto{\pgfqpoint{0.764581in}{3.424511in}}%
\pgfpathlineto{\pgfqpoint{0.777750in}{3.454215in}}%
\pgfpathlineto{\pgfqpoint{0.786179in}{3.455179in}}%
\pgfpathlineto{\pgfqpoint{0.794608in}{3.458402in}}%
\pgfpathlineto{\pgfqpoint{0.802509in}{3.463386in}}%
\pgfpathlineto{\pgfqpoint{0.803036in}{3.462317in}}%
\pgfpathlineto{\pgfqpoint{0.815679in}{3.439315in}}%
\pgfpathlineto{\pgfqpoint{0.827795in}{3.421796in}}%
\pgfpathlineto{\pgfqpoint{0.835697in}{3.413558in}}%
\pgfpathlineto{\pgfqpoint{0.846759in}{3.404982in}}%
\pgfpathlineto{\pgfqpoint{0.855188in}{3.400487in}}%
\pgfpathlineto{\pgfqpoint{0.871518in}{3.440592in}}%
\pgfpathlineto{\pgfqpoint{0.883107in}{3.473998in}}%
\pgfpathlineto{\pgfqpoint{0.904179in}{3.560948in}}%
\pgfpathlineto{\pgfqpoint{0.909973in}{3.581752in}}%
\pgfpathlineto{\pgfqpoint{0.926304in}{3.621073in}}%
\pgfpathlineto{\pgfqpoint{0.937366in}{3.647059in}}%
\pgfpathlineto{\pgfqpoint{0.951589in}{3.672081in}}%
\pgfpathlineto{\pgfqpoint{0.962125in}{3.690262in}}%
\pgfpathlineto{\pgfqpoint{0.972661in}{3.693525in}}%
\pgfpathlineto{\pgfqpoint{0.983723in}{3.699450in}}%
\pgfpathlineto{\pgfqpoint{0.987411in}{3.701045in}}%
\pgfpathlineto{\pgfqpoint{0.999527in}{3.689840in}}%
\pgfpathlineto{\pgfqpoint{1.011116in}{3.681987in}}%
\pgfpathlineto{\pgfqpoint{1.013750in}{3.679806in}}%
\pgfpathlineto{\pgfqpoint{1.029027in}{3.652117in}}%
\pgfpathlineto{\pgfqpoint{1.042196in}{3.628518in}}%
\pgfpathlineto{\pgfqpoint{1.060107in}{3.582958in}}%
\pgfpathlineto{\pgfqpoint{1.094348in}{3.499348in}}%
\pgfpathlineto{\pgfqpoint{1.113312in}{3.450891in}}%
\pgfpathlineto{\pgfqpoint{1.144393in}{3.379395in}}%
\pgfpathlineto{\pgfqpoint{1.154928in}{3.354788in}}%
\pgfpathlineto{\pgfqpoint{1.179687in}{3.301244in}}%
\pgfpathlineto{\pgfqpoint{1.199178in}{3.262709in}}%
\pgfpathlineto{\pgfqpoint{1.219196in}{3.229515in}}%
\pgfpathlineto{\pgfqpoint{1.249750in}{3.184211in}}%
\pgfpathlineto{\pgfqpoint{1.259759in}{3.172537in}}%
\pgfpathlineto{\pgfqpoint{1.279250in}{3.152377in}}%
\pgfpathlineto{\pgfqpoint{1.300321in}{3.134593in}}%
\pgfpathlineto{\pgfqpoint{1.312437in}{3.126563in}}%
\pgfpathlineto{\pgfqpoint{1.336669in}{3.113109in}}%
\pgfpathlineto{\pgfqpoint{1.357214in}{3.103868in}}%
\pgfpathlineto{\pgfqpoint{1.384080in}{3.098168in}}%
\pgfpathlineto{\pgfqpoint{1.423589in}{3.095310in}}%
\pgfpathlineto{\pgfqpoint{1.447821in}{3.095313in}}%
\pgfpathlineto{\pgfqpoint{1.554758in}{3.097712in}}%
\pgfpathlineto{\pgfqpoint{1.588999in}{3.095955in}}%
\pgfpathlineto{\pgfqpoint{1.609017in}{3.093325in}}%
\pgfpathlineto{\pgfqpoint{1.633249in}{3.088515in}}%
\pgfpathlineto{\pgfqpoint{1.659588in}{3.081172in}}%
\pgfpathlineto{\pgfqpoint{1.681713in}{3.073014in}}%
\pgfpathlineto{\pgfqpoint{1.721749in}{3.054691in}}%
\pgfpathlineto{\pgfqpoint{1.745981in}{3.041757in}}%
\pgfpathlineto{\pgfqpoint{1.779695in}{3.022050in}}%
\pgfpathlineto{\pgfqpoint{1.810249in}{2.999173in}}%
\pgfpathlineto{\pgfqpoint{1.882418in}{2.934465in}}%
\pgfpathlineto{\pgfqpoint{1.958275in}{2.853942in}}%
\pgfpathlineto{\pgfqpoint{1.979873in}{2.828455in}}%
\pgfpathlineto{\pgfqpoint{1.998311in}{2.804839in}}%
\pgfpathlineto{\pgfqpoint{2.107882in}{2.655536in}}%
\pgfpathlineto{\pgfqpoint{2.180578in}{2.545946in}}%
\pgfpathlineto{\pgfqpoint{2.214292in}{2.496004in}}%
\pgfpathlineto{\pgfqpoint{2.260649in}{2.428367in}}%
\pgfpathlineto{\pgfqpoint{2.392346in}{2.223748in}}%
\pgfpathlineto{\pgfqpoint{2.425533in}{2.172962in}}%
\pgfpathlineto{\pgfqpoint{2.452926in}{2.131476in}}%
\pgfpathlineto{\pgfqpoint{2.468203in}{2.107250in}}%
\pgfpathlineto{\pgfqpoint{2.494015in}{2.064123in}}%
\pgfpathlineto{\pgfqpoint{2.554595in}{1.975507in}}%
\pgfpathlineto{\pgfqpoint{2.635193in}{1.857802in}}%
\pgfpathlineto{\pgfqpoint{2.676809in}{1.803510in}}%
\pgfpathlineto{\pgfqpoint{2.711050in}{1.759511in}}%
\pgfpathlineto{\pgfqpoint{2.762675in}{1.689747in}}%
\pgfpathlineto{\pgfqpoint{2.791122in}{1.654811in}}%
\pgfpathlineto{\pgfqpoint{2.834845in}{1.602479in}}%
\pgfpathlineto{\pgfqpoint{2.869613in}{1.560425in}}%
\pgfpathlineto{\pgfqpoint{2.998148in}{1.416149in}}%
\pgfpathlineto{\pgfqpoint{3.056094in}{1.356275in}}%
\pgfpathlineto{\pgfqpoint{3.147755in}{1.272934in}}%
\pgfpathlineto{\pgfqpoint{3.181469in}{1.241432in}}%
\pgfpathlineto{\pgfqpoint{3.240469in}{1.193922in}}%
\pgfpathlineto{\pgfqpoint{3.267335in}{1.173696in}}%
\pgfpathlineto{\pgfqpoint{3.304737in}{1.148389in}}%
\pgfpathlineto{\pgfqpoint{3.338451in}{1.128226in}}%
\pgfpathlineto{\pgfqpoint{3.384281in}{1.094944in}}%
\pgfpathlineto{\pgfqpoint{3.424844in}{1.064668in}}%
\pgfpathlineto{\pgfqpoint{3.441701in}{1.052310in}}%
\pgfpathlineto{\pgfqpoint{3.491745in}{1.024911in}}%
\pgfpathlineto{\pgfqpoint{3.519665in}{1.013829in}}%
\pgfpathlineto{\pgfqpoint{3.554433in}{0.992538in}}%
\pgfpathlineto{\pgfqpoint{3.631870in}{0.950835in}}%
\pgfpathlineto{\pgfqpoint{3.669272in}{0.936756in}}%
\pgfpathlineto{\pgfqpoint{3.706674in}{0.920799in}}%
\pgfpathlineto{\pgfqpoint{3.730379in}{0.908992in}}%
\pgfpathlineto{\pgfqpoint{3.756718in}{0.900267in}}%
\pgfpathlineto{\pgfqpoint{3.782531in}{0.886943in}}%
\pgfpathlineto{\pgfqpoint{3.813611in}{0.878774in}}%
\pgfpathlineto{\pgfqpoint{3.834682in}{0.874188in}}%
\pgfpathlineto{\pgfqpoint{3.865763in}{0.859311in}}%
\pgfpathlineto{\pgfqpoint{3.894209in}{0.849486in}}%
\pgfpathlineto{\pgfqpoint{3.940039in}{0.835675in}}%
\pgfpathlineto{\pgfqpoint{3.967959in}{0.826723in}}%
\pgfpathlineto{\pgfqpoint{4.000619in}{0.820508in}}%
\pgfpathlineto{\pgfqpoint{4.022744in}{0.815840in}}%
\pgfpathlineto{\pgfqpoint{4.050137in}{0.812011in}}%
\pgfpathlineto{\pgfqpoint{4.072789in}{0.807378in}}%
\pgfpathlineto{\pgfqpoint{4.099655in}{0.805999in}}%
\pgfpathlineto{\pgfqpoint{4.131262in}{0.797882in}}%
\pgfpathlineto{\pgfqpoint{4.157074in}{0.794176in}}%
\pgfpathlineto{\pgfqpoint{4.178673in}{0.791567in}}%
\pgfpathlineto{\pgfqpoint{4.205012in}{0.784577in}}%
\pgfpathlineto{\pgfqpoint{4.237146in}{0.782337in}}%
\pgfpathlineto{\pgfqpoint{4.256110in}{0.781286in}}%
\pgfpathlineto{\pgfqpoint{4.281922in}{0.773449in}}%
\pgfpathlineto{\pgfqpoint{4.368842in}{0.766365in}}%
\pgfpathlineto{\pgfqpoint{4.395181in}{0.760896in}}%
\pgfpathlineto{\pgfqpoint{4.416779in}{0.758582in}}%
\pgfpathlineto{\pgfqpoint{4.476306in}{0.758072in}}%
\pgfpathlineto{\pgfqpoint{4.531092in}{0.754882in}}%
\pgfpathlineto{\pgfqpoint{4.553743in}{0.753415in}}%
\pgfpathlineto{\pgfqpoint{4.615377in}{0.751163in}}%
\pgfpathlineto{\pgfqpoint{4.629600in}{0.751312in}}%
\pgfpathlineto{\pgfqpoint{4.661734in}{0.752904in}}%
\pgfpathlineto{\pgfqpoint{4.685966in}{0.750970in}}%
\pgfpathlineto{\pgfqpoint{4.706511in}{0.749834in}}%
\pgfpathlineto{\pgfqpoint{4.993609in}{0.750976in}}%
\pgfpathlineto{\pgfqpoint{5.050501in}{0.753750in}}%
\pgfpathlineto{\pgfqpoint{5.205376in}{0.746144in}}%
\pgfpathlineto{\pgfqpoint{5.241724in}{0.745710in}}%
\pgfpathlineto{\pgfqpoint{5.285974in}{0.745829in}}%
\pgfpathlineto{\pgfqpoint{5.305465in}{0.747744in}}%
\pgfpathlineto{\pgfqpoint{5.336545in}{0.751445in}}%
\pgfpathlineto{\pgfqpoint{5.370260in}{0.750113in}}%
\pgfpathlineto{\pgfqpoint{5.394492in}{0.749596in}}%
\pgfpathlineto{\pgfqpoint{5.430313in}{0.747261in}}%
\pgfpathlineto{\pgfqpoint{5.455599in}{0.747356in}}%
\pgfpathlineto{\pgfqpoint{5.480358in}{0.749071in}}%
\pgfpathlineto{\pgfqpoint{5.529349in}{0.752939in}}%
\pgfpathlineto{\pgfqpoint{5.549893in}{0.750355in}}%
\pgfpathlineto{\pgfqpoint{5.575179in}{0.745301in}}%
\pgfpathlineto{\pgfqpoint{5.602045in}{0.744171in}}%
\pgfpathlineto{\pgfqpoint{5.628384in}{0.745666in}}%
\pgfpathlineto{\pgfqpoint{5.678956in}{0.740492in}}%
\pgfpathlineto{\pgfqpoint{5.710036in}{0.740992in}}%
\pgfpathlineto{\pgfqpoint{5.754286in}{0.737857in}}%
\pgfpathlineto{\pgfqpoint{5.780625in}{0.740096in}}%
\pgfpathlineto{\pgfqpoint{5.780625in}{0.740096in}}%
\pgfusepath{stroke}%
\end{pgfscope}%
\begin{pgfscope}%
\pgfpathrectangle{\pgfqpoint{0.512778in}{0.585278in}}{\pgfqpoint{5.267847in}{3.264722in}}%
\pgfusepath{clip}%
\pgfsetrectcap%
\pgfsetroundjoin%
\pgfsetlinewidth{1.204500pt}%
\definecolor{currentstroke}{rgb}{0.000000,0.501961,0.000000}%
\pgfsetstrokecolor{currentstroke}%
\pgfsetstrokeopacity{0.800000}%
\pgfsetdash{}{0pt}%
\pgfpathmoveto{\pgfqpoint{0.512778in}{1.352238in}}%
\pgfpathlineto{\pgfqpoint{0.518572in}{1.353426in}}%
\pgfpathlineto{\pgfqpoint{0.524367in}{1.356985in}}%
\pgfpathlineto{\pgfqpoint{0.530688in}{1.363547in}}%
\pgfpathlineto{\pgfqpoint{0.538063in}{1.374679in}}%
\pgfpathlineto{\pgfqpoint{0.539117in}{1.376570in}}%
\pgfpathlineto{\pgfqpoint{0.542278in}{1.367803in}}%
\pgfpathlineto{\pgfqpoint{0.544385in}{1.366131in}}%
\pgfpathlineto{\pgfqpoint{0.545965in}{1.367089in}}%
\pgfpathlineto{\pgfqpoint{0.548072in}{1.371299in}}%
\pgfpathlineto{\pgfqpoint{0.551233in}{1.383741in}}%
\pgfpathlineto{\pgfqpoint{0.555447in}{1.411081in}}%
\pgfpathlineto{\pgfqpoint{0.561242in}{1.466270in}}%
\pgfpathlineto{\pgfqpoint{0.567037in}{1.522634in}}%
\pgfpathlineto{\pgfqpoint{0.582313in}{1.581274in}}%
\pgfpathlineto{\pgfqpoint{0.591795in}{1.621536in}}%
\pgfpathlineto{\pgfqpoint{0.592849in}{1.620176in}}%
\pgfpathlineto{\pgfqpoint{0.604965in}{1.606570in}}%
\pgfpathlineto{\pgfqpoint{0.616028in}{1.597323in}}%
\pgfpathlineto{\pgfqpoint{0.618135in}{1.595897in}}%
\pgfpathlineto{\pgfqpoint{0.638679in}{1.643826in}}%
\pgfpathlineto{\pgfqpoint{0.647635in}{1.668470in}}%
\pgfpathlineto{\pgfqpoint{0.671340in}{1.749143in}}%
\pgfpathlineto{\pgfqpoint{0.697152in}{1.865340in}}%
\pgfpathlineto{\pgfqpoint{0.724018in}{2.048062in}}%
\pgfpathlineto{\pgfqpoint{0.754572in}{2.326735in}}%
\pgfpathlineto{\pgfqpoint{0.784072in}{2.593294in}}%
\pgfpathlineto{\pgfqpoint{0.806197in}{2.754484in}}%
\pgfpathlineto{\pgfqpoint{0.834116in}{2.884482in}}%
\pgfpathlineto{\pgfqpoint{0.853607in}{2.948829in}}%
\pgfpathlineto{\pgfqpoint{0.857295in}{2.958209in}}%
\pgfpathlineto{\pgfqpoint{0.872572in}{2.988674in}}%
\pgfpathlineto{\pgfqpoint{0.883634in}{3.010063in}}%
\pgfpathlineto{\pgfqpoint{0.895223in}{3.019495in}}%
\pgfpathlineto{\pgfqpoint{0.914188in}{3.035836in}}%
\pgfpathlineto{\pgfqpoint{0.925777in}{3.043466in}}%
\pgfpathlineto{\pgfqpoint{0.935786in}{3.050221in}}%
\pgfpathlineto{\pgfqpoint{0.946848in}{3.049167in}}%
\pgfpathlineto{\pgfqpoint{0.957911in}{3.050309in}}%
\pgfpathlineto{\pgfqpoint{0.961598in}{3.050258in}}%
\pgfpathlineto{\pgfqpoint{0.973714in}{3.044336in}}%
\pgfpathlineto{\pgfqpoint{0.985830in}{3.040925in}}%
\pgfpathlineto{\pgfqpoint{0.987411in}{3.039999in}}%
\pgfpathlineto{\pgfqpoint{1.002161in}{3.021120in}}%
\pgfpathlineto{\pgfqpoint{1.017964in}{3.001392in}}%
\pgfpathlineto{\pgfqpoint{1.034295in}{2.976793in}}%
\pgfpathlineto{\pgfqpoint{1.046411in}{2.958323in}}%
\pgfpathlineto{\pgfqpoint{1.063268in}{2.933161in}}%
\pgfpathlineto{\pgfqpoint{1.068536in}{2.924191in}}%
\pgfpathlineto{\pgfqpoint{1.087500in}{2.888267in}}%
\pgfpathlineto{\pgfqpoint{1.103303in}{2.858224in}}%
\pgfpathlineto{\pgfqpoint{1.148080in}{2.775601in}}%
\pgfpathlineto{\pgfqpoint{1.168098in}{2.745113in}}%
\pgfpathlineto{\pgfqpoint{1.179687in}{2.726727in}}%
\pgfpathlineto{\pgfqpoint{1.201285in}{2.694041in}}%
\pgfpathlineto{\pgfqpoint{1.222357in}{2.665362in}}%
\pgfpathlineto{\pgfqpoint{1.230259in}{2.657052in}}%
\pgfpathlineto{\pgfqpoint{1.253964in}{2.635468in}}%
\pgfpathlineto{\pgfqpoint{1.276089in}{2.617212in}}%
\pgfpathlineto{\pgfqpoint{1.289258in}{2.608558in}}%
\pgfpathlineto{\pgfqpoint{1.324026in}{2.588263in}}%
\pgfpathlineto{\pgfqpoint{1.344571in}{2.578632in}}%
\pgfpathlineto{\pgfqpoint{1.364589in}{2.571380in}}%
\pgfpathlineto{\pgfqpoint{1.386187in}{2.566037in}}%
\pgfpathlineto{\pgfqpoint{1.423062in}{2.562386in}}%
\pgfpathlineto{\pgfqpoint{1.444660in}{2.561994in}}%
\pgfpathlineto{\pgfqpoint{1.568981in}{2.564840in}}%
\pgfpathlineto{\pgfqpoint{1.622186in}{2.560886in}}%
\pgfpathlineto{\pgfqpoint{1.674865in}{2.552271in}}%
\pgfpathlineto{\pgfqpoint{1.706472in}{2.543642in}}%
\pgfpathlineto{\pgfqpoint{1.747034in}{2.528315in}}%
\pgfpathlineto{\pgfqpoint{1.767052in}{2.519081in}}%
\pgfpathlineto{\pgfqpoint{1.816570in}{2.494139in}}%
\pgfpathlineto{\pgfqpoint{1.837115in}{2.481999in}}%
\pgfpathlineto{\pgfqpoint{1.899275in}{2.437814in}}%
\pgfpathlineto{\pgfqpoint{1.934043in}{2.409003in}}%
\pgfpathlineto{\pgfqpoint{1.973552in}{2.374265in}}%
\pgfpathlineto{\pgfqpoint{2.034132in}{2.315627in}}%
\pgfpathlineto{\pgfqpoint{2.052043in}{2.299038in}}%
\pgfpathlineto{\pgfqpoint{2.074168in}{2.278176in}}%
\pgfpathlineto{\pgfqpoint{2.118944in}{2.228623in}}%
\pgfpathlineto{\pgfqpoint{2.251694in}{2.077210in}}%
\pgfpathlineto{\pgfqpoint{2.321230in}{1.990787in}}%
\pgfpathlineto{\pgfqpoint{2.344408in}{1.961004in}}%
\pgfpathlineto{\pgfqpoint{2.373908in}{1.923733in}}%
\pgfpathlineto{\pgfqpoint{2.391819in}{1.903629in}}%
\pgfpathlineto{\pgfqpoint{2.438703in}{1.853837in}}%
\pgfpathlineto{\pgfqpoint{2.481899in}{1.801990in}}%
\pgfpathlineto{\pgfqpoint{2.530363in}{1.747551in}}%
\pgfpathlineto{\pgfqpoint{2.548274in}{1.728111in}}%
\pgfpathlineto{\pgfqpoint{2.595158in}{1.669791in}}%
\pgfpathlineto{\pgfqpoint{2.647309in}{1.604208in}}%
\pgfpathlineto{\pgfqpoint{2.714211in}{1.531353in}}%
\pgfpathlineto{\pgfqpoint{2.741077in}{1.503273in}}%
\pgfpathlineto{\pgfqpoint{2.785854in}{1.460016in}}%
\pgfpathlineto{\pgfqpoint{2.818515in}{1.432782in}}%
\pgfpathlineto{\pgfqpoint{2.850648in}{1.402672in}}%
\pgfpathlineto{\pgfqpoint{2.869613in}{1.385970in}}%
\pgfpathlineto{\pgfqpoint{2.894898in}{1.363355in}}%
\pgfpathlineto{\pgfqpoint{2.947577in}{1.314218in}}%
\pgfpathlineto{\pgfqpoint{2.977077in}{1.290234in}}%
\pgfpathlineto{\pgfqpoint{3.003416in}{1.266204in}}%
\pgfpathlineto{\pgfqpoint{3.026594in}{1.245666in}}%
\pgfpathlineto{\pgfqpoint{3.101398in}{1.185500in}}%
\pgfpathlineto{\pgfqpoint{3.143541in}{1.161443in}}%
\pgfpathlineto{\pgfqpoint{3.150389in}{1.156611in}}%
\pgfpathlineto{\pgfqpoint{3.178308in}{1.134466in}}%
\pgfpathlineto{\pgfqpoint{3.215183in}{1.110382in}}%
\pgfpathlineto{\pgfqpoint{3.249951in}{1.087358in}}%
\pgfpathlineto{\pgfqpoint{3.302103in}{1.055276in}}%
\pgfpathlineto{\pgfqpoint{3.317906in}{1.047053in}}%
\pgfpathlineto{\pgfqpoint{3.351094in}{1.029186in}}%
\pgfpathlineto{\pgfqpoint{3.368478in}{1.020695in}}%
\pgfpathlineto{\pgfqpoint{3.423263in}{0.995027in}}%
\pgfpathlineto{\pgfqpoint{3.497013in}{0.960752in}}%
\pgfpathlineto{\pgfqpoint{3.542317in}{0.944870in}}%
\pgfpathlineto{\pgfqpoint{3.575504in}{0.929411in}}%
\pgfpathlineto{\pgfqpoint{3.613959in}{0.916703in}}%
\pgfpathlineto{\pgfqpoint{3.625549in}{0.912158in}}%
\pgfpathlineto{\pgfqpoint{3.651361in}{0.899862in}}%
\pgfpathlineto{\pgfqpoint{3.683495in}{0.890261in}}%
\pgfpathlineto{\pgfqpoint{3.708254in}{0.882035in}}%
\pgfpathlineto{\pgfqpoint{3.735120in}{0.874339in}}%
\pgfpathlineto{\pgfqpoint{3.809923in}{0.847498in}}%
\pgfpathlineto{\pgfqpoint{3.869977in}{0.837211in}}%
\pgfpathlineto{\pgfqpoint{3.891048in}{0.832729in}}%
\pgfpathlineto{\pgfqpoint{3.915280in}{0.829138in}}%
\pgfpathlineto{\pgfqpoint{3.941620in}{0.821155in}}%
\pgfpathlineto{\pgfqpoint{3.973753in}{0.815599in}}%
\pgfpathlineto{\pgfqpoint{4.002200in}{0.809677in}}%
\pgfpathlineto{\pgfqpoint{4.030646in}{0.806215in}}%
\pgfpathlineto{\pgfqpoint{4.052771in}{0.803273in}}%
\pgfpathlineto{\pgfqpoint{4.084905in}{0.798394in}}%
\pgfpathlineto{\pgfqpoint{4.102816in}{0.797347in}}%
\pgfpathlineto{\pgfqpoint{4.126521in}{0.796782in}}%
\pgfpathlineto{\pgfqpoint{4.153914in}{0.792602in}}%
\pgfpathlineto{\pgfqpoint{4.178673in}{0.790938in}}%
\pgfpathlineto{\pgfqpoint{4.204485in}{0.785131in}}%
\pgfpathlineto{\pgfqpoint{4.240306in}{0.784779in}}%
\pgfpathlineto{\pgfqpoint{4.255056in}{0.784367in}}%
\pgfpathlineto{\pgfqpoint{4.280342in}{0.774228in}}%
\pgfpathlineto{\pgfqpoint{4.312476in}{0.771278in}}%
\pgfpathlineto{\pgfqpoint{4.344610in}{0.770818in}}%
\pgfpathlineto{\pgfqpoint{4.376744in}{0.769573in}}%
\pgfpathlineto{\pgfqpoint{4.395181in}{0.768269in}}%
\pgfpathlineto{\pgfqpoint{4.423101in}{0.766189in}}%
\pgfpathlineto{\pgfqpoint{4.481574in}{0.765640in}}%
\pgfpathlineto{\pgfqpoint{4.499485in}{0.765782in}}%
\pgfpathlineto{\pgfqpoint{4.644877in}{0.760070in}}%
\pgfpathlineto{\pgfqpoint{4.871395in}{0.762868in}}%
\pgfpathlineto{\pgfqpoint{4.907743in}{0.765005in}}%
\pgfpathlineto{\pgfqpoint{4.917225in}{0.764742in}}%
\pgfpathlineto{\pgfqpoint{5.003091in}{0.757346in}}%
\pgfpathlineto{\pgfqpoint{5.086323in}{0.757795in}}%
\pgfpathlineto{\pgfqpoint{5.103707in}{0.758447in}}%
\pgfpathlineto{\pgfqpoint{5.156385in}{0.756426in}}%
\pgfpathlineto{\pgfqpoint{5.184305in}{0.756543in}}%
\pgfpathlineto{\pgfqpoint{5.211171in}{0.755246in}}%
\pgfpathlineto{\pgfqpoint{5.362358in}{0.754239in}}%
\pgfpathlineto{\pgfqpoint{5.404501in}{0.755873in}}%
\pgfpathlineto{\pgfqpoint{5.419777in}{0.755778in}}%
\pgfpathlineto{\pgfqpoint{5.446117in}{0.754944in}}%
\pgfpathlineto{\pgfqpoint{5.478251in}{0.755584in}}%
\pgfpathlineto{\pgfqpoint{5.515126in}{0.755373in}}%
\pgfpathlineto{\pgfqpoint{5.528822in}{0.754443in}}%
\pgfpathlineto{\pgfqpoint{5.584661in}{0.751120in}}%
\pgfpathlineto{\pgfqpoint{5.607313in}{0.750769in}}%
\pgfpathlineto{\pgfqpoint{5.697920in}{0.752799in}}%
\pgfpathlineto{\pgfqpoint{5.713723in}{0.751407in}}%
\pgfpathlineto{\pgfqpoint{5.759027in}{0.747714in}}%
\pgfpathlineto{\pgfqpoint{5.780625in}{0.748949in}}%
\pgfpathlineto{\pgfqpoint{5.780625in}{0.748949in}}%
\pgfusepath{stroke}%
\end{pgfscope}%
\begin{pgfscope}%
\pgfpathrectangle{\pgfqpoint{0.512778in}{0.585278in}}{\pgfqpoint{5.267847in}{3.264722in}}%
\pgfusepath{clip}%
\pgfsetbuttcap%
\pgfsetroundjoin%
\pgfsetlinewidth{1.204500pt}%
\definecolor{currentstroke}{rgb}{1.000000,0.000000,0.000000}%
\pgfsetstrokecolor{currentstroke}%
\pgfsetstrokeopacity{0.800000}%
\pgfsetdash{{4.440000pt}{1.920000pt}}{0.000000pt}%
\pgfpathmoveto{\pgfqpoint{0.512778in}{1.352238in}}%
\pgfpathlineto{\pgfqpoint{0.514885in}{1.353313in}}%
\pgfpathlineto{\pgfqpoint{0.517519in}{1.357665in}}%
\pgfpathlineto{\pgfqpoint{0.521206in}{1.369239in}}%
\pgfpathlineto{\pgfqpoint{0.525947in}{1.393004in}}%
\pgfpathlineto{\pgfqpoint{0.531742in}{1.434241in}}%
\pgfpathlineto{\pgfqpoint{0.556501in}{1.647383in}}%
\pgfpathlineto{\pgfqpoint{0.570724in}{1.768490in}}%
\pgfpathlineto{\pgfqpoint{0.590215in}{1.908377in}}%
\pgfpathlineto{\pgfqpoint{0.594429in}{1.932497in}}%
\pgfpathlineto{\pgfqpoint{0.614447in}{2.030876in}}%
\pgfpathlineto{\pgfqpoint{0.620242in}{2.055913in}}%
\pgfpathlineto{\pgfqpoint{0.637099in}{2.105716in}}%
\pgfpathlineto{\pgfqpoint{0.646581in}{2.132122in}}%
\pgfpathlineto{\pgfqpoint{0.659224in}{2.149511in}}%
\pgfpathlineto{\pgfqpoint{0.672920in}{2.168567in}}%
\pgfpathlineto{\pgfqpoint{0.683456in}{2.170390in}}%
\pgfpathlineto{\pgfqpoint{0.693992in}{2.174484in}}%
\pgfpathlineto{\pgfqpoint{0.697679in}{2.175696in}}%
\pgfpathlineto{\pgfqpoint{0.709795in}{2.166929in}}%
\pgfpathlineto{\pgfqpoint{0.721385in}{2.161170in}}%
\pgfpathlineto{\pgfqpoint{0.724545in}{2.158840in}}%
\pgfpathlineto{\pgfqpoint{0.739295in}{2.139273in}}%
\pgfpathlineto{\pgfqpoint{0.753518in}{2.120700in}}%
\pgfpathlineto{\pgfqpoint{0.770376in}{2.091885in}}%
\pgfpathlineto{\pgfqpoint{0.779858in}{2.075065in}}%
\pgfpathlineto{\pgfqpoint{0.798822in}{2.037382in}}%
\pgfpathlineto{\pgfqpoint{0.807777in}{2.018534in}}%
\pgfpathlineto{\pgfqpoint{0.828322in}{1.974825in}}%
\pgfpathlineto{\pgfqpoint{0.835697in}{1.957684in}}%
\pgfpathlineto{\pgfqpoint{0.873625in}{1.873035in}}%
\pgfpathlineto{\pgfqpoint{0.931572in}{1.749743in}}%
\pgfpathlineto{\pgfqpoint{0.968447in}{1.679367in}}%
\pgfpathlineto{\pgfqpoint{0.993732in}{1.636086in}}%
\pgfpathlineto{\pgfqpoint{1.018491in}{1.597940in}}%
\pgfpathlineto{\pgfqpoint{1.043250in}{1.564134in}}%
\pgfpathlineto{\pgfqpoint{1.068009in}{1.534690in}}%
\pgfpathlineto{\pgfqpoint{1.093295in}{1.509038in}}%
\pgfpathlineto{\pgfqpoint{1.119107in}{1.487242in}}%
\pgfpathlineto{\pgfqpoint{1.145446in}{1.469238in}}%
\pgfpathlineto{\pgfqpoint{1.172839in}{1.454604in}}%
\pgfpathlineto{\pgfqpoint{1.200759in}{1.443402in}}%
\pgfpathlineto{\pgfqpoint{1.230259in}{1.434985in}}%
\pgfpathlineto{\pgfqpoint{1.260812in}{1.429211in}}%
\pgfpathlineto{\pgfqpoint{1.294000in}{1.425490in}}%
\pgfpathlineto{\pgfqpoint{1.331401in}{1.423488in}}%
\pgfpathlineto{\pgfqpoint{1.400410in}{1.422856in}}%
\pgfpathlineto{\pgfqpoint{1.489437in}{1.421820in}}%
\pgfpathlineto{\pgfqpoint{1.550544in}{1.418318in}}%
\pgfpathlineto{\pgfqpoint{1.602169in}{1.412901in}}%
\pgfpathlineto{\pgfqpoint{1.653793in}{1.405019in}}%
\pgfpathlineto{\pgfqpoint{1.707526in}{1.394196in}}%
\pgfpathlineto{\pgfqpoint{1.762311in}{1.380558in}}%
\pgfpathlineto{\pgfqpoint{1.818677in}{1.364039in}}%
\pgfpathlineto{\pgfqpoint{1.875570in}{1.345127in}}%
\pgfpathlineto{\pgfqpoint{1.934043in}{1.323716in}}%
\pgfpathlineto{\pgfqpoint{1.991462in}{1.301039in}}%
\pgfpathlineto{\pgfqpoint{2.096819in}{1.256489in}}%
\pgfpathlineto{\pgfqpoint{2.294364in}{1.168296in}}%
\pgfpathlineto{\pgfqpoint{2.454506in}{1.097885in}}%
\pgfpathlineto{\pgfqpoint{2.565131in}{1.051866in}}%
\pgfpathlineto{\pgfqpoint{2.664167in}{1.013195in}}%
\pgfpathlineto{\pgfqpoint{2.755300in}{0.980028in}}%
\pgfpathlineto{\pgfqpoint{2.844327in}{0.950020in}}%
\pgfpathlineto{\pgfqpoint{2.931246in}{0.923061in}}%
\pgfpathlineto{\pgfqpoint{3.016586in}{0.898846in}}%
\pgfpathlineto{\pgfqpoint{3.107719in}{0.875422in}}%
\pgfpathlineto{\pgfqpoint{3.196746in}{0.854866in}}%
\pgfpathlineto{\pgfqpoint{3.283665in}{0.836932in}}%
\pgfpathlineto{\pgfqpoint{3.381121in}{0.819171in}}%
\pgfpathlineto{\pgfqpoint{3.473308in}{0.804509in}}%
\pgfpathlineto{\pgfqpoint{3.576031in}{0.790403in}}%
\pgfpathlineto{\pgfqpoint{3.685602in}{0.777714in}}%
\pgfpathlineto{\pgfqpoint{3.799914in}{0.766779in}}%
\pgfpathlineto{\pgfqpoint{3.916334in}{0.757776in}}%
\pgfpathlineto{\pgfqpoint{4.051191in}{0.749678in}}%
\pgfpathlineto{\pgfqpoint{4.195003in}{0.743407in}}%
\pgfpathlineto{\pgfqpoint{4.340396in}{0.739220in}}%
\pgfpathlineto{\pgfqpoint{4.506860in}{0.736707in}}%
\pgfpathlineto{\pgfqpoint{4.725475in}{0.735857in}}%
\pgfpathlineto{\pgfqpoint{5.708982in}{0.733903in}}%
\pgfpathlineto{\pgfqpoint{5.780625in}{0.733674in}}%
\pgfpathlineto{\pgfqpoint{5.780625in}{0.733674in}}%
\pgfusepath{stroke}%
\end{pgfscope}%
\begin{pgfscope}%
\pgfsetrectcap%
\pgfsetmiterjoin%
\pgfsetlinewidth{0.803000pt}%
\definecolor{currentstroke}{rgb}{0.000000,0.000000,0.000000}%
\pgfsetstrokecolor{currentstroke}%
\pgfsetdash{}{0pt}%
\pgfpathmoveto{\pgfqpoint{0.512778in}{0.585278in}}%
\pgfpathlineto{\pgfqpoint{0.512778in}{3.850000in}}%
\pgfusepath{stroke}%
\end{pgfscope}%
\begin{pgfscope}%
\pgfsetrectcap%
\pgfsetmiterjoin%
\pgfsetlinewidth{0.803000pt}%
\definecolor{currentstroke}{rgb}{0.000000,0.000000,0.000000}%
\pgfsetstrokecolor{currentstroke}%
\pgfsetdash{}{0pt}%
\pgfpathmoveto{\pgfqpoint{5.780625in}{0.585278in}}%
\pgfpathlineto{\pgfqpoint{5.780625in}{3.850000in}}%
\pgfusepath{stroke}%
\end{pgfscope}%
\begin{pgfscope}%
\pgfsetrectcap%
\pgfsetmiterjoin%
\pgfsetlinewidth{0.803000pt}%
\definecolor{currentstroke}{rgb}{0.000000,0.000000,0.000000}%
\pgfsetstrokecolor{currentstroke}%
\pgfsetdash{}{0pt}%
\pgfpathmoveto{\pgfqpoint{0.512778in}{0.585278in}}%
\pgfpathlineto{\pgfqpoint{5.780625in}{0.585278in}}%
\pgfusepath{stroke}%
\end{pgfscope}%
\begin{pgfscope}%
\pgfsetrectcap%
\pgfsetmiterjoin%
\pgfsetlinewidth{0.803000pt}%
\definecolor{currentstroke}{rgb}{0.000000,0.000000,0.000000}%
\pgfsetstrokecolor{currentstroke}%
\pgfsetdash{}{0pt}%
\pgfpathmoveto{\pgfqpoint{0.512778in}{3.850000in}}%
\pgfpathlineto{\pgfqpoint{5.780625in}{3.850000in}}%
\pgfusepath{stroke}%
\end{pgfscope}%
\begin{pgfscope}%
\pgfsetbuttcap%
\pgfsetmiterjoin%
\definecolor{currentfill}{rgb}{1.000000,1.000000,1.000000}%
\pgfsetfillcolor{currentfill}%
\pgfsetfillopacity{0.800000}%
\pgfsetlinewidth{1.003750pt}%
\definecolor{currentstroke}{rgb}{0.800000,0.800000,0.800000}%
\pgfsetstrokecolor{currentstroke}%
\pgfsetstrokeopacity{0.800000}%
\pgfsetdash{}{0pt}%
\pgfpathmoveto{\pgfqpoint{3.705390in}{3.143210in}}%
\pgfpathlineto{\pgfqpoint{5.683403in}{3.143210in}}%
\pgfpathquadraticcurveto{\pgfqpoint{5.711181in}{3.143210in}}{\pgfqpoint{5.711181in}{3.170988in}}%
\pgfpathlineto{\pgfqpoint{5.711181in}{3.752778in}}%
\pgfpathquadraticcurveto{\pgfqpoint{5.711181in}{3.780556in}}{\pgfqpoint{5.683403in}{3.780556in}}%
\pgfpathlineto{\pgfqpoint{3.705390in}{3.780556in}}%
\pgfpathquadraticcurveto{\pgfqpoint{3.677613in}{3.780556in}}{\pgfqpoint{3.677613in}{3.752778in}}%
\pgfpathlineto{\pgfqpoint{3.677613in}{3.170988in}}%
\pgfpathquadraticcurveto{\pgfqpoint{3.677613in}{3.143210in}}{\pgfqpoint{3.705390in}{3.143210in}}%
\pgfpathlineto{\pgfqpoint{3.705390in}{3.143210in}}%
\pgfpathclose%
\pgfusepath{stroke,fill}%
\end{pgfscope}%
\begin{pgfscope}%
\pgfsetrectcap%
\pgfsetroundjoin%
\pgfsetlinewidth{1.204500pt}%
\definecolor{currentstroke}{rgb}{0.000000,0.000000,1.000000}%
\pgfsetstrokecolor{currentstroke}%
\pgfsetstrokeopacity{0.800000}%
\pgfsetdash{}{0pt}%
\pgfpathmoveto{\pgfqpoint{3.733168in}{3.676389in}}%
\pgfpathlineto{\pgfqpoint{3.872057in}{3.676389in}}%
\pgfpathlineto{\pgfqpoint{4.010946in}{3.676389in}}%
\pgfusepath{stroke}%
\end{pgfscope}%
\begin{pgfscope}%
\definecolor{textcolor}{rgb}{0.000000,0.000000,0.000000}%
\pgfsetstrokecolor{textcolor}%
\pgfsetfillcolor{textcolor}%
\pgftext[x=4.122057in,y=3.627778in,left,base]{\color{textcolor}\rmfamily\fontsize{10.000000}{12.000000}\selectfont DD-SDLQR}%
\end{pgfscope}%
\begin{pgfscope}%
\pgfsetrectcap%
\pgfsetroundjoin%
\pgfsetlinewidth{1.204500pt}%
\definecolor{currentstroke}{rgb}{0.000000,0.501961,0.000000}%
\pgfsetstrokecolor{currentstroke}%
\pgfsetstrokeopacity{0.800000}%
\pgfsetdash{}{0pt}%
\pgfpathmoveto{\pgfqpoint{3.733168in}{3.482716in}}%
\pgfpathlineto{\pgfqpoint{3.872057in}{3.482716in}}%
\pgfpathlineto{\pgfqpoint{4.010946in}{3.482716in}}%
\pgfusepath{stroke}%
\end{pgfscope}%
\begin{pgfscope}%
\definecolor{textcolor}{rgb}{0.000000,0.000000,0.000000}%
\pgfsetstrokecolor{textcolor}%
\pgfsetfillcolor{textcolor}%
\pgftext[x=4.122057in,y=3.434105in,left,base]{\color{textcolor}\rmfamily\fontsize{10.000000}{12.000000}\selectfont DD-LQR}%
\end{pgfscope}%
\begin{pgfscope}%
\pgfsetbuttcap%
\pgfsetroundjoin%
\pgfsetlinewidth{1.204500pt}%
\definecolor{currentstroke}{rgb}{1.000000,0.000000,0.000000}%
\pgfsetstrokecolor{currentstroke}%
\pgfsetstrokeopacity{0.800000}%
\pgfsetdash{{4.440000pt}{1.920000pt}}{0.000000pt}%
\pgfpathmoveto{\pgfqpoint{3.733168in}{3.282099in}}%
\pgfpathlineto{\pgfqpoint{3.872057in}{3.282099in}}%
\pgfpathlineto{\pgfqpoint{4.010946in}{3.282099in}}%
\pgfusepath{stroke}%
\end{pgfscope}%
\begin{pgfscope}%
\definecolor{textcolor}{rgb}{0.000000,0.000000,0.000000}%
\pgfsetstrokecolor{textcolor}%
\pgfsetfillcolor{textcolor}%
\pgftext[x=4.122057in,y=3.233488in,left,base]{\color{textcolor}\rmfamily\fontsize{10.000000}{12.000000}\selectfont SD-LQR (Known Model)}%
\end{pgfscope}%
\end{pgfpicture}%
\makeatother%
\endgroup%

  \caption{State norm comparison among DD-SDLQR, DD-LQR, and SD-LQR controllers. The results demonstrate that DD-SDLQR achieves performance close to the ideal case while using only data-driven system identification.}
  \label{fig:comparison3way}
\end{figure}

\subsection{System Estimation Error}

The convergence of the system estimation error is illustrated in Figure~\ref{fig:error}.

\begin{figure}[htbp]
  \centering
  %% Creator: Matplotlib, PGF backend
%%
%% To include the figure in your LaTeX document, write
%%   \input{<filename>.pgf}
%%
%% Make sure the required packages are loaded in your preamble
%%   \usepackage{pgf}
%%
%% Also ensure that all the required font packages are loaded; for instance,
%% the lmodern package is sometimes necessary when using math font.
%%   \usepackage{lmodern}
%%
%% Figures using additional raster images can only be included by \input if
%% they are in the same directory as the main LaTeX file. For loading figures
%% from other directories you can use the `import` package
%%   \usepackage{import}
%%
%% and then include the figures with
%%   \import{<path to file>}{<filename>.pgf}
%%
%% Matplotlib used the following preamble
%%   
%%   \makeatletter\@ifpackageloaded{underscore}{}{\usepackage[strings]{underscore}}\makeatother
%%
\begingroup%
\makeatletter%
\begin{pgfpicture}%
\pgfpathrectangle{\pgfpointorigin}{\pgfqpoint{6.000000in}{4.000000in}}%
\pgfusepath{use as bounding box, clip}%
\begin{pgfscope}%
\pgfsetbuttcap%
\pgfsetmiterjoin%
\definecolor{currentfill}{rgb}{1.000000,1.000000,1.000000}%
\pgfsetfillcolor{currentfill}%
\pgfsetlinewidth{0.000000pt}%
\definecolor{currentstroke}{rgb}{1.000000,1.000000,1.000000}%
\pgfsetstrokecolor{currentstroke}%
\pgfsetdash{}{0pt}%
\pgfpathmoveto{\pgfqpoint{0.000000in}{0.000000in}}%
\pgfpathlineto{\pgfqpoint{6.000000in}{0.000000in}}%
\pgfpathlineto{\pgfqpoint{6.000000in}{4.000000in}}%
\pgfpathlineto{\pgfqpoint{0.000000in}{4.000000in}}%
\pgfpathlineto{\pgfqpoint{0.000000in}{0.000000in}}%
\pgfpathclose%
\pgfusepath{fill}%
\end{pgfscope}%
\begin{pgfscope}%
\pgfsetbuttcap%
\pgfsetmiterjoin%
\definecolor{currentfill}{rgb}{1.000000,1.000000,1.000000}%
\pgfsetfillcolor{currentfill}%
\pgfsetlinewidth{0.000000pt}%
\definecolor{currentstroke}{rgb}{0.000000,0.000000,0.000000}%
\pgfsetstrokecolor{currentstroke}%
\pgfsetstrokeopacity{0.000000}%
\pgfsetdash{}{0pt}%
\pgfpathmoveto{\pgfqpoint{0.703703in}{0.585278in}}%
\pgfpathlineto{\pgfqpoint{5.780625in}{0.585278in}}%
\pgfpathlineto{\pgfqpoint{5.780625in}{3.850000in}}%
\pgfpathlineto{\pgfqpoint{0.703703in}{3.850000in}}%
\pgfpathlineto{\pgfqpoint{0.703703in}{0.585278in}}%
\pgfpathclose%
\pgfusepath{fill}%
\end{pgfscope}%
\begin{pgfscope}%
\pgfpathrectangle{\pgfqpoint{0.703703in}{0.585278in}}{\pgfqpoint{5.076922in}{3.264722in}}%
\pgfusepath{clip}%
\pgfsetrectcap%
\pgfsetroundjoin%
\pgfsetlinewidth{0.803000pt}%
\definecolor{currentstroke}{rgb}{0.690196,0.690196,0.690196}%
\pgfsetstrokecolor{currentstroke}%
\pgfsetdash{}{0pt}%
\pgfpathmoveto{\pgfqpoint{0.703703in}{0.585278in}}%
\pgfpathlineto{\pgfqpoint{0.703703in}{3.850000in}}%
\pgfusepath{stroke}%
\end{pgfscope}%
\begin{pgfscope}%
\pgfsetbuttcap%
\pgfsetroundjoin%
\definecolor{currentfill}{rgb}{0.000000,0.000000,0.000000}%
\pgfsetfillcolor{currentfill}%
\pgfsetlinewidth{0.803000pt}%
\definecolor{currentstroke}{rgb}{0.000000,0.000000,0.000000}%
\pgfsetstrokecolor{currentstroke}%
\pgfsetdash{}{0pt}%
\pgfsys@defobject{currentmarker}{\pgfqpoint{0.000000in}{-0.048611in}}{\pgfqpoint{0.000000in}{0.000000in}}{%
\pgfpathmoveto{\pgfqpoint{0.000000in}{0.000000in}}%
\pgfpathlineto{\pgfqpoint{0.000000in}{-0.048611in}}%
\pgfusepath{stroke,fill}%
}%
\begin{pgfscope}%
\pgfsys@transformshift{0.703703in}{0.585278in}%
\pgfsys@useobject{currentmarker}{}%
\end{pgfscope}%
\end{pgfscope}%
\begin{pgfscope}%
\definecolor{textcolor}{rgb}{0.000000,0.000000,0.000000}%
\pgfsetstrokecolor{textcolor}%
\pgfsetfillcolor{textcolor}%
\pgftext[x=0.703703in,y=0.488056in,,top]{\color{textcolor}\rmfamily\fontsize{10.000000}{12.000000}\selectfont 0}%
\end{pgfscope}%
\begin{pgfscope}%
\pgfpathrectangle{\pgfqpoint{0.703703in}{0.585278in}}{\pgfqpoint{5.076922in}{3.264722in}}%
\pgfusepath{clip}%
\pgfsetrectcap%
\pgfsetroundjoin%
\pgfsetlinewidth{0.803000pt}%
\definecolor{currentstroke}{rgb}{0.690196,0.690196,0.690196}%
\pgfsetstrokecolor{currentstroke}%
\pgfsetdash{}{0pt}%
\pgfpathmoveto{\pgfqpoint{1.719087in}{0.585278in}}%
\pgfpathlineto{\pgfqpoint{1.719087in}{3.850000in}}%
\pgfusepath{stroke}%
\end{pgfscope}%
\begin{pgfscope}%
\pgfsetbuttcap%
\pgfsetroundjoin%
\definecolor{currentfill}{rgb}{0.000000,0.000000,0.000000}%
\pgfsetfillcolor{currentfill}%
\pgfsetlinewidth{0.803000pt}%
\definecolor{currentstroke}{rgb}{0.000000,0.000000,0.000000}%
\pgfsetstrokecolor{currentstroke}%
\pgfsetdash{}{0pt}%
\pgfsys@defobject{currentmarker}{\pgfqpoint{0.000000in}{-0.048611in}}{\pgfqpoint{0.000000in}{0.000000in}}{%
\pgfpathmoveto{\pgfqpoint{0.000000in}{0.000000in}}%
\pgfpathlineto{\pgfqpoint{0.000000in}{-0.048611in}}%
\pgfusepath{stroke,fill}%
}%
\begin{pgfscope}%
\pgfsys@transformshift{1.719087in}{0.585278in}%
\pgfsys@useobject{currentmarker}{}%
\end{pgfscope}%
\end{pgfscope}%
\begin{pgfscope}%
\definecolor{textcolor}{rgb}{0.000000,0.000000,0.000000}%
\pgfsetstrokecolor{textcolor}%
\pgfsetfillcolor{textcolor}%
\pgftext[x=1.719087in,y=0.488056in,,top]{\color{textcolor}\rmfamily\fontsize{10.000000}{12.000000}\selectfont 2}%
\end{pgfscope}%
\begin{pgfscope}%
\pgfpathrectangle{\pgfqpoint{0.703703in}{0.585278in}}{\pgfqpoint{5.076922in}{3.264722in}}%
\pgfusepath{clip}%
\pgfsetrectcap%
\pgfsetroundjoin%
\pgfsetlinewidth{0.803000pt}%
\definecolor{currentstroke}{rgb}{0.690196,0.690196,0.690196}%
\pgfsetstrokecolor{currentstroke}%
\pgfsetdash{}{0pt}%
\pgfpathmoveto{\pgfqpoint{2.734472in}{0.585278in}}%
\pgfpathlineto{\pgfqpoint{2.734472in}{3.850000in}}%
\pgfusepath{stroke}%
\end{pgfscope}%
\begin{pgfscope}%
\pgfsetbuttcap%
\pgfsetroundjoin%
\definecolor{currentfill}{rgb}{0.000000,0.000000,0.000000}%
\pgfsetfillcolor{currentfill}%
\pgfsetlinewidth{0.803000pt}%
\definecolor{currentstroke}{rgb}{0.000000,0.000000,0.000000}%
\pgfsetstrokecolor{currentstroke}%
\pgfsetdash{}{0pt}%
\pgfsys@defobject{currentmarker}{\pgfqpoint{0.000000in}{-0.048611in}}{\pgfqpoint{0.000000in}{0.000000in}}{%
\pgfpathmoveto{\pgfqpoint{0.000000in}{0.000000in}}%
\pgfpathlineto{\pgfqpoint{0.000000in}{-0.048611in}}%
\pgfusepath{stroke,fill}%
}%
\begin{pgfscope}%
\pgfsys@transformshift{2.734472in}{0.585278in}%
\pgfsys@useobject{currentmarker}{}%
\end{pgfscope}%
\end{pgfscope}%
\begin{pgfscope}%
\definecolor{textcolor}{rgb}{0.000000,0.000000,0.000000}%
\pgfsetstrokecolor{textcolor}%
\pgfsetfillcolor{textcolor}%
\pgftext[x=2.734472in,y=0.488056in,,top]{\color{textcolor}\rmfamily\fontsize{10.000000}{12.000000}\selectfont 4}%
\end{pgfscope}%
\begin{pgfscope}%
\pgfpathrectangle{\pgfqpoint{0.703703in}{0.585278in}}{\pgfqpoint{5.076922in}{3.264722in}}%
\pgfusepath{clip}%
\pgfsetrectcap%
\pgfsetroundjoin%
\pgfsetlinewidth{0.803000pt}%
\definecolor{currentstroke}{rgb}{0.690196,0.690196,0.690196}%
\pgfsetstrokecolor{currentstroke}%
\pgfsetdash{}{0pt}%
\pgfpathmoveto{\pgfqpoint{3.749856in}{0.585278in}}%
\pgfpathlineto{\pgfqpoint{3.749856in}{3.850000in}}%
\pgfusepath{stroke}%
\end{pgfscope}%
\begin{pgfscope}%
\pgfsetbuttcap%
\pgfsetroundjoin%
\definecolor{currentfill}{rgb}{0.000000,0.000000,0.000000}%
\pgfsetfillcolor{currentfill}%
\pgfsetlinewidth{0.803000pt}%
\definecolor{currentstroke}{rgb}{0.000000,0.000000,0.000000}%
\pgfsetstrokecolor{currentstroke}%
\pgfsetdash{}{0pt}%
\pgfsys@defobject{currentmarker}{\pgfqpoint{0.000000in}{-0.048611in}}{\pgfqpoint{0.000000in}{0.000000in}}{%
\pgfpathmoveto{\pgfqpoint{0.000000in}{0.000000in}}%
\pgfpathlineto{\pgfqpoint{0.000000in}{-0.048611in}}%
\pgfusepath{stroke,fill}%
}%
\begin{pgfscope}%
\pgfsys@transformshift{3.749856in}{0.585278in}%
\pgfsys@useobject{currentmarker}{}%
\end{pgfscope}%
\end{pgfscope}%
\begin{pgfscope}%
\definecolor{textcolor}{rgb}{0.000000,0.000000,0.000000}%
\pgfsetstrokecolor{textcolor}%
\pgfsetfillcolor{textcolor}%
\pgftext[x=3.749856in,y=0.488056in,,top]{\color{textcolor}\rmfamily\fontsize{10.000000}{12.000000}\selectfont 6}%
\end{pgfscope}%
\begin{pgfscope}%
\pgfpathrectangle{\pgfqpoint{0.703703in}{0.585278in}}{\pgfqpoint{5.076922in}{3.264722in}}%
\pgfusepath{clip}%
\pgfsetrectcap%
\pgfsetroundjoin%
\pgfsetlinewidth{0.803000pt}%
\definecolor{currentstroke}{rgb}{0.690196,0.690196,0.690196}%
\pgfsetstrokecolor{currentstroke}%
\pgfsetdash{}{0pt}%
\pgfpathmoveto{\pgfqpoint{4.765241in}{0.585278in}}%
\pgfpathlineto{\pgfqpoint{4.765241in}{3.850000in}}%
\pgfusepath{stroke}%
\end{pgfscope}%
\begin{pgfscope}%
\pgfsetbuttcap%
\pgfsetroundjoin%
\definecolor{currentfill}{rgb}{0.000000,0.000000,0.000000}%
\pgfsetfillcolor{currentfill}%
\pgfsetlinewidth{0.803000pt}%
\definecolor{currentstroke}{rgb}{0.000000,0.000000,0.000000}%
\pgfsetstrokecolor{currentstroke}%
\pgfsetdash{}{0pt}%
\pgfsys@defobject{currentmarker}{\pgfqpoint{0.000000in}{-0.048611in}}{\pgfqpoint{0.000000in}{0.000000in}}{%
\pgfpathmoveto{\pgfqpoint{0.000000in}{0.000000in}}%
\pgfpathlineto{\pgfqpoint{0.000000in}{-0.048611in}}%
\pgfusepath{stroke,fill}%
}%
\begin{pgfscope}%
\pgfsys@transformshift{4.765241in}{0.585278in}%
\pgfsys@useobject{currentmarker}{}%
\end{pgfscope}%
\end{pgfscope}%
\begin{pgfscope}%
\definecolor{textcolor}{rgb}{0.000000,0.000000,0.000000}%
\pgfsetstrokecolor{textcolor}%
\pgfsetfillcolor{textcolor}%
\pgftext[x=4.765241in,y=0.488056in,,top]{\color{textcolor}\rmfamily\fontsize{10.000000}{12.000000}\selectfont 8}%
\end{pgfscope}%
\begin{pgfscope}%
\pgfpathrectangle{\pgfqpoint{0.703703in}{0.585278in}}{\pgfqpoint{5.076922in}{3.264722in}}%
\pgfusepath{clip}%
\pgfsetrectcap%
\pgfsetroundjoin%
\pgfsetlinewidth{0.803000pt}%
\definecolor{currentstroke}{rgb}{0.690196,0.690196,0.690196}%
\pgfsetstrokecolor{currentstroke}%
\pgfsetdash{}{0pt}%
\pgfpathmoveto{\pgfqpoint{5.780625in}{0.585278in}}%
\pgfpathlineto{\pgfqpoint{5.780625in}{3.850000in}}%
\pgfusepath{stroke}%
\end{pgfscope}%
\begin{pgfscope}%
\pgfsetbuttcap%
\pgfsetroundjoin%
\definecolor{currentfill}{rgb}{0.000000,0.000000,0.000000}%
\pgfsetfillcolor{currentfill}%
\pgfsetlinewidth{0.803000pt}%
\definecolor{currentstroke}{rgb}{0.000000,0.000000,0.000000}%
\pgfsetstrokecolor{currentstroke}%
\pgfsetdash{}{0pt}%
\pgfsys@defobject{currentmarker}{\pgfqpoint{0.000000in}{-0.048611in}}{\pgfqpoint{0.000000in}{0.000000in}}{%
\pgfpathmoveto{\pgfqpoint{0.000000in}{0.000000in}}%
\pgfpathlineto{\pgfqpoint{0.000000in}{-0.048611in}}%
\pgfusepath{stroke,fill}%
}%
\begin{pgfscope}%
\pgfsys@transformshift{5.780625in}{0.585278in}%
\pgfsys@useobject{currentmarker}{}%
\end{pgfscope}%
\end{pgfscope}%
\begin{pgfscope}%
\definecolor{textcolor}{rgb}{0.000000,0.000000,0.000000}%
\pgfsetstrokecolor{textcolor}%
\pgfsetfillcolor{textcolor}%
\pgftext[x=5.780625in,y=0.488056in,,top]{\color{textcolor}\rmfamily\fontsize{10.000000}{12.000000}\selectfont 10}%
\end{pgfscope}%
\begin{pgfscope}%
\definecolor{textcolor}{rgb}{0.000000,0.000000,0.000000}%
\pgfsetstrokecolor{textcolor}%
\pgfsetfillcolor{textcolor}%
\pgftext[x=3.242164in,y=0.309043in,,top]{\color{textcolor}\rmfamily\fontsize{10.000000}{12.000000}\selectfont Time (s)}%
\end{pgfscope}%
\begin{pgfscope}%
\pgfpathrectangle{\pgfqpoint{0.703703in}{0.585278in}}{\pgfqpoint{5.076922in}{3.264722in}}%
\pgfusepath{clip}%
\pgfsetrectcap%
\pgfsetroundjoin%
\pgfsetlinewidth{0.803000pt}%
\definecolor{currentstroke}{rgb}{0.690196,0.690196,0.690196}%
\pgfsetstrokecolor{currentstroke}%
\pgfsetdash{}{0pt}%
\pgfpathmoveto{\pgfqpoint{0.703703in}{0.901779in}}%
\pgfpathlineto{\pgfqpoint{5.780625in}{0.901779in}}%
\pgfusepath{stroke}%
\end{pgfscope}%
\begin{pgfscope}%
\pgfsetbuttcap%
\pgfsetroundjoin%
\definecolor{currentfill}{rgb}{0.000000,0.000000,0.000000}%
\pgfsetfillcolor{currentfill}%
\pgfsetlinewidth{0.803000pt}%
\definecolor{currentstroke}{rgb}{0.000000,0.000000,0.000000}%
\pgfsetstrokecolor{currentstroke}%
\pgfsetdash{}{0pt}%
\pgfsys@defobject{currentmarker}{\pgfqpoint{-0.048611in}{0.000000in}}{\pgfqpoint{-0.000000in}{0.000000in}}{%
\pgfpathmoveto{\pgfqpoint{-0.000000in}{0.000000in}}%
\pgfpathlineto{\pgfqpoint{-0.048611in}{0.000000in}}%
\pgfusepath{stroke,fill}%
}%
\begin{pgfscope}%
\pgfsys@transformshift{0.703703in}{0.901779in}%
\pgfsys@useobject{currentmarker}{}%
\end{pgfscope}%
\end{pgfscope}%
\begin{pgfscope}%
\definecolor{textcolor}{rgb}{0.000000,0.000000,0.000000}%
\pgfsetstrokecolor{textcolor}%
\pgfsetfillcolor{textcolor}%
\pgftext[x=0.318478in, y=0.853553in, left, base]{\color{textcolor}\rmfamily\fontsize{10.000000}{12.000000}\selectfont \(\displaystyle {10^{-4}}\)}%
\end{pgfscope}%
\begin{pgfscope}%
\pgfpathrectangle{\pgfqpoint{0.703703in}{0.585278in}}{\pgfqpoint{5.076922in}{3.264722in}}%
\pgfusepath{clip}%
\pgfsetrectcap%
\pgfsetroundjoin%
\pgfsetlinewidth{0.803000pt}%
\definecolor{currentstroke}{rgb}{0.690196,0.690196,0.690196}%
\pgfsetstrokecolor{currentstroke}%
\pgfsetdash{}{0pt}%
\pgfpathmoveto{\pgfqpoint{0.703703in}{1.549413in}}%
\pgfpathlineto{\pgfqpoint{5.780625in}{1.549413in}}%
\pgfusepath{stroke}%
\end{pgfscope}%
\begin{pgfscope}%
\pgfsetbuttcap%
\pgfsetroundjoin%
\definecolor{currentfill}{rgb}{0.000000,0.000000,0.000000}%
\pgfsetfillcolor{currentfill}%
\pgfsetlinewidth{0.803000pt}%
\definecolor{currentstroke}{rgb}{0.000000,0.000000,0.000000}%
\pgfsetstrokecolor{currentstroke}%
\pgfsetdash{}{0pt}%
\pgfsys@defobject{currentmarker}{\pgfqpoint{-0.048611in}{0.000000in}}{\pgfqpoint{-0.000000in}{0.000000in}}{%
\pgfpathmoveto{\pgfqpoint{-0.000000in}{0.000000in}}%
\pgfpathlineto{\pgfqpoint{-0.048611in}{0.000000in}}%
\pgfusepath{stroke,fill}%
}%
\begin{pgfscope}%
\pgfsys@transformshift{0.703703in}{1.549413in}%
\pgfsys@useobject{currentmarker}{}%
\end{pgfscope}%
\end{pgfscope}%
\begin{pgfscope}%
\definecolor{textcolor}{rgb}{0.000000,0.000000,0.000000}%
\pgfsetstrokecolor{textcolor}%
\pgfsetfillcolor{textcolor}%
\pgftext[x=0.318478in, y=1.501188in, left, base]{\color{textcolor}\rmfamily\fontsize{10.000000}{12.000000}\selectfont \(\displaystyle {10^{-3}}\)}%
\end{pgfscope}%
\begin{pgfscope}%
\pgfpathrectangle{\pgfqpoint{0.703703in}{0.585278in}}{\pgfqpoint{5.076922in}{3.264722in}}%
\pgfusepath{clip}%
\pgfsetrectcap%
\pgfsetroundjoin%
\pgfsetlinewidth{0.803000pt}%
\definecolor{currentstroke}{rgb}{0.690196,0.690196,0.690196}%
\pgfsetstrokecolor{currentstroke}%
\pgfsetdash{}{0pt}%
\pgfpathmoveto{\pgfqpoint{0.703703in}{2.197047in}}%
\pgfpathlineto{\pgfqpoint{5.780625in}{2.197047in}}%
\pgfusepath{stroke}%
\end{pgfscope}%
\begin{pgfscope}%
\pgfsetbuttcap%
\pgfsetroundjoin%
\definecolor{currentfill}{rgb}{0.000000,0.000000,0.000000}%
\pgfsetfillcolor{currentfill}%
\pgfsetlinewidth{0.803000pt}%
\definecolor{currentstroke}{rgb}{0.000000,0.000000,0.000000}%
\pgfsetstrokecolor{currentstroke}%
\pgfsetdash{}{0pt}%
\pgfsys@defobject{currentmarker}{\pgfqpoint{-0.048611in}{0.000000in}}{\pgfqpoint{-0.000000in}{0.000000in}}{%
\pgfpathmoveto{\pgfqpoint{-0.000000in}{0.000000in}}%
\pgfpathlineto{\pgfqpoint{-0.048611in}{0.000000in}}%
\pgfusepath{stroke,fill}%
}%
\begin{pgfscope}%
\pgfsys@transformshift{0.703703in}{2.197047in}%
\pgfsys@useobject{currentmarker}{}%
\end{pgfscope}%
\end{pgfscope}%
\begin{pgfscope}%
\definecolor{textcolor}{rgb}{0.000000,0.000000,0.000000}%
\pgfsetstrokecolor{textcolor}%
\pgfsetfillcolor{textcolor}%
\pgftext[x=0.318478in, y=2.148822in, left, base]{\color{textcolor}\rmfamily\fontsize{10.000000}{12.000000}\selectfont \(\displaystyle {10^{-2}}\)}%
\end{pgfscope}%
\begin{pgfscope}%
\pgfpathrectangle{\pgfqpoint{0.703703in}{0.585278in}}{\pgfqpoint{5.076922in}{3.264722in}}%
\pgfusepath{clip}%
\pgfsetrectcap%
\pgfsetroundjoin%
\pgfsetlinewidth{0.803000pt}%
\definecolor{currentstroke}{rgb}{0.690196,0.690196,0.690196}%
\pgfsetstrokecolor{currentstroke}%
\pgfsetdash{}{0pt}%
\pgfpathmoveto{\pgfqpoint{0.703703in}{2.844682in}}%
\pgfpathlineto{\pgfqpoint{5.780625in}{2.844682in}}%
\pgfusepath{stroke}%
\end{pgfscope}%
\begin{pgfscope}%
\pgfsetbuttcap%
\pgfsetroundjoin%
\definecolor{currentfill}{rgb}{0.000000,0.000000,0.000000}%
\pgfsetfillcolor{currentfill}%
\pgfsetlinewidth{0.803000pt}%
\definecolor{currentstroke}{rgb}{0.000000,0.000000,0.000000}%
\pgfsetstrokecolor{currentstroke}%
\pgfsetdash{}{0pt}%
\pgfsys@defobject{currentmarker}{\pgfqpoint{-0.048611in}{0.000000in}}{\pgfqpoint{-0.000000in}{0.000000in}}{%
\pgfpathmoveto{\pgfqpoint{-0.000000in}{0.000000in}}%
\pgfpathlineto{\pgfqpoint{-0.048611in}{0.000000in}}%
\pgfusepath{stroke,fill}%
}%
\begin{pgfscope}%
\pgfsys@transformshift{0.703703in}{2.844682in}%
\pgfsys@useobject{currentmarker}{}%
\end{pgfscope}%
\end{pgfscope}%
\begin{pgfscope}%
\definecolor{textcolor}{rgb}{0.000000,0.000000,0.000000}%
\pgfsetstrokecolor{textcolor}%
\pgfsetfillcolor{textcolor}%
\pgftext[x=0.318478in, y=2.796456in, left, base]{\color{textcolor}\rmfamily\fontsize{10.000000}{12.000000}\selectfont \(\displaystyle {10^{-1}}\)}%
\end{pgfscope}%
\begin{pgfscope}%
\pgfpathrectangle{\pgfqpoint{0.703703in}{0.585278in}}{\pgfqpoint{5.076922in}{3.264722in}}%
\pgfusepath{clip}%
\pgfsetrectcap%
\pgfsetroundjoin%
\pgfsetlinewidth{0.803000pt}%
\definecolor{currentstroke}{rgb}{0.690196,0.690196,0.690196}%
\pgfsetstrokecolor{currentstroke}%
\pgfsetdash{}{0pt}%
\pgfpathmoveto{\pgfqpoint{0.703703in}{3.492316in}}%
\pgfpathlineto{\pgfqpoint{5.780625in}{3.492316in}}%
\pgfusepath{stroke}%
\end{pgfscope}%
\begin{pgfscope}%
\pgfsetbuttcap%
\pgfsetroundjoin%
\definecolor{currentfill}{rgb}{0.000000,0.000000,0.000000}%
\pgfsetfillcolor{currentfill}%
\pgfsetlinewidth{0.803000pt}%
\definecolor{currentstroke}{rgb}{0.000000,0.000000,0.000000}%
\pgfsetstrokecolor{currentstroke}%
\pgfsetdash{}{0pt}%
\pgfsys@defobject{currentmarker}{\pgfqpoint{-0.048611in}{0.000000in}}{\pgfqpoint{-0.000000in}{0.000000in}}{%
\pgfpathmoveto{\pgfqpoint{-0.000000in}{0.000000in}}%
\pgfpathlineto{\pgfqpoint{-0.048611in}{0.000000in}}%
\pgfusepath{stroke,fill}%
}%
\begin{pgfscope}%
\pgfsys@transformshift{0.703703in}{3.492316in}%
\pgfsys@useobject{currentmarker}{}%
\end{pgfscope}%
\end{pgfscope}%
\begin{pgfscope}%
\definecolor{textcolor}{rgb}{0.000000,0.000000,0.000000}%
\pgfsetstrokecolor{textcolor}%
\pgfsetfillcolor{textcolor}%
\pgftext[x=0.405284in, y=3.444091in, left, base]{\color{textcolor}\rmfamily\fontsize{10.000000}{12.000000}\selectfont \(\displaystyle {10^{0}}\)}%
\end{pgfscope}%
\begin{pgfscope}%
\pgfsetbuttcap%
\pgfsetroundjoin%
\definecolor{currentfill}{rgb}{0.000000,0.000000,0.000000}%
\pgfsetfillcolor{currentfill}%
\pgfsetlinewidth{0.602250pt}%
\definecolor{currentstroke}{rgb}{0.000000,0.000000,0.000000}%
\pgfsetstrokecolor{currentstroke}%
\pgfsetdash{}{0pt}%
\pgfsys@defobject{currentmarker}{\pgfqpoint{-0.027778in}{0.000000in}}{\pgfqpoint{-0.000000in}{0.000000in}}{%
\pgfpathmoveto{\pgfqpoint{-0.000000in}{0.000000in}}%
\pgfpathlineto{\pgfqpoint{-0.027778in}{0.000000in}}%
\pgfusepath{stroke,fill}%
}%
\begin{pgfscope}%
\pgfsys@transformshift{0.703703in}{0.644059in}%
\pgfsys@useobject{currentmarker}{}%
\end{pgfscope}%
\end{pgfscope}%
\begin{pgfscope}%
\pgfsetbuttcap%
\pgfsetroundjoin%
\definecolor{currentfill}{rgb}{0.000000,0.000000,0.000000}%
\pgfsetfillcolor{currentfill}%
\pgfsetlinewidth{0.602250pt}%
\definecolor{currentstroke}{rgb}{0.000000,0.000000,0.000000}%
\pgfsetstrokecolor{currentstroke}%
\pgfsetdash{}{0pt}%
\pgfsys@defobject{currentmarker}{\pgfqpoint{-0.027778in}{0.000000in}}{\pgfqpoint{-0.000000in}{0.000000in}}{%
\pgfpathmoveto{\pgfqpoint{-0.000000in}{0.000000in}}%
\pgfpathlineto{\pgfqpoint{-0.027778in}{0.000000in}}%
\pgfusepath{stroke,fill}%
}%
\begin{pgfscope}%
\pgfsys@transformshift{0.703703in}{0.706821in}%
\pgfsys@useobject{currentmarker}{}%
\end{pgfscope}%
\end{pgfscope}%
\begin{pgfscope}%
\pgfsetbuttcap%
\pgfsetroundjoin%
\definecolor{currentfill}{rgb}{0.000000,0.000000,0.000000}%
\pgfsetfillcolor{currentfill}%
\pgfsetlinewidth{0.602250pt}%
\definecolor{currentstroke}{rgb}{0.000000,0.000000,0.000000}%
\pgfsetstrokecolor{currentstroke}%
\pgfsetdash{}{0pt}%
\pgfsys@defobject{currentmarker}{\pgfqpoint{-0.027778in}{0.000000in}}{\pgfqpoint{-0.000000in}{0.000000in}}{%
\pgfpathmoveto{\pgfqpoint{-0.000000in}{0.000000in}}%
\pgfpathlineto{\pgfqpoint{-0.027778in}{0.000000in}}%
\pgfusepath{stroke,fill}%
}%
\begin{pgfscope}%
\pgfsys@transformshift{0.703703in}{0.758102in}%
\pgfsys@useobject{currentmarker}{}%
\end{pgfscope}%
\end{pgfscope}%
\begin{pgfscope}%
\pgfsetbuttcap%
\pgfsetroundjoin%
\definecolor{currentfill}{rgb}{0.000000,0.000000,0.000000}%
\pgfsetfillcolor{currentfill}%
\pgfsetlinewidth{0.602250pt}%
\definecolor{currentstroke}{rgb}{0.000000,0.000000,0.000000}%
\pgfsetstrokecolor{currentstroke}%
\pgfsetdash{}{0pt}%
\pgfsys@defobject{currentmarker}{\pgfqpoint{-0.027778in}{0.000000in}}{\pgfqpoint{-0.000000in}{0.000000in}}{%
\pgfpathmoveto{\pgfqpoint{-0.000000in}{0.000000in}}%
\pgfpathlineto{\pgfqpoint{-0.027778in}{0.000000in}}%
\pgfusepath{stroke,fill}%
}%
\begin{pgfscope}%
\pgfsys@transformshift{0.703703in}{0.801459in}%
\pgfsys@useobject{currentmarker}{}%
\end{pgfscope}%
\end{pgfscope}%
\begin{pgfscope}%
\pgfsetbuttcap%
\pgfsetroundjoin%
\definecolor{currentfill}{rgb}{0.000000,0.000000,0.000000}%
\pgfsetfillcolor{currentfill}%
\pgfsetlinewidth{0.602250pt}%
\definecolor{currentstroke}{rgb}{0.000000,0.000000,0.000000}%
\pgfsetstrokecolor{currentstroke}%
\pgfsetdash{}{0pt}%
\pgfsys@defobject{currentmarker}{\pgfqpoint{-0.027778in}{0.000000in}}{\pgfqpoint{-0.000000in}{0.000000in}}{%
\pgfpathmoveto{\pgfqpoint{-0.000000in}{0.000000in}}%
\pgfpathlineto{\pgfqpoint{-0.027778in}{0.000000in}}%
\pgfusepath{stroke,fill}%
}%
\begin{pgfscope}%
\pgfsys@transformshift{0.703703in}{0.839016in}%
\pgfsys@useobject{currentmarker}{}%
\end{pgfscope}%
\end{pgfscope}%
\begin{pgfscope}%
\pgfsetbuttcap%
\pgfsetroundjoin%
\definecolor{currentfill}{rgb}{0.000000,0.000000,0.000000}%
\pgfsetfillcolor{currentfill}%
\pgfsetlinewidth{0.602250pt}%
\definecolor{currentstroke}{rgb}{0.000000,0.000000,0.000000}%
\pgfsetstrokecolor{currentstroke}%
\pgfsetdash{}{0pt}%
\pgfsys@defobject{currentmarker}{\pgfqpoint{-0.027778in}{0.000000in}}{\pgfqpoint{-0.000000in}{0.000000in}}{%
\pgfpathmoveto{\pgfqpoint{-0.000000in}{0.000000in}}%
\pgfpathlineto{\pgfqpoint{-0.027778in}{0.000000in}}%
\pgfusepath{stroke,fill}%
}%
\begin{pgfscope}%
\pgfsys@transformshift{0.703703in}{0.872145in}%
\pgfsys@useobject{currentmarker}{}%
\end{pgfscope}%
\end{pgfscope}%
\begin{pgfscope}%
\pgfsetbuttcap%
\pgfsetroundjoin%
\definecolor{currentfill}{rgb}{0.000000,0.000000,0.000000}%
\pgfsetfillcolor{currentfill}%
\pgfsetlinewidth{0.602250pt}%
\definecolor{currentstroke}{rgb}{0.000000,0.000000,0.000000}%
\pgfsetstrokecolor{currentstroke}%
\pgfsetdash{}{0pt}%
\pgfsys@defobject{currentmarker}{\pgfqpoint{-0.027778in}{0.000000in}}{\pgfqpoint{-0.000000in}{0.000000in}}{%
\pgfpathmoveto{\pgfqpoint{-0.000000in}{0.000000in}}%
\pgfpathlineto{\pgfqpoint{-0.027778in}{0.000000in}}%
\pgfusepath{stroke,fill}%
}%
\begin{pgfscope}%
\pgfsys@transformshift{0.703703in}{1.096736in}%
\pgfsys@useobject{currentmarker}{}%
\end{pgfscope}%
\end{pgfscope}%
\begin{pgfscope}%
\pgfsetbuttcap%
\pgfsetroundjoin%
\definecolor{currentfill}{rgb}{0.000000,0.000000,0.000000}%
\pgfsetfillcolor{currentfill}%
\pgfsetlinewidth{0.602250pt}%
\definecolor{currentstroke}{rgb}{0.000000,0.000000,0.000000}%
\pgfsetstrokecolor{currentstroke}%
\pgfsetdash{}{0pt}%
\pgfsys@defobject{currentmarker}{\pgfqpoint{-0.027778in}{0.000000in}}{\pgfqpoint{-0.000000in}{0.000000in}}{%
\pgfpathmoveto{\pgfqpoint{-0.000000in}{0.000000in}}%
\pgfpathlineto{\pgfqpoint{-0.027778in}{0.000000in}}%
\pgfusepath{stroke,fill}%
}%
\begin{pgfscope}%
\pgfsys@transformshift{0.703703in}{1.210779in}%
\pgfsys@useobject{currentmarker}{}%
\end{pgfscope}%
\end{pgfscope}%
\begin{pgfscope}%
\pgfsetbuttcap%
\pgfsetroundjoin%
\definecolor{currentfill}{rgb}{0.000000,0.000000,0.000000}%
\pgfsetfillcolor{currentfill}%
\pgfsetlinewidth{0.602250pt}%
\definecolor{currentstroke}{rgb}{0.000000,0.000000,0.000000}%
\pgfsetstrokecolor{currentstroke}%
\pgfsetdash{}{0pt}%
\pgfsys@defobject{currentmarker}{\pgfqpoint{-0.027778in}{0.000000in}}{\pgfqpoint{-0.000000in}{0.000000in}}{%
\pgfpathmoveto{\pgfqpoint{-0.000000in}{0.000000in}}%
\pgfpathlineto{\pgfqpoint{-0.027778in}{0.000000in}}%
\pgfusepath{stroke,fill}%
}%
\begin{pgfscope}%
\pgfsys@transformshift{0.703703in}{1.291693in}%
\pgfsys@useobject{currentmarker}{}%
\end{pgfscope}%
\end{pgfscope}%
\begin{pgfscope}%
\pgfsetbuttcap%
\pgfsetroundjoin%
\definecolor{currentfill}{rgb}{0.000000,0.000000,0.000000}%
\pgfsetfillcolor{currentfill}%
\pgfsetlinewidth{0.602250pt}%
\definecolor{currentstroke}{rgb}{0.000000,0.000000,0.000000}%
\pgfsetstrokecolor{currentstroke}%
\pgfsetdash{}{0pt}%
\pgfsys@defobject{currentmarker}{\pgfqpoint{-0.027778in}{0.000000in}}{\pgfqpoint{-0.000000in}{0.000000in}}{%
\pgfpathmoveto{\pgfqpoint{-0.000000in}{0.000000in}}%
\pgfpathlineto{\pgfqpoint{-0.027778in}{0.000000in}}%
\pgfusepath{stroke,fill}%
}%
\begin{pgfscope}%
\pgfsys@transformshift{0.703703in}{1.354456in}%
\pgfsys@useobject{currentmarker}{}%
\end{pgfscope}%
\end{pgfscope}%
\begin{pgfscope}%
\pgfsetbuttcap%
\pgfsetroundjoin%
\definecolor{currentfill}{rgb}{0.000000,0.000000,0.000000}%
\pgfsetfillcolor{currentfill}%
\pgfsetlinewidth{0.602250pt}%
\definecolor{currentstroke}{rgb}{0.000000,0.000000,0.000000}%
\pgfsetstrokecolor{currentstroke}%
\pgfsetdash{}{0pt}%
\pgfsys@defobject{currentmarker}{\pgfqpoint{-0.027778in}{0.000000in}}{\pgfqpoint{-0.000000in}{0.000000in}}{%
\pgfpathmoveto{\pgfqpoint{-0.000000in}{0.000000in}}%
\pgfpathlineto{\pgfqpoint{-0.027778in}{0.000000in}}%
\pgfusepath{stroke,fill}%
}%
\begin{pgfscope}%
\pgfsys@transformshift{0.703703in}{1.405736in}%
\pgfsys@useobject{currentmarker}{}%
\end{pgfscope}%
\end{pgfscope}%
\begin{pgfscope}%
\pgfsetbuttcap%
\pgfsetroundjoin%
\definecolor{currentfill}{rgb}{0.000000,0.000000,0.000000}%
\pgfsetfillcolor{currentfill}%
\pgfsetlinewidth{0.602250pt}%
\definecolor{currentstroke}{rgb}{0.000000,0.000000,0.000000}%
\pgfsetstrokecolor{currentstroke}%
\pgfsetdash{}{0pt}%
\pgfsys@defobject{currentmarker}{\pgfqpoint{-0.027778in}{0.000000in}}{\pgfqpoint{-0.000000in}{0.000000in}}{%
\pgfpathmoveto{\pgfqpoint{-0.000000in}{0.000000in}}%
\pgfpathlineto{\pgfqpoint{-0.027778in}{0.000000in}}%
\pgfusepath{stroke,fill}%
}%
\begin{pgfscope}%
\pgfsys@transformshift{0.703703in}{1.449093in}%
\pgfsys@useobject{currentmarker}{}%
\end{pgfscope}%
\end{pgfscope}%
\begin{pgfscope}%
\pgfsetbuttcap%
\pgfsetroundjoin%
\definecolor{currentfill}{rgb}{0.000000,0.000000,0.000000}%
\pgfsetfillcolor{currentfill}%
\pgfsetlinewidth{0.602250pt}%
\definecolor{currentstroke}{rgb}{0.000000,0.000000,0.000000}%
\pgfsetstrokecolor{currentstroke}%
\pgfsetdash{}{0pt}%
\pgfsys@defobject{currentmarker}{\pgfqpoint{-0.027778in}{0.000000in}}{\pgfqpoint{-0.000000in}{0.000000in}}{%
\pgfpathmoveto{\pgfqpoint{-0.000000in}{0.000000in}}%
\pgfpathlineto{\pgfqpoint{-0.027778in}{0.000000in}}%
\pgfusepath{stroke,fill}%
}%
\begin{pgfscope}%
\pgfsys@transformshift{0.703703in}{1.486651in}%
\pgfsys@useobject{currentmarker}{}%
\end{pgfscope}%
\end{pgfscope}%
\begin{pgfscope}%
\pgfsetbuttcap%
\pgfsetroundjoin%
\definecolor{currentfill}{rgb}{0.000000,0.000000,0.000000}%
\pgfsetfillcolor{currentfill}%
\pgfsetlinewidth{0.602250pt}%
\definecolor{currentstroke}{rgb}{0.000000,0.000000,0.000000}%
\pgfsetstrokecolor{currentstroke}%
\pgfsetdash{}{0pt}%
\pgfsys@defobject{currentmarker}{\pgfqpoint{-0.027778in}{0.000000in}}{\pgfqpoint{-0.000000in}{0.000000in}}{%
\pgfpathmoveto{\pgfqpoint{-0.000000in}{0.000000in}}%
\pgfpathlineto{\pgfqpoint{-0.027778in}{0.000000in}}%
\pgfusepath{stroke,fill}%
}%
\begin{pgfscope}%
\pgfsys@transformshift{0.703703in}{1.519779in}%
\pgfsys@useobject{currentmarker}{}%
\end{pgfscope}%
\end{pgfscope}%
\begin{pgfscope}%
\pgfsetbuttcap%
\pgfsetroundjoin%
\definecolor{currentfill}{rgb}{0.000000,0.000000,0.000000}%
\pgfsetfillcolor{currentfill}%
\pgfsetlinewidth{0.602250pt}%
\definecolor{currentstroke}{rgb}{0.000000,0.000000,0.000000}%
\pgfsetstrokecolor{currentstroke}%
\pgfsetdash{}{0pt}%
\pgfsys@defobject{currentmarker}{\pgfqpoint{-0.027778in}{0.000000in}}{\pgfqpoint{-0.000000in}{0.000000in}}{%
\pgfpathmoveto{\pgfqpoint{-0.000000in}{0.000000in}}%
\pgfpathlineto{\pgfqpoint{-0.027778in}{0.000000in}}%
\pgfusepath{stroke,fill}%
}%
\begin{pgfscope}%
\pgfsys@transformshift{0.703703in}{1.744370in}%
\pgfsys@useobject{currentmarker}{}%
\end{pgfscope}%
\end{pgfscope}%
\begin{pgfscope}%
\pgfsetbuttcap%
\pgfsetroundjoin%
\definecolor{currentfill}{rgb}{0.000000,0.000000,0.000000}%
\pgfsetfillcolor{currentfill}%
\pgfsetlinewidth{0.602250pt}%
\definecolor{currentstroke}{rgb}{0.000000,0.000000,0.000000}%
\pgfsetstrokecolor{currentstroke}%
\pgfsetdash{}{0pt}%
\pgfsys@defobject{currentmarker}{\pgfqpoint{-0.027778in}{0.000000in}}{\pgfqpoint{-0.000000in}{0.000000in}}{%
\pgfpathmoveto{\pgfqpoint{-0.000000in}{0.000000in}}%
\pgfpathlineto{\pgfqpoint{-0.027778in}{0.000000in}}%
\pgfusepath{stroke,fill}%
}%
\begin{pgfscope}%
\pgfsys@transformshift{0.703703in}{1.858413in}%
\pgfsys@useobject{currentmarker}{}%
\end{pgfscope}%
\end{pgfscope}%
\begin{pgfscope}%
\pgfsetbuttcap%
\pgfsetroundjoin%
\definecolor{currentfill}{rgb}{0.000000,0.000000,0.000000}%
\pgfsetfillcolor{currentfill}%
\pgfsetlinewidth{0.602250pt}%
\definecolor{currentstroke}{rgb}{0.000000,0.000000,0.000000}%
\pgfsetstrokecolor{currentstroke}%
\pgfsetdash{}{0pt}%
\pgfsys@defobject{currentmarker}{\pgfqpoint{-0.027778in}{0.000000in}}{\pgfqpoint{-0.000000in}{0.000000in}}{%
\pgfpathmoveto{\pgfqpoint{-0.000000in}{0.000000in}}%
\pgfpathlineto{\pgfqpoint{-0.027778in}{0.000000in}}%
\pgfusepath{stroke,fill}%
}%
\begin{pgfscope}%
\pgfsys@transformshift{0.703703in}{1.939328in}%
\pgfsys@useobject{currentmarker}{}%
\end{pgfscope}%
\end{pgfscope}%
\begin{pgfscope}%
\pgfsetbuttcap%
\pgfsetroundjoin%
\definecolor{currentfill}{rgb}{0.000000,0.000000,0.000000}%
\pgfsetfillcolor{currentfill}%
\pgfsetlinewidth{0.602250pt}%
\definecolor{currentstroke}{rgb}{0.000000,0.000000,0.000000}%
\pgfsetstrokecolor{currentstroke}%
\pgfsetdash{}{0pt}%
\pgfsys@defobject{currentmarker}{\pgfqpoint{-0.027778in}{0.000000in}}{\pgfqpoint{-0.000000in}{0.000000in}}{%
\pgfpathmoveto{\pgfqpoint{-0.000000in}{0.000000in}}%
\pgfpathlineto{\pgfqpoint{-0.027778in}{0.000000in}}%
\pgfusepath{stroke,fill}%
}%
\begin{pgfscope}%
\pgfsys@transformshift{0.703703in}{2.002090in}%
\pgfsys@useobject{currentmarker}{}%
\end{pgfscope}%
\end{pgfscope}%
\begin{pgfscope}%
\pgfsetbuttcap%
\pgfsetroundjoin%
\definecolor{currentfill}{rgb}{0.000000,0.000000,0.000000}%
\pgfsetfillcolor{currentfill}%
\pgfsetlinewidth{0.602250pt}%
\definecolor{currentstroke}{rgb}{0.000000,0.000000,0.000000}%
\pgfsetstrokecolor{currentstroke}%
\pgfsetdash{}{0pt}%
\pgfsys@defobject{currentmarker}{\pgfqpoint{-0.027778in}{0.000000in}}{\pgfqpoint{-0.000000in}{0.000000in}}{%
\pgfpathmoveto{\pgfqpoint{-0.000000in}{0.000000in}}%
\pgfpathlineto{\pgfqpoint{-0.027778in}{0.000000in}}%
\pgfusepath{stroke,fill}%
}%
\begin{pgfscope}%
\pgfsys@transformshift{0.703703in}{2.053370in}%
\pgfsys@useobject{currentmarker}{}%
\end{pgfscope}%
\end{pgfscope}%
\begin{pgfscope}%
\pgfsetbuttcap%
\pgfsetroundjoin%
\definecolor{currentfill}{rgb}{0.000000,0.000000,0.000000}%
\pgfsetfillcolor{currentfill}%
\pgfsetlinewidth{0.602250pt}%
\definecolor{currentstroke}{rgb}{0.000000,0.000000,0.000000}%
\pgfsetstrokecolor{currentstroke}%
\pgfsetdash{}{0pt}%
\pgfsys@defobject{currentmarker}{\pgfqpoint{-0.027778in}{0.000000in}}{\pgfqpoint{-0.000000in}{0.000000in}}{%
\pgfpathmoveto{\pgfqpoint{-0.000000in}{0.000000in}}%
\pgfpathlineto{\pgfqpoint{-0.027778in}{0.000000in}}%
\pgfusepath{stroke,fill}%
}%
\begin{pgfscope}%
\pgfsys@transformshift{0.703703in}{2.096727in}%
\pgfsys@useobject{currentmarker}{}%
\end{pgfscope}%
\end{pgfscope}%
\begin{pgfscope}%
\pgfsetbuttcap%
\pgfsetroundjoin%
\definecolor{currentfill}{rgb}{0.000000,0.000000,0.000000}%
\pgfsetfillcolor{currentfill}%
\pgfsetlinewidth{0.602250pt}%
\definecolor{currentstroke}{rgb}{0.000000,0.000000,0.000000}%
\pgfsetstrokecolor{currentstroke}%
\pgfsetdash{}{0pt}%
\pgfsys@defobject{currentmarker}{\pgfqpoint{-0.027778in}{0.000000in}}{\pgfqpoint{-0.000000in}{0.000000in}}{%
\pgfpathmoveto{\pgfqpoint{-0.000000in}{0.000000in}}%
\pgfpathlineto{\pgfqpoint{-0.027778in}{0.000000in}}%
\pgfusepath{stroke,fill}%
}%
\begin{pgfscope}%
\pgfsys@transformshift{0.703703in}{2.134285in}%
\pgfsys@useobject{currentmarker}{}%
\end{pgfscope}%
\end{pgfscope}%
\begin{pgfscope}%
\pgfsetbuttcap%
\pgfsetroundjoin%
\definecolor{currentfill}{rgb}{0.000000,0.000000,0.000000}%
\pgfsetfillcolor{currentfill}%
\pgfsetlinewidth{0.602250pt}%
\definecolor{currentstroke}{rgb}{0.000000,0.000000,0.000000}%
\pgfsetstrokecolor{currentstroke}%
\pgfsetdash{}{0pt}%
\pgfsys@defobject{currentmarker}{\pgfqpoint{-0.027778in}{0.000000in}}{\pgfqpoint{-0.000000in}{0.000000in}}{%
\pgfpathmoveto{\pgfqpoint{-0.000000in}{0.000000in}}%
\pgfpathlineto{\pgfqpoint{-0.027778in}{0.000000in}}%
\pgfusepath{stroke,fill}%
}%
\begin{pgfscope}%
\pgfsys@transformshift{0.703703in}{2.167413in}%
\pgfsys@useobject{currentmarker}{}%
\end{pgfscope}%
\end{pgfscope}%
\begin{pgfscope}%
\pgfsetbuttcap%
\pgfsetroundjoin%
\definecolor{currentfill}{rgb}{0.000000,0.000000,0.000000}%
\pgfsetfillcolor{currentfill}%
\pgfsetlinewidth{0.602250pt}%
\definecolor{currentstroke}{rgb}{0.000000,0.000000,0.000000}%
\pgfsetstrokecolor{currentstroke}%
\pgfsetdash{}{0pt}%
\pgfsys@defobject{currentmarker}{\pgfqpoint{-0.027778in}{0.000000in}}{\pgfqpoint{-0.000000in}{0.000000in}}{%
\pgfpathmoveto{\pgfqpoint{-0.000000in}{0.000000in}}%
\pgfpathlineto{\pgfqpoint{-0.027778in}{0.000000in}}%
\pgfusepath{stroke,fill}%
}%
\begin{pgfscope}%
\pgfsys@transformshift{0.703703in}{2.392005in}%
\pgfsys@useobject{currentmarker}{}%
\end{pgfscope}%
\end{pgfscope}%
\begin{pgfscope}%
\pgfsetbuttcap%
\pgfsetroundjoin%
\definecolor{currentfill}{rgb}{0.000000,0.000000,0.000000}%
\pgfsetfillcolor{currentfill}%
\pgfsetlinewidth{0.602250pt}%
\definecolor{currentstroke}{rgb}{0.000000,0.000000,0.000000}%
\pgfsetstrokecolor{currentstroke}%
\pgfsetdash{}{0pt}%
\pgfsys@defobject{currentmarker}{\pgfqpoint{-0.027778in}{0.000000in}}{\pgfqpoint{-0.000000in}{0.000000in}}{%
\pgfpathmoveto{\pgfqpoint{-0.000000in}{0.000000in}}%
\pgfpathlineto{\pgfqpoint{-0.027778in}{0.000000in}}%
\pgfusepath{stroke,fill}%
}%
\begin{pgfscope}%
\pgfsys@transformshift{0.703703in}{2.506047in}%
\pgfsys@useobject{currentmarker}{}%
\end{pgfscope}%
\end{pgfscope}%
\begin{pgfscope}%
\pgfsetbuttcap%
\pgfsetroundjoin%
\definecolor{currentfill}{rgb}{0.000000,0.000000,0.000000}%
\pgfsetfillcolor{currentfill}%
\pgfsetlinewidth{0.602250pt}%
\definecolor{currentstroke}{rgb}{0.000000,0.000000,0.000000}%
\pgfsetstrokecolor{currentstroke}%
\pgfsetdash{}{0pt}%
\pgfsys@defobject{currentmarker}{\pgfqpoint{-0.027778in}{0.000000in}}{\pgfqpoint{-0.000000in}{0.000000in}}{%
\pgfpathmoveto{\pgfqpoint{-0.000000in}{0.000000in}}%
\pgfpathlineto{\pgfqpoint{-0.027778in}{0.000000in}}%
\pgfusepath{stroke,fill}%
}%
\begin{pgfscope}%
\pgfsys@transformshift{0.703703in}{2.586962in}%
\pgfsys@useobject{currentmarker}{}%
\end{pgfscope}%
\end{pgfscope}%
\begin{pgfscope}%
\pgfsetbuttcap%
\pgfsetroundjoin%
\definecolor{currentfill}{rgb}{0.000000,0.000000,0.000000}%
\pgfsetfillcolor{currentfill}%
\pgfsetlinewidth{0.602250pt}%
\definecolor{currentstroke}{rgb}{0.000000,0.000000,0.000000}%
\pgfsetstrokecolor{currentstroke}%
\pgfsetdash{}{0pt}%
\pgfsys@defobject{currentmarker}{\pgfqpoint{-0.027778in}{0.000000in}}{\pgfqpoint{-0.000000in}{0.000000in}}{%
\pgfpathmoveto{\pgfqpoint{-0.000000in}{0.000000in}}%
\pgfpathlineto{\pgfqpoint{-0.027778in}{0.000000in}}%
\pgfusepath{stroke,fill}%
}%
\begin{pgfscope}%
\pgfsys@transformshift{0.703703in}{2.649724in}%
\pgfsys@useobject{currentmarker}{}%
\end{pgfscope}%
\end{pgfscope}%
\begin{pgfscope}%
\pgfsetbuttcap%
\pgfsetroundjoin%
\definecolor{currentfill}{rgb}{0.000000,0.000000,0.000000}%
\pgfsetfillcolor{currentfill}%
\pgfsetlinewidth{0.602250pt}%
\definecolor{currentstroke}{rgb}{0.000000,0.000000,0.000000}%
\pgfsetstrokecolor{currentstroke}%
\pgfsetdash{}{0pt}%
\pgfsys@defobject{currentmarker}{\pgfqpoint{-0.027778in}{0.000000in}}{\pgfqpoint{-0.000000in}{0.000000in}}{%
\pgfpathmoveto{\pgfqpoint{-0.000000in}{0.000000in}}%
\pgfpathlineto{\pgfqpoint{-0.027778in}{0.000000in}}%
\pgfusepath{stroke,fill}%
}%
\begin{pgfscope}%
\pgfsys@transformshift{0.703703in}{2.701005in}%
\pgfsys@useobject{currentmarker}{}%
\end{pgfscope}%
\end{pgfscope}%
\begin{pgfscope}%
\pgfsetbuttcap%
\pgfsetroundjoin%
\definecolor{currentfill}{rgb}{0.000000,0.000000,0.000000}%
\pgfsetfillcolor{currentfill}%
\pgfsetlinewidth{0.602250pt}%
\definecolor{currentstroke}{rgb}{0.000000,0.000000,0.000000}%
\pgfsetstrokecolor{currentstroke}%
\pgfsetdash{}{0pt}%
\pgfsys@defobject{currentmarker}{\pgfqpoint{-0.027778in}{0.000000in}}{\pgfqpoint{-0.000000in}{0.000000in}}{%
\pgfpathmoveto{\pgfqpoint{-0.000000in}{0.000000in}}%
\pgfpathlineto{\pgfqpoint{-0.027778in}{0.000000in}}%
\pgfusepath{stroke,fill}%
}%
\begin{pgfscope}%
\pgfsys@transformshift{0.703703in}{2.744362in}%
\pgfsys@useobject{currentmarker}{}%
\end{pgfscope}%
\end{pgfscope}%
\begin{pgfscope}%
\pgfsetbuttcap%
\pgfsetroundjoin%
\definecolor{currentfill}{rgb}{0.000000,0.000000,0.000000}%
\pgfsetfillcolor{currentfill}%
\pgfsetlinewidth{0.602250pt}%
\definecolor{currentstroke}{rgb}{0.000000,0.000000,0.000000}%
\pgfsetstrokecolor{currentstroke}%
\pgfsetdash{}{0pt}%
\pgfsys@defobject{currentmarker}{\pgfqpoint{-0.027778in}{0.000000in}}{\pgfqpoint{-0.000000in}{0.000000in}}{%
\pgfpathmoveto{\pgfqpoint{-0.000000in}{0.000000in}}%
\pgfpathlineto{\pgfqpoint{-0.027778in}{0.000000in}}%
\pgfusepath{stroke,fill}%
}%
\begin{pgfscope}%
\pgfsys@transformshift{0.703703in}{2.781919in}%
\pgfsys@useobject{currentmarker}{}%
\end{pgfscope}%
\end{pgfscope}%
\begin{pgfscope}%
\pgfsetbuttcap%
\pgfsetroundjoin%
\definecolor{currentfill}{rgb}{0.000000,0.000000,0.000000}%
\pgfsetfillcolor{currentfill}%
\pgfsetlinewidth{0.602250pt}%
\definecolor{currentstroke}{rgb}{0.000000,0.000000,0.000000}%
\pgfsetstrokecolor{currentstroke}%
\pgfsetdash{}{0pt}%
\pgfsys@defobject{currentmarker}{\pgfqpoint{-0.027778in}{0.000000in}}{\pgfqpoint{-0.000000in}{0.000000in}}{%
\pgfpathmoveto{\pgfqpoint{-0.000000in}{0.000000in}}%
\pgfpathlineto{\pgfqpoint{-0.027778in}{0.000000in}}%
\pgfusepath{stroke,fill}%
}%
\begin{pgfscope}%
\pgfsys@transformshift{0.703703in}{2.815047in}%
\pgfsys@useobject{currentmarker}{}%
\end{pgfscope}%
\end{pgfscope}%
\begin{pgfscope}%
\pgfsetbuttcap%
\pgfsetroundjoin%
\definecolor{currentfill}{rgb}{0.000000,0.000000,0.000000}%
\pgfsetfillcolor{currentfill}%
\pgfsetlinewidth{0.602250pt}%
\definecolor{currentstroke}{rgb}{0.000000,0.000000,0.000000}%
\pgfsetstrokecolor{currentstroke}%
\pgfsetdash{}{0pt}%
\pgfsys@defobject{currentmarker}{\pgfqpoint{-0.027778in}{0.000000in}}{\pgfqpoint{-0.000000in}{0.000000in}}{%
\pgfpathmoveto{\pgfqpoint{-0.000000in}{0.000000in}}%
\pgfpathlineto{\pgfqpoint{-0.027778in}{0.000000in}}%
\pgfusepath{stroke,fill}%
}%
\begin{pgfscope}%
\pgfsys@transformshift{0.703703in}{3.039639in}%
\pgfsys@useobject{currentmarker}{}%
\end{pgfscope}%
\end{pgfscope}%
\begin{pgfscope}%
\pgfsetbuttcap%
\pgfsetroundjoin%
\definecolor{currentfill}{rgb}{0.000000,0.000000,0.000000}%
\pgfsetfillcolor{currentfill}%
\pgfsetlinewidth{0.602250pt}%
\definecolor{currentstroke}{rgb}{0.000000,0.000000,0.000000}%
\pgfsetstrokecolor{currentstroke}%
\pgfsetdash{}{0pt}%
\pgfsys@defobject{currentmarker}{\pgfqpoint{-0.027778in}{0.000000in}}{\pgfqpoint{-0.000000in}{0.000000in}}{%
\pgfpathmoveto{\pgfqpoint{-0.000000in}{0.000000in}}%
\pgfpathlineto{\pgfqpoint{-0.027778in}{0.000000in}}%
\pgfusepath{stroke,fill}%
}%
\begin{pgfscope}%
\pgfsys@transformshift{0.703703in}{3.153682in}%
\pgfsys@useobject{currentmarker}{}%
\end{pgfscope}%
\end{pgfscope}%
\begin{pgfscope}%
\pgfsetbuttcap%
\pgfsetroundjoin%
\definecolor{currentfill}{rgb}{0.000000,0.000000,0.000000}%
\pgfsetfillcolor{currentfill}%
\pgfsetlinewidth{0.602250pt}%
\definecolor{currentstroke}{rgb}{0.000000,0.000000,0.000000}%
\pgfsetstrokecolor{currentstroke}%
\pgfsetdash{}{0pt}%
\pgfsys@defobject{currentmarker}{\pgfqpoint{-0.027778in}{0.000000in}}{\pgfqpoint{-0.000000in}{0.000000in}}{%
\pgfpathmoveto{\pgfqpoint{-0.000000in}{0.000000in}}%
\pgfpathlineto{\pgfqpoint{-0.027778in}{0.000000in}}%
\pgfusepath{stroke,fill}%
}%
\begin{pgfscope}%
\pgfsys@transformshift{0.703703in}{3.234596in}%
\pgfsys@useobject{currentmarker}{}%
\end{pgfscope}%
\end{pgfscope}%
\begin{pgfscope}%
\pgfsetbuttcap%
\pgfsetroundjoin%
\definecolor{currentfill}{rgb}{0.000000,0.000000,0.000000}%
\pgfsetfillcolor{currentfill}%
\pgfsetlinewidth{0.602250pt}%
\definecolor{currentstroke}{rgb}{0.000000,0.000000,0.000000}%
\pgfsetstrokecolor{currentstroke}%
\pgfsetdash{}{0pt}%
\pgfsys@defobject{currentmarker}{\pgfqpoint{-0.027778in}{0.000000in}}{\pgfqpoint{-0.000000in}{0.000000in}}{%
\pgfpathmoveto{\pgfqpoint{-0.000000in}{0.000000in}}%
\pgfpathlineto{\pgfqpoint{-0.027778in}{0.000000in}}%
\pgfusepath{stroke,fill}%
}%
\begin{pgfscope}%
\pgfsys@transformshift{0.703703in}{3.297359in}%
\pgfsys@useobject{currentmarker}{}%
\end{pgfscope}%
\end{pgfscope}%
\begin{pgfscope}%
\pgfsetbuttcap%
\pgfsetroundjoin%
\definecolor{currentfill}{rgb}{0.000000,0.000000,0.000000}%
\pgfsetfillcolor{currentfill}%
\pgfsetlinewidth{0.602250pt}%
\definecolor{currentstroke}{rgb}{0.000000,0.000000,0.000000}%
\pgfsetstrokecolor{currentstroke}%
\pgfsetdash{}{0pt}%
\pgfsys@defobject{currentmarker}{\pgfqpoint{-0.027778in}{0.000000in}}{\pgfqpoint{-0.000000in}{0.000000in}}{%
\pgfpathmoveto{\pgfqpoint{-0.000000in}{0.000000in}}%
\pgfpathlineto{\pgfqpoint{-0.027778in}{0.000000in}}%
\pgfusepath{stroke,fill}%
}%
\begin{pgfscope}%
\pgfsys@transformshift{0.703703in}{3.348639in}%
\pgfsys@useobject{currentmarker}{}%
\end{pgfscope}%
\end{pgfscope}%
\begin{pgfscope}%
\pgfsetbuttcap%
\pgfsetroundjoin%
\definecolor{currentfill}{rgb}{0.000000,0.000000,0.000000}%
\pgfsetfillcolor{currentfill}%
\pgfsetlinewidth{0.602250pt}%
\definecolor{currentstroke}{rgb}{0.000000,0.000000,0.000000}%
\pgfsetstrokecolor{currentstroke}%
\pgfsetdash{}{0pt}%
\pgfsys@defobject{currentmarker}{\pgfqpoint{-0.027778in}{0.000000in}}{\pgfqpoint{-0.000000in}{0.000000in}}{%
\pgfpathmoveto{\pgfqpoint{-0.000000in}{0.000000in}}%
\pgfpathlineto{\pgfqpoint{-0.027778in}{0.000000in}}%
\pgfusepath{stroke,fill}%
}%
\begin{pgfscope}%
\pgfsys@transformshift{0.703703in}{3.391996in}%
\pgfsys@useobject{currentmarker}{}%
\end{pgfscope}%
\end{pgfscope}%
\begin{pgfscope}%
\pgfsetbuttcap%
\pgfsetroundjoin%
\definecolor{currentfill}{rgb}{0.000000,0.000000,0.000000}%
\pgfsetfillcolor{currentfill}%
\pgfsetlinewidth{0.602250pt}%
\definecolor{currentstroke}{rgb}{0.000000,0.000000,0.000000}%
\pgfsetstrokecolor{currentstroke}%
\pgfsetdash{}{0pt}%
\pgfsys@defobject{currentmarker}{\pgfqpoint{-0.027778in}{0.000000in}}{\pgfqpoint{-0.000000in}{0.000000in}}{%
\pgfpathmoveto{\pgfqpoint{-0.000000in}{0.000000in}}%
\pgfpathlineto{\pgfqpoint{-0.027778in}{0.000000in}}%
\pgfusepath{stroke,fill}%
}%
\begin{pgfscope}%
\pgfsys@transformshift{0.703703in}{3.429554in}%
\pgfsys@useobject{currentmarker}{}%
\end{pgfscope}%
\end{pgfscope}%
\begin{pgfscope}%
\pgfsetbuttcap%
\pgfsetroundjoin%
\definecolor{currentfill}{rgb}{0.000000,0.000000,0.000000}%
\pgfsetfillcolor{currentfill}%
\pgfsetlinewidth{0.602250pt}%
\definecolor{currentstroke}{rgb}{0.000000,0.000000,0.000000}%
\pgfsetstrokecolor{currentstroke}%
\pgfsetdash{}{0pt}%
\pgfsys@defobject{currentmarker}{\pgfqpoint{-0.027778in}{0.000000in}}{\pgfqpoint{-0.000000in}{0.000000in}}{%
\pgfpathmoveto{\pgfqpoint{-0.000000in}{0.000000in}}%
\pgfpathlineto{\pgfqpoint{-0.027778in}{0.000000in}}%
\pgfusepath{stroke,fill}%
}%
\begin{pgfscope}%
\pgfsys@transformshift{0.703703in}{3.462682in}%
\pgfsys@useobject{currentmarker}{}%
\end{pgfscope}%
\end{pgfscope}%
\begin{pgfscope}%
\pgfsetbuttcap%
\pgfsetroundjoin%
\definecolor{currentfill}{rgb}{0.000000,0.000000,0.000000}%
\pgfsetfillcolor{currentfill}%
\pgfsetlinewidth{0.602250pt}%
\definecolor{currentstroke}{rgb}{0.000000,0.000000,0.000000}%
\pgfsetstrokecolor{currentstroke}%
\pgfsetdash{}{0pt}%
\pgfsys@defobject{currentmarker}{\pgfqpoint{-0.027778in}{0.000000in}}{\pgfqpoint{-0.000000in}{0.000000in}}{%
\pgfpathmoveto{\pgfqpoint{-0.000000in}{0.000000in}}%
\pgfpathlineto{\pgfqpoint{-0.027778in}{0.000000in}}%
\pgfusepath{stroke,fill}%
}%
\begin{pgfscope}%
\pgfsys@transformshift{0.703703in}{3.687273in}%
\pgfsys@useobject{currentmarker}{}%
\end{pgfscope}%
\end{pgfscope}%
\begin{pgfscope}%
\pgfsetbuttcap%
\pgfsetroundjoin%
\definecolor{currentfill}{rgb}{0.000000,0.000000,0.000000}%
\pgfsetfillcolor{currentfill}%
\pgfsetlinewidth{0.602250pt}%
\definecolor{currentstroke}{rgb}{0.000000,0.000000,0.000000}%
\pgfsetstrokecolor{currentstroke}%
\pgfsetdash{}{0pt}%
\pgfsys@defobject{currentmarker}{\pgfqpoint{-0.027778in}{0.000000in}}{\pgfqpoint{-0.000000in}{0.000000in}}{%
\pgfpathmoveto{\pgfqpoint{-0.000000in}{0.000000in}}%
\pgfpathlineto{\pgfqpoint{-0.027778in}{0.000000in}}%
\pgfusepath{stroke,fill}%
}%
\begin{pgfscope}%
\pgfsys@transformshift{0.703703in}{3.801316in}%
\pgfsys@useobject{currentmarker}{}%
\end{pgfscope}%
\end{pgfscope}%
\begin{pgfscope}%
\definecolor{textcolor}{rgb}{0.000000,0.000000,0.000000}%
\pgfsetstrokecolor{textcolor}%
\pgfsetfillcolor{textcolor}%
\pgftext[x=0.262922in,y=2.217639in,,bottom,rotate=90.000000]{\color{textcolor}\rmfamily\fontsize{10.000000}{12.000000}\selectfont System Estimation Error}%
\end{pgfscope}%
\begin{pgfscope}%
\pgfpathrectangle{\pgfqpoint{0.703703in}{0.585278in}}{\pgfqpoint{5.076922in}{3.264722in}}%
\pgfusepath{clip}%
\pgfsetrectcap%
\pgfsetroundjoin%
\pgfsetlinewidth{1.204500pt}%
\definecolor{currentstroke}{rgb}{0.000000,0.000000,1.000000}%
\pgfsetstrokecolor{currentstroke}%
\pgfsetstrokeopacity{0.800000}%
\pgfsetdash{}{0pt}%
\pgfpathmoveto{\pgfqpoint{0.703703in}{3.701604in}}%
\pgfpathlineto{\pgfqpoint{0.753456in}{3.701604in}}%
\pgfpathlineto{\pgfqpoint{0.754980in}{3.612892in}}%
\pgfpathlineto{\pgfqpoint{0.778841in}{3.612892in}}%
\pgfpathlineto{\pgfqpoint{0.780364in}{3.608723in}}%
\pgfpathlineto{\pgfqpoint{0.804226in}{3.608723in}}%
\pgfpathlineto{\pgfqpoint{0.805749in}{3.528242in}}%
\pgfpathlineto{\pgfqpoint{0.829610in}{3.528242in}}%
\pgfpathlineto{\pgfqpoint{0.831133in}{3.259768in}}%
\pgfpathlineto{\pgfqpoint{0.854995in}{3.259768in}}%
\pgfpathlineto{\pgfqpoint{0.856518in}{3.158098in}}%
\pgfpathlineto{\pgfqpoint{0.880380in}{3.158098in}}%
\pgfpathlineto{\pgfqpoint{0.881903in}{3.111642in}}%
\pgfpathlineto{\pgfqpoint{0.905764in}{3.111642in}}%
\pgfpathlineto{\pgfqpoint{0.907287in}{3.052698in}}%
\pgfpathlineto{\pgfqpoint{0.931149in}{3.052698in}}%
\pgfpathlineto{\pgfqpoint{0.932672in}{2.999391in}}%
\pgfpathlineto{\pgfqpoint{0.956533in}{2.999391in}}%
\pgfpathlineto{\pgfqpoint{0.958056in}{2.980532in}}%
\pgfpathlineto{\pgfqpoint{1.007303in}{2.979835in}}%
\pgfpathlineto{\pgfqpoint{1.008826in}{2.908472in}}%
\pgfpathlineto{\pgfqpoint{1.032687in}{2.908472in}}%
\pgfpathlineto{\pgfqpoint{1.034210in}{2.739679in}}%
\pgfpathlineto{\pgfqpoint{1.058072in}{2.739679in}}%
\pgfpathlineto{\pgfqpoint{1.059595in}{2.546079in}}%
\pgfpathlineto{\pgfqpoint{1.083456in}{2.546079in}}%
\pgfpathlineto{\pgfqpoint{1.084979in}{2.356748in}}%
\pgfpathlineto{\pgfqpoint{1.108841in}{2.356748in}}%
\pgfpathlineto{\pgfqpoint{1.110364in}{2.183650in}}%
\pgfpathlineto{\pgfqpoint{1.134226in}{2.183650in}}%
\pgfpathlineto{\pgfqpoint{1.135749in}{2.039956in}}%
\pgfpathlineto{\pgfqpoint{1.159610in}{2.039956in}}%
\pgfpathlineto{\pgfqpoint{1.161133in}{1.917480in}}%
\pgfpathlineto{\pgfqpoint{1.184995in}{1.917480in}}%
\pgfpathlineto{\pgfqpoint{1.186518in}{1.812870in}}%
\pgfpathlineto{\pgfqpoint{1.210379in}{1.812870in}}%
\pgfpathlineto{\pgfqpoint{1.211903in}{1.722461in}}%
\pgfpathlineto{\pgfqpoint{1.235764in}{1.722461in}}%
\pgfpathlineto{\pgfqpoint{1.237287in}{1.642870in}}%
\pgfpathlineto{\pgfqpoint{1.261149in}{1.642870in}}%
\pgfpathlineto{\pgfqpoint{1.262672in}{1.572299in}}%
\pgfpathlineto{\pgfqpoint{1.286533in}{1.572299in}}%
\pgfpathlineto{\pgfqpoint{1.288056in}{1.507487in}}%
\pgfpathlineto{\pgfqpoint{1.311918in}{1.507487in}}%
\pgfpathlineto{\pgfqpoint{1.313441in}{1.447495in}}%
\pgfpathlineto{\pgfqpoint{1.337303in}{1.447495in}}%
\pgfpathlineto{\pgfqpoint{1.338826in}{1.390912in}}%
\pgfpathlineto{\pgfqpoint{1.362687in}{1.390912in}}%
\pgfpathlineto{\pgfqpoint{1.364210in}{1.337699in}}%
\pgfpathlineto{\pgfqpoint{1.388072in}{1.337699in}}%
\pgfpathlineto{\pgfqpoint{1.389595in}{1.287829in}}%
\pgfpathlineto{\pgfqpoint{1.413456in}{1.287829in}}%
\pgfpathlineto{\pgfqpoint{1.414979in}{1.240281in}}%
\pgfpathlineto{\pgfqpoint{1.438841in}{1.240281in}}%
\pgfpathlineto{\pgfqpoint{1.440364in}{1.195751in}}%
\pgfpathlineto{\pgfqpoint{1.464226in}{1.195751in}}%
\pgfpathlineto{\pgfqpoint{1.465749in}{1.153564in}}%
\pgfpathlineto{\pgfqpoint{1.489610in}{1.153564in}}%
\pgfpathlineto{\pgfqpoint{1.491133in}{1.113950in}}%
\pgfpathlineto{\pgfqpoint{1.514995in}{1.113950in}}%
\pgfpathlineto{\pgfqpoint{1.516518in}{1.076384in}}%
\pgfpathlineto{\pgfqpoint{1.540379in}{1.076384in}}%
\pgfpathlineto{\pgfqpoint{1.541902in}{1.042204in}}%
\pgfpathlineto{\pgfqpoint{1.565764in}{1.042204in}}%
\pgfpathlineto{\pgfqpoint{1.567287in}{1.010524in}}%
\pgfpathlineto{\pgfqpoint{1.591149in}{1.010524in}}%
\pgfpathlineto{\pgfqpoint{1.592672in}{0.980683in}}%
\pgfpathlineto{\pgfqpoint{1.616533in}{0.980683in}}%
\pgfpathlineto{\pgfqpoint{1.618056in}{0.954026in}}%
\pgfpathlineto{\pgfqpoint{1.641918in}{0.954026in}}%
\pgfpathlineto{\pgfqpoint{1.643441in}{0.929270in}}%
\pgfpathlineto{\pgfqpoint{1.667302in}{0.929270in}}%
\pgfpathlineto{\pgfqpoint{1.668826in}{0.907524in}}%
\pgfpathlineto{\pgfqpoint{1.692687in}{0.907524in}}%
\pgfpathlineto{\pgfqpoint{1.694210in}{0.887664in}}%
\pgfpathlineto{\pgfqpoint{1.718072in}{0.887664in}}%
\pgfpathlineto{\pgfqpoint{1.719595in}{0.869636in}}%
\pgfpathlineto{\pgfqpoint{1.743456in}{0.869636in}}%
\pgfpathlineto{\pgfqpoint{1.744979in}{0.853731in}}%
\pgfpathlineto{\pgfqpoint{1.768841in}{0.853731in}}%
\pgfpathlineto{\pgfqpoint{1.770364in}{0.839944in}}%
\pgfpathlineto{\pgfqpoint{1.794226in}{0.839944in}}%
\pgfpathlineto{\pgfqpoint{1.795749in}{0.827883in}}%
\pgfpathlineto{\pgfqpoint{1.819610in}{0.827883in}}%
\pgfpathlineto{\pgfqpoint{1.821133in}{0.817275in}}%
\pgfpathlineto{\pgfqpoint{1.844995in}{0.817275in}}%
\pgfpathlineto{\pgfqpoint{1.846518in}{0.808176in}}%
\pgfpathlineto{\pgfqpoint{1.870379in}{0.808176in}}%
\pgfpathlineto{\pgfqpoint{1.871902in}{0.799951in}}%
\pgfpathlineto{\pgfqpoint{1.895764in}{0.799951in}}%
\pgfpathlineto{\pgfqpoint{1.897287in}{0.792677in}}%
\pgfpathlineto{\pgfqpoint{1.921149in}{0.792677in}}%
\pgfpathlineto{\pgfqpoint{1.922672in}{0.786083in}}%
\pgfpathlineto{\pgfqpoint{1.946533in}{0.786083in}}%
\pgfpathlineto{\pgfqpoint{1.948056in}{0.780500in}}%
\pgfpathlineto{\pgfqpoint{1.971918in}{0.780500in}}%
\pgfpathlineto{\pgfqpoint{1.973441in}{0.775920in}}%
\pgfpathlineto{\pgfqpoint{1.997302in}{0.775920in}}%
\pgfpathlineto{\pgfqpoint{1.998826in}{0.771521in}}%
\pgfpathlineto{\pgfqpoint{2.022687in}{0.771521in}}%
\pgfpathlineto{\pgfqpoint{2.024210in}{0.767776in}}%
\pgfpathlineto{\pgfqpoint{2.048072in}{0.767776in}}%
\pgfpathlineto{\pgfqpoint{2.049595in}{0.764506in}}%
\pgfpathlineto{\pgfqpoint{2.073456in}{0.764506in}}%
\pgfpathlineto{\pgfqpoint{2.074979in}{0.761445in}}%
\pgfpathlineto{\pgfqpoint{2.098841in}{0.761445in}}%
\pgfpathlineto{\pgfqpoint{2.100364in}{0.758708in}}%
\pgfpathlineto{\pgfqpoint{2.124225in}{0.758708in}}%
\pgfpathlineto{\pgfqpoint{2.125749in}{0.756311in}}%
\pgfpathlineto{\pgfqpoint{2.149610in}{0.756311in}}%
\pgfpathlineto{\pgfqpoint{2.151133in}{0.754373in}}%
\pgfpathlineto{\pgfqpoint{2.174995in}{0.754373in}}%
\pgfpathlineto{\pgfqpoint{2.176518in}{0.752564in}}%
\pgfpathlineto{\pgfqpoint{2.200379in}{0.752564in}}%
\pgfpathlineto{\pgfqpoint{2.201902in}{0.750877in}}%
\pgfpathlineto{\pgfqpoint{2.225764in}{0.750877in}}%
\pgfpathlineto{\pgfqpoint{2.227287in}{0.749310in}}%
\pgfpathlineto{\pgfqpoint{2.251149in}{0.749310in}}%
\pgfpathlineto{\pgfqpoint{2.252672in}{0.748045in}}%
\pgfpathlineto{\pgfqpoint{2.276533in}{0.748045in}}%
\pgfpathlineto{\pgfqpoint{2.278056in}{0.746871in}}%
\pgfpathlineto{\pgfqpoint{2.327302in}{0.745884in}}%
\pgfpathlineto{\pgfqpoint{2.328825in}{0.744990in}}%
\pgfpathlineto{\pgfqpoint{2.378072in}{0.744026in}}%
\pgfpathlineto{\pgfqpoint{2.379595in}{0.743126in}}%
\pgfpathlineto{\pgfqpoint{2.428841in}{0.742421in}}%
\pgfpathlineto{\pgfqpoint{2.430364in}{0.741704in}}%
\pgfpathlineto{\pgfqpoint{2.504995in}{0.740605in}}%
\pgfpathlineto{\pgfqpoint{2.507025in}{0.740088in}}%
\pgfpathlineto{\pgfqpoint{2.581149in}{0.739144in}}%
\pgfpathlineto{\pgfqpoint{2.583179in}{0.738684in}}%
\pgfpathlineto{\pgfqpoint{2.682687in}{0.737792in}}%
\pgfpathlineto{\pgfqpoint{2.685225in}{0.737533in}}%
\pgfpathlineto{\pgfqpoint{2.834995in}{0.736439in}}%
\pgfpathlineto{\pgfqpoint{2.839056in}{0.736292in}}%
\pgfpathlineto{\pgfqpoint{3.088841in}{0.735204in}}%
\pgfpathlineto{\pgfqpoint{3.096964in}{0.735130in}}%
\pgfpathlineto{\pgfqpoint{4.154994in}{0.734022in}}%
\pgfpathlineto{\pgfqpoint{4.177841in}{0.733997in}}%
\pgfpathlineto{\pgfqpoint{5.780117in}{0.733674in}}%
\pgfpathlineto{\pgfqpoint{5.780117in}{0.733674in}}%
\pgfusepath{stroke}%
\end{pgfscope}%
\begin{pgfscope}%
\pgfpathrectangle{\pgfqpoint{0.703703in}{0.585278in}}{\pgfqpoint{5.076922in}{3.264722in}}%
\pgfusepath{clip}%
\pgfsetrectcap%
\pgfsetroundjoin%
\pgfsetlinewidth{1.204500pt}%
\definecolor{currentstroke}{rgb}{0.000000,0.501961,0.000000}%
\pgfsetstrokecolor{currentstroke}%
\pgfsetstrokeopacity{0.800000}%
\pgfsetdash{}{0pt}%
\pgfpathmoveto{\pgfqpoint{0.703703in}{3.701604in}}%
\pgfpathlineto{\pgfqpoint{0.728072in}{3.701604in}}%
\pgfpathlineto{\pgfqpoint{0.729595in}{3.662303in}}%
\pgfpathlineto{\pgfqpoint{0.778841in}{3.662161in}}%
\pgfpathlineto{\pgfqpoint{0.780364in}{3.611975in}}%
\pgfpathlineto{\pgfqpoint{0.804226in}{3.611975in}}%
\pgfpathlineto{\pgfqpoint{0.805749in}{3.528108in}}%
\pgfpathlineto{\pgfqpoint{0.829610in}{3.528108in}}%
\pgfpathlineto{\pgfqpoint{0.831133in}{3.502687in}}%
\pgfpathlineto{\pgfqpoint{0.854995in}{3.502687in}}%
\pgfpathlineto{\pgfqpoint{0.856518in}{3.371829in}}%
\pgfpathlineto{\pgfqpoint{0.880380in}{3.371829in}}%
\pgfpathlineto{\pgfqpoint{0.881903in}{3.144461in}}%
\pgfpathlineto{\pgfqpoint{0.905764in}{3.144461in}}%
\pgfpathlineto{\pgfqpoint{0.907287in}{2.897323in}}%
\pgfpathlineto{\pgfqpoint{0.931149in}{2.897323in}}%
\pgfpathlineto{\pgfqpoint{0.932672in}{2.706532in}}%
\pgfpathlineto{\pgfqpoint{0.956533in}{2.706532in}}%
\pgfpathlineto{\pgfqpoint{0.958056in}{2.614021in}}%
\pgfpathlineto{\pgfqpoint{0.981918in}{2.614021in}}%
\pgfpathlineto{\pgfqpoint{0.983441in}{2.575550in}}%
\pgfpathlineto{\pgfqpoint{1.007303in}{2.575550in}}%
\pgfpathlineto{\pgfqpoint{1.008826in}{2.546317in}}%
\pgfpathlineto{\pgfqpoint{1.032687in}{2.546317in}}%
\pgfpathlineto{\pgfqpoint{1.034210in}{2.503511in}}%
\pgfpathlineto{\pgfqpoint{1.058072in}{2.503511in}}%
\pgfpathlineto{\pgfqpoint{1.059595in}{2.430993in}}%
\pgfpathlineto{\pgfqpoint{1.083456in}{2.430993in}}%
\pgfpathlineto{\pgfqpoint{1.084979in}{2.326034in}}%
\pgfpathlineto{\pgfqpoint{1.108841in}{2.326034in}}%
\pgfpathlineto{\pgfqpoint{1.110364in}{2.207343in}}%
\pgfpathlineto{\pgfqpoint{1.134226in}{2.207343in}}%
\pgfpathlineto{\pgfqpoint{1.135749in}{2.094595in}}%
\pgfpathlineto{\pgfqpoint{1.159610in}{2.094595in}}%
\pgfpathlineto{\pgfqpoint{1.161133in}{1.995796in}}%
\pgfpathlineto{\pgfqpoint{1.184995in}{1.995796in}}%
\pgfpathlineto{\pgfqpoint{1.186518in}{1.912594in}}%
\pgfpathlineto{\pgfqpoint{1.210379in}{1.912594in}}%
\pgfpathlineto{\pgfqpoint{1.211903in}{1.843978in}}%
\pgfpathlineto{\pgfqpoint{1.235764in}{1.843978in}}%
\pgfpathlineto{\pgfqpoint{1.237287in}{1.786941in}}%
\pgfpathlineto{\pgfqpoint{1.261149in}{1.786941in}}%
\pgfpathlineto{\pgfqpoint{1.262672in}{1.738104in}}%
\pgfpathlineto{\pgfqpoint{1.286533in}{1.738104in}}%
\pgfpathlineto{\pgfqpoint{1.288056in}{1.695517in}}%
\pgfpathlineto{\pgfqpoint{1.311918in}{1.695517in}}%
\pgfpathlineto{\pgfqpoint{1.313441in}{1.656848in}}%
\pgfpathlineto{\pgfqpoint{1.337303in}{1.656848in}}%
\pgfpathlineto{\pgfqpoint{1.338826in}{1.620155in}}%
\pgfpathlineto{\pgfqpoint{1.362687in}{1.620155in}}%
\pgfpathlineto{\pgfqpoint{1.364210in}{1.583119in}}%
\pgfpathlineto{\pgfqpoint{1.388072in}{1.583119in}}%
\pgfpathlineto{\pgfqpoint{1.389595in}{1.546873in}}%
\pgfpathlineto{\pgfqpoint{1.413456in}{1.546873in}}%
\pgfpathlineto{\pgfqpoint{1.414979in}{1.510568in}}%
\pgfpathlineto{\pgfqpoint{1.438841in}{1.510568in}}%
\pgfpathlineto{\pgfqpoint{1.440364in}{1.472110in}}%
\pgfpathlineto{\pgfqpoint{1.464226in}{1.472110in}}%
\pgfpathlineto{\pgfqpoint{1.465749in}{1.433735in}}%
\pgfpathlineto{\pgfqpoint{1.489610in}{1.433735in}}%
\pgfpathlineto{\pgfqpoint{1.491133in}{1.395846in}}%
\pgfpathlineto{\pgfqpoint{1.514995in}{1.395846in}}%
\pgfpathlineto{\pgfqpoint{1.516518in}{1.361112in}}%
\pgfpathlineto{\pgfqpoint{1.540379in}{1.361112in}}%
\pgfpathlineto{\pgfqpoint{1.541902in}{1.329976in}}%
\pgfpathlineto{\pgfqpoint{1.565764in}{1.329976in}}%
\pgfpathlineto{\pgfqpoint{1.567287in}{1.302662in}}%
\pgfpathlineto{\pgfqpoint{1.591149in}{1.302662in}}%
\pgfpathlineto{\pgfqpoint{1.592672in}{1.277318in}}%
\pgfpathlineto{\pgfqpoint{1.616533in}{1.277318in}}%
\pgfpathlineto{\pgfqpoint{1.618056in}{1.256055in}}%
\pgfpathlineto{\pgfqpoint{1.641918in}{1.256055in}}%
\pgfpathlineto{\pgfqpoint{1.643441in}{1.236576in}}%
\pgfpathlineto{\pgfqpoint{1.667302in}{1.236576in}}%
\pgfpathlineto{\pgfqpoint{1.668826in}{1.219359in}}%
\pgfpathlineto{\pgfqpoint{1.692687in}{1.219359in}}%
\pgfpathlineto{\pgfqpoint{1.694210in}{1.205544in}}%
\pgfpathlineto{\pgfqpoint{1.718072in}{1.205544in}}%
\pgfpathlineto{\pgfqpoint{1.719595in}{1.194717in}}%
\pgfpathlineto{\pgfqpoint{1.743456in}{1.194717in}}%
\pgfpathlineto{\pgfqpoint{1.744979in}{1.186276in}}%
\pgfpathlineto{\pgfqpoint{1.768841in}{1.186276in}}%
\pgfpathlineto{\pgfqpoint{1.770364in}{1.178951in}}%
\pgfpathlineto{\pgfqpoint{1.794226in}{1.178951in}}%
\pgfpathlineto{\pgfqpoint{1.795749in}{1.173406in}}%
\pgfpathlineto{\pgfqpoint{1.819610in}{1.173406in}}%
\pgfpathlineto{\pgfqpoint{1.821133in}{1.168756in}}%
\pgfpathlineto{\pgfqpoint{1.844995in}{1.168756in}}%
\pgfpathlineto{\pgfqpoint{1.846518in}{1.165247in}}%
\pgfpathlineto{\pgfqpoint{1.870379in}{1.165247in}}%
\pgfpathlineto{\pgfqpoint{1.871902in}{1.162518in}}%
\pgfpathlineto{\pgfqpoint{1.895764in}{1.162518in}}%
\pgfpathlineto{\pgfqpoint{1.897287in}{1.160215in}}%
\pgfpathlineto{\pgfqpoint{1.921149in}{1.160215in}}%
\pgfpathlineto{\pgfqpoint{1.922672in}{1.158325in}}%
\pgfpathlineto{\pgfqpoint{1.946533in}{1.158325in}}%
\pgfpathlineto{\pgfqpoint{1.948056in}{1.156734in}}%
\pgfpathlineto{\pgfqpoint{1.971918in}{1.156734in}}%
\pgfpathlineto{\pgfqpoint{1.973441in}{1.155432in}}%
\pgfpathlineto{\pgfqpoint{2.022687in}{1.154378in}}%
\pgfpathlineto{\pgfqpoint{2.024210in}{1.153475in}}%
\pgfpathlineto{\pgfqpoint{2.073456in}{1.152703in}}%
\pgfpathlineto{\pgfqpoint{2.074979in}{1.152011in}}%
\pgfpathlineto{\pgfqpoint{2.124225in}{1.151426in}}%
\pgfpathlineto{\pgfqpoint{2.125749in}{1.150877in}}%
\pgfpathlineto{\pgfqpoint{2.200379in}{1.149962in}}%
\pgfpathlineto{\pgfqpoint{2.202918in}{1.149638in}}%
\pgfpathlineto{\pgfqpoint{2.327302in}{1.148587in}}%
\pgfpathlineto{\pgfqpoint{2.330349in}{1.148370in}}%
\pgfpathlineto{\pgfqpoint{2.479610in}{1.147514in}}%
\pgfpathlineto{\pgfqpoint{2.481641in}{1.147176in}}%
\pgfpathlineto{\pgfqpoint{2.758841in}{1.146179in}}%
\pgfpathlineto{\pgfqpoint{2.761887in}{1.145950in}}%
\pgfpathlineto{\pgfqpoint{3.190379in}{1.144843in}}%
\pgfpathlineto{\pgfqpoint{3.194948in}{1.144715in}}%
\pgfpathlineto{\pgfqpoint{3.951918in}{1.143629in}}%
\pgfpathlineto{\pgfqpoint{3.968671in}{1.143595in}}%
\pgfpathlineto{\pgfqpoint{4.891148in}{1.142554in}}%
\pgfpathlineto{\pgfqpoint{4.894194in}{1.142338in}}%
\pgfpathlineto{\pgfqpoint{5.398840in}{1.141236in}}%
\pgfpathlineto{\pgfqpoint{5.428287in}{1.141167in}}%
\pgfpathlineto{\pgfqpoint{5.780117in}{1.140814in}}%
\pgfpathlineto{\pgfqpoint{5.780117in}{1.140814in}}%
\pgfusepath{stroke}%
\end{pgfscope}%
\begin{pgfscope}%
\pgfsetrectcap%
\pgfsetmiterjoin%
\pgfsetlinewidth{0.803000pt}%
\definecolor{currentstroke}{rgb}{0.000000,0.000000,0.000000}%
\pgfsetstrokecolor{currentstroke}%
\pgfsetdash{}{0pt}%
\pgfpathmoveto{\pgfqpoint{0.703703in}{0.585278in}}%
\pgfpathlineto{\pgfqpoint{0.703703in}{3.850000in}}%
\pgfusepath{stroke}%
\end{pgfscope}%
\begin{pgfscope}%
\pgfsetrectcap%
\pgfsetmiterjoin%
\pgfsetlinewidth{0.803000pt}%
\definecolor{currentstroke}{rgb}{0.000000,0.000000,0.000000}%
\pgfsetstrokecolor{currentstroke}%
\pgfsetdash{}{0pt}%
\pgfpathmoveto{\pgfqpoint{5.780625in}{0.585278in}}%
\pgfpathlineto{\pgfqpoint{5.780625in}{3.850000in}}%
\pgfusepath{stroke}%
\end{pgfscope}%
\begin{pgfscope}%
\pgfsetrectcap%
\pgfsetmiterjoin%
\pgfsetlinewidth{0.803000pt}%
\definecolor{currentstroke}{rgb}{0.000000,0.000000,0.000000}%
\pgfsetstrokecolor{currentstroke}%
\pgfsetdash{}{0pt}%
\pgfpathmoveto{\pgfqpoint{0.703703in}{0.585278in}}%
\pgfpathlineto{\pgfqpoint{5.780625in}{0.585278in}}%
\pgfusepath{stroke}%
\end{pgfscope}%
\begin{pgfscope}%
\pgfsetrectcap%
\pgfsetmiterjoin%
\pgfsetlinewidth{0.803000pt}%
\definecolor{currentstroke}{rgb}{0.000000,0.000000,0.000000}%
\pgfsetstrokecolor{currentstroke}%
\pgfsetdash{}{0pt}%
\pgfpathmoveto{\pgfqpoint{0.703703in}{3.850000in}}%
\pgfpathlineto{\pgfqpoint{5.780625in}{3.850000in}}%
\pgfusepath{stroke}%
\end{pgfscope}%
\begin{pgfscope}%
\pgfsetbuttcap%
\pgfsetmiterjoin%
\definecolor{currentfill}{rgb}{1.000000,1.000000,1.000000}%
\pgfsetfillcolor{currentfill}%
\pgfsetfillopacity{0.800000}%
\pgfsetlinewidth{1.003750pt}%
\definecolor{currentstroke}{rgb}{0.800000,0.800000,0.800000}%
\pgfsetstrokecolor{currentstroke}%
\pgfsetstrokeopacity{0.800000}%
\pgfsetdash{}{0pt}%
\pgfpathmoveto{\pgfqpoint{4.500145in}{3.351543in}}%
\pgfpathlineto{\pgfqpoint{5.683403in}{3.351543in}}%
\pgfpathquadraticcurveto{\pgfqpoint{5.711181in}{3.351543in}}{\pgfqpoint{5.711181in}{3.379321in}}%
\pgfpathlineto{\pgfqpoint{5.711181in}{3.752778in}}%
\pgfpathquadraticcurveto{\pgfqpoint{5.711181in}{3.780556in}}{\pgfqpoint{5.683403in}{3.780556in}}%
\pgfpathlineto{\pgfqpoint{4.500145in}{3.780556in}}%
\pgfpathquadraticcurveto{\pgfqpoint{4.472368in}{3.780556in}}{\pgfqpoint{4.472368in}{3.752778in}}%
\pgfpathlineto{\pgfqpoint{4.472368in}{3.379321in}}%
\pgfpathquadraticcurveto{\pgfqpoint{4.472368in}{3.351543in}}{\pgfqpoint{4.500145in}{3.351543in}}%
\pgfpathlineto{\pgfqpoint{4.500145in}{3.351543in}}%
\pgfpathclose%
\pgfusepath{stroke,fill}%
\end{pgfscope}%
\begin{pgfscope}%
\pgfsetrectcap%
\pgfsetroundjoin%
\pgfsetlinewidth{1.204500pt}%
\definecolor{currentstroke}{rgb}{0.000000,0.000000,1.000000}%
\pgfsetstrokecolor{currentstroke}%
\pgfsetstrokeopacity{0.800000}%
\pgfsetdash{}{0pt}%
\pgfpathmoveto{\pgfqpoint{4.527923in}{3.676389in}}%
\pgfpathlineto{\pgfqpoint{4.666812in}{3.676389in}}%
\pgfpathlineto{\pgfqpoint{4.805701in}{3.676389in}}%
\pgfusepath{stroke}%
\end{pgfscope}%
\begin{pgfscope}%
\definecolor{textcolor}{rgb}{0.000000,0.000000,0.000000}%
\pgfsetstrokecolor{textcolor}%
\pgfsetfillcolor{textcolor}%
\pgftext[x=4.916812in,y=3.627778in,left,base]{\color{textcolor}\rmfamily\fontsize{10.000000}{12.000000}\selectfont DD-SDLQR}%
\end{pgfscope}%
\begin{pgfscope}%
\pgfsetrectcap%
\pgfsetroundjoin%
\pgfsetlinewidth{1.204500pt}%
\definecolor{currentstroke}{rgb}{0.000000,0.501961,0.000000}%
\pgfsetstrokecolor{currentstroke}%
\pgfsetstrokeopacity{0.800000}%
\pgfsetdash{}{0pt}%
\pgfpathmoveto{\pgfqpoint{4.527923in}{3.482716in}}%
\pgfpathlineto{\pgfqpoint{4.666812in}{3.482716in}}%
\pgfpathlineto{\pgfqpoint{4.805701in}{3.482716in}}%
\pgfusepath{stroke}%
\end{pgfscope}%
\begin{pgfscope}%
\definecolor{textcolor}{rgb}{0.000000,0.000000,0.000000}%
\pgfsetstrokecolor{textcolor}%
\pgfsetfillcolor{textcolor}%
\pgftext[x=4.916812in,y=3.434105in,left,base]{\color{textcolor}\rmfamily\fontsize{10.000000}{12.000000}\selectfont DD-LQR}%
\end{pgfscope}%
\end{pgfpicture}%
\makeatother%
\endgroup%

  \caption{System estimation error comparison between DD-SDLQR and DD-LQR. The error is computed as the Frobenius norm of the difference between estimated and true system matrices.}
  \label{fig:error}
\end{figure}

\section{Conclusion}

The results demonstrate the effectiveness of the data-driven sampled-data LQR approach in achieving robust control performance without requiring explicit knowledge of the system model.

\end{document}
